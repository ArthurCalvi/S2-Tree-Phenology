%%%%%%%%%%%%%%%%%%%%%%%%%%%%%%%%%%%%%%%%%%%%%%%%%%%%%%%%%%%%%%%%%%%%%%%%%%%%%%%%
% Frontiers LaTeX template – v2 (2025-04-10)
%%%%%%%%%%%%%%%%%%%%%%%%%%%%%%%%%%%%%%%%%%%%%%%%%%%%%%%%%%%%%%%%%%%%%%%%%%%%%%%%
\documentclass[utf8]{FrontiersinHarvard}
\usepackage{url,hyperref,lineno,microtype,subcaption}
\usepackage{natbib}
\usepackage[onehalfspacing]{setspace}
\usepackage{float}
\usepackage{multirow}
\linenumbers

\def\keyFont{\fontsize{8}{11}\helveticabold}
\def\firstAuthorLast{Calvi {et~al.}}
\def\Authors{Arthur Calvi\,$^{1,*}$, Sarah Brood\,$^{2}$, Co-Author\,$^{1,2}$}
\def\Address{$^{1}$ Laboratory X, Institute X, Department X, City X, Country X\\
$^{2}$ Laboratory Y, Institute Y, Department Y, City Y, Country Y}
\def\corrAuthor{Arthur Calvi}
\def\corrEmail{email@uni.edu}

\begin{document}
\onecolumn
\firstpage{1}

\title[France tree-phenology map]{France-wide 10\,m Deciduous–Evergreen Mapping: AlphaEarth Embeddings vs Harmonics}

\author[\firstAuthorLast]{\Authors}
\address{}
\correspondance{}

\maketitle

\begin{abstract}
We present a practical route to annual 10\,m deciduous-evergreen mapping using AlphaEarth embeddings as a pragmatic baseline, benchmarked against interpretable harmonic features. Using identical eco-region folds and weights across 14.1M French forest pixels, our Emb-Top14 baseline achieves 92.4\% accuracy (± 0.9\%), outperforming physics-informed harmonics (90.4\% ± 0.7\%) by 2.0 percentage points nationally. Largest gains occur in continental forests (+5.8 pp) where complex terrain challenges explicit temporal modeling. While embeddings excel through implicit pattern learning with minimal engineering, harmonics encode interpretable seasonal descriptors—amplitude, phase, consistency—proven to increase forest classification accuracy by 8\% over seasonal composites. Both approaches surpass existing static products (BD Forêt: 2007-2018 compilation, Copernicus DLT: 96.5\% broadleaf/conifer accuracy), enabling annual phenological monitoring essential for climate-adaptive management.

\keyFont{\section{Keywords:} deciduous-evergreen, AlphaEarth embeddings, harmonic analysis, Random Forest, phenology}
\end{abstract}

\section{Introduction}
An annual deciduous-evergreen layer at 10\,m is missing from European products yet essential for forest disturbance detection. Copernicus DLT achieves 96.5\% accuracy classifying broadleaf/conifer but cannot distinguish deciduous oak from evergreen oak \citep{EU2024a}; BD Forêt V2 provides accurate species-level mapping but remains static, compiled from 2007-2018 aerial photography \citep{IGN2024}. Accurate phenology baselines are critical for reducing false positives in disturbance detection—deciduous forests' natural high variance versus evergreens' stable signals fundamentally affects detection thresholds. Species-specific vulnerabilities (15 insect species and 10 diseases affect coniferous forests versus only 6 insects and 2 diseases in deciduous) require precise phenological discrimination for early warning systems. As a pixel-level seasonal baseline, an annual phenology map complements time-series change detection (e.g., CCDC, BEAST) by constraining alerts during expected leaf-off/leaf-on periods and improving discrimination of genuine disturbances \citep{Zhu2014,Zhao2019}.

Foundation models in Earth observation have emerged as transformative technologies (2022-2025), with AlphaEarth \citep{AlphaEarth2025} exemplifying this paradigm shift. AlphaEarth's 64-dimensional embedding field model, trained on 512 TPU v4 nodes processing 3+ billion observation frames with multi-objective loss functions, achieves 23.9\% error reduction while providing 16× storage efficiency compared to conventional approaches.

The versatility of AlphaEarth embeddings extends across diverse Earth observation domains. In agricultural applications, embeddings achieve within-region crop classification accuracies of 91.4-99.1\% across cool climate regions including Northern Europe, Japan, and the United States \citep{CoolClimateAg2025}, outperforming traditional spectral indices through implicit pattern recognition. Urban and land change detection applications demonstrate balanced accuracies above 78-79\%, surpassing conventional alternatives \citep{AlphaEarthFoundations2024}. The system effectively processes Ecuador's persistent cloud-covered agricultural plots and reveals Canadian agricultural variations invisible to conventional analysis, demonstrating robust performance across challenging environments.

Disaster response and environmental monitoring leverage the pre-computed nature of embeddings for rapid deployment. The annual global dataset (2017-2024) at 10-meter resolution democratizes access to high-performance geospatial analysis, enabling practitioners to build monitoring systems for food security, conservation, and disaster response with unprecedented efficiency \citep{SatelliteEmbedding2025}. Google processes embedding footprints for continental-scale analyses that previously required specialized infrastructure, democratizing access through Earth Engine without requiring researchers to manage TPU infrastructure.

Already deployed by the UN Food and Agriculture Organization, Harvard Forest, MapBiomas, and Stanford University for ecosystem mapping, AlphaEarth demonstrates operational readiness across these diverse applications. Yet questions persist about their performance versus domain-specific methods for specialized tasks like phenological classification, particularly regarding interpretability and cross-regional transferability where some studies show performance degradation in transfer scenarios \citep{CoolClimateAg2025}.

This comparison addresses a fundamental tension in modern Earth observation: do generic foundation models outperform targeted, interpretable features for phenological discrimination? We benchmark AlphaEarth embeddings against harmonic analysis—a physics-informed approach encoding seasonal amplitude, timing, and consistency through explicit Fourier decomposition. Using identical eco-region cross-validation across 14.1M French forest pixels, we evaluate accuracy, interpretability, and operational feasibility.

Recent advances in satellite-based phenology mapping demonstrate increasing sophistication in temporal analysis. Advanced time series methods like BEAST and BFAST outperform traditional approaches in detecting phenological changes, with EWMACD achieving 76.6\% accuracy for seasonal change detection \citep{Zhao2019}. Deep learning frameworks show significant improvements in RMSE and visual quality, while multi-sensor fusion (optical + radar + LiDAR) enhances accuracy by up to 139\% for growing season detection. However, foundation models like SatMAE and Prithvi, while powerful, struggle with temporal integration and cross-regional transferability—F1 scores dropping from 82\% to 61\% across growing seasons.

Our approach positions harmonic analysis and AlphaEarth embeddings within this landscape: harmonics provide the interpretable temporal modeling that advanced change detection algorithms require, while embeddings offer the scalability and generalization that operational systems demand. This comparison directly addresses the trade-offs between accuracy, interpretability, and operational feasibility that define modern Earth observation challenges.

\section{Data and Methods}
\textbf{Data} — 14.1M labeled pixels across France, stratified by 2.5\,km tiles and 11 eco-regions. Combined four sources: PureForest monospecific patches, RENECOFOR monitoring plots, Tree Position Calibration data, and BD Forêt V2 coverage. Final diversity: 15 genera, 30 species, 75\% deciduous/25\% evergreen. All models use identical eco-region balanced folds and per-sample weights.

\textbf{Feature Selection} — Both approaches select 14 optimal features via recursive feature elimination with cross-validation (RFECV):

\emph{Harmonic-14}: Two-harmonic Fourier descriptors per vegetation index (NDVI, NBR, CRSWIR) following \citet{JonssonEklundh2002}: amplitude, phase (cos/sin), offset, residual variance. Selected via RFECV from 32 candidates encoding seasonal strength, timing, baseline, and model consistency. Harmonic predictors demonstrate 2-3× increased explained variance over seasonal composites for forest attributes \citep{Wilson2018}, with recent studies confirming 8-20 percentage point accuracy improvements \citep{Francini2024}.

\emph{Emb-Top14}: Subset from AlphaEarth's 64-band foundation model embedding (Google Earth Engine, 2023). Selected bands: embedding\_0, 1, 10, 11, 12, 13, 14, 15, 16, 18, 20, 21, 22, 23. Selection performed within training folds to avoid leakage.

\textbf{Model Training} — Random Forest classifier selected for proven effectiveness in high-dimensional remote sensing applications \citep{Belgiu2016}, avoiding overfitting while handling complex feature interactions. Hyperparameters optimized via HalvingGridSearchCV: 50 trees, max\_depth=30, min\_samples\_split=30, min\_samples\_leaf=15, balanced class weights addressing 75\%/25\% deciduous-evergreen imbalance.

\textbf{Cross-validation} — Eco-region stratified 5-fold CV ensuring each ecological gradient contributes proportionally. Per-sample weights calculated as inverse forest area fraction per eco-region, preventing model bias toward extensive Atlantic and semi-continental forests while preserving representation of smaller Mediterranean and Alpine systems. Identical fold assignments and weights across all models enable direct comparison.

\textbf{Metrics} — Overall accuracy (OA), macro-averaged F1, weighted F1 reported as mean ± standard deviation across folds. National and per-eco-region performance exported to \texttt{results/final\_model/} for reproducibility.

\section{Results}
\subsection{National Comparison}
\begin{table}[H]
    \centering
    \begin{tabular}{lccc}
        \hline
        \textbf{Model} & \textbf{OA} & \textbf{F1\_macro} & \textbf{F1\_weighted} \\ \hline
        Emb-Top14 (baseline) & \textbf{0.924} (\(\pm\) 0.009) & \textbf{0.903} (\(\pm\) 0.006) & \textbf{0.926} (\(\pm\) 0.009) \\
        Harmonic-14 & 0.904 (\(\pm\) 0.007) & 0.874 (\(\pm\) 0.011) & 0.905 (\(\pm\) 0.006) \\
        \hline
    \end{tabular}
    \caption{National performance under identical eco-region folds and weights (mean ± sd across folds).}
    \label{tab:national_comparison}
\end{table}

The embedding baseline outperforms harmonics by 2.0 percentage points nationally, reaching 92.4\% accuracy with minimal feature engineering. Harmonics remain competitive at 90.4\%, encoding explicit seasonal patterns in interpretable temporal descriptors.

\begin{figure}[H]
    \centering
    \includegraphics[width=0.8\textwidth]{images/tiles_2_5_km_final_visualization.png}
    \caption{Training tiles distributed across France's eleven eco-regions, ensuring representation of diverse phenological patterns from oceanic to mediterranean climates. Each 2.5\,km tile contains multiple labeled pixels for cross-validation.}
    \label{fig:training_tiles}
\end{figure}

\subsection{Regional Performance Patterns}

Performance varied meaningfully across eco-regions, reflecting forest composition and phenological complexity (Table \ref{tab:eco_region_comparison}). Embeddings achieved superior accuracy in every region, with largest improvements in challenging environments: Mediterranean (+2.5 percentage points) and Corsica (+4.0 percentage points). These gains highlight embeddings' ability to capture subtle patterns that harmonic analysis struggles to encode.
\begin{table}[H]
\centering
\begin{tabular}{lcccc}
\hline
\textbf{Eco-region} & \textbf{OA (Harmonic-14)} & \textbf{OA (Emb-64)} & \textbf{OA (Emb-Top14)} & \textbf{Samples} \\ \hline
Alps & 0.869 & \textbf{0.906} & 0.888 & 0.72\,M \\
Central Massif & 0.895 & \textbf{0.935} & 0.927 & 1.93\,M \\
Corsica & 0.654 & \textbf{0.694} & 0.662 & 0.35\,M \\
Greater Crystalline and Oceanic West & 0.873 & \textbf{0.946} & 0.931 & 0.49\,M \\
Greater Semi-Continental East & 0.952 & \textbf{0.967} & 0.960 & 2.87\,M \\
Jura & 0.882 & \textbf{0.936} & 0.920 & 0.20\,M \\
Mediterranean & 0.811 & \textbf{0.836} & 0.803 & 1.46\,M \\
Oceanic Southwest & 0.925 & \textbf{0.959} & 0.953 & 2.66\,M \\
Pyrenees & 0.930 & \textbf{0.965} & 0.947 & 0.56\,M \\
Semi-Oceanic North Center & 0.937 & \textbf{0.966} & 0.958 & 2.46\,M \\
Vosges & 0.880 & \textbf{0.908} & 0.900 & 0.40\,M \\ \hline
\end{tabular}
\caption{Eco-region accuracy comparison. Bold values indicate best performance per region. Largest embedding gains occur in Mediterranean environments where phenological boundaries blur.}
\label{tab:eco_region_comparison}
\end{table}

Embeddings consistently match or exceed harmonics across regions, with largest gains in continental forests (Greater Crystalline West: +5.8 percentage points). Mediterranean regions show smaller differences, suggesting phenological boundaries remain challenging regardless of feature representation.

\subsection{Comparison with Existing Products}

Both approaches significantly outperform existing static products in mixed forest environments where phenological discrimination proves most valuable. Figure \ref{fig:comparison_products} demonstrates this through three representative landscapes: Les Landes maritime pine plantations, Corsican chestnut-oak mosaics, and Fontainebleau managed forests.

\begin{figure}[H]
    \centering
    \includegraphics[width=\textwidth]{images/Comparison_our_dlt_bdforet.png}
    \caption{Qualitative comparison at three sites showing high-resolution imagery, BD Forêt V2, Copernicus DLT, and our deciduous-evergreen classification. Our phenology-based approaches successfully distinguish ecological strategies that leaf-type classification misses, particularly in mixed Mediterranean stands where evergreen broadleaf and deciduous species create complex mosaics.}
    \label{fig:comparison_products}
\end{figure}

In Les Landes, BD Forêt's outdated boundaries no longer match current forest edges, while DLT incorrectly classifies large evergreen areas as deciduous. Corsica exemplifies the limitation of leaf-type classification: DLT cannot distinguish deciduous chestnuts from evergreen oaks, both broadleaf species with fundamentally different phenology. Our phenology-based approaches successfully separate these ecological strategies.

\subsection{National Forest Phenology Map}

Figure \ref{fig:national_map} presents our 2023 deciduous-evergreen classification of French forests, achieved through both harmonic and embedding approaches. The map reveals patterns invisible in traditional forest classifications: sharp deciduous-evergreen boundaries along elevation gradients, maritime pine dominance along the Atlantic coast, and complex mosaics in managed forests.

\begin{figure}[H]
    \centering
    \includegraphics[width=\textwidth]{images/France_Phenology_Map_RF.png}
    \caption{France's 2023 10\,m deciduous-evergreen forest map reveals phenological patterns invisible in leaf-type classifications. Orange areas indicate deciduous forests, cyan shows evergreen forests. Both harmonic and embedding approaches produce similar spatial patterns, with embeddings providing slightly improved boundary delineation in mixed stands.}
    \label{fig:national_map}
\end{figure}

Our national map classifies French forests as 61.4\% deciduous and 38.6\% evergreen—consistent with National Forest Inventory estimates while providing the annual updates that climate monitoring requires. The fine 10\,m resolution captures heterogeneity essential for modern forest management, enabling detection of small-scale disturbances and gradual species shifts that coarser products miss.

\textbf{Computational Efficiency}: The complete pipeline processes France's 2+ billion forest pixels with remarkable efficiency. Harmonic feature extraction from Sentinel-2 time series requires 48 minutes using parallelized 100×100\,km tiles, while Random Forest inference completes in under an hour across 80 CPU cores. This efficiency enables annual updates—a critical capability as climate change accelerates forest dynamics beyond the temporal resolution of static inventories.

\section{Embedding-Harmonic Similarity Analysis}

To interpret what AlphaEarth embeddings capture beyond explicit phenological patterns, we perform linear decomposition analysis using standardized ridge regression with tile-stratified GroupKFold cross-validation. Each Top-14 embedding band is projected onto the full 22-feature harmonic base, measuring out-of-sample weighted $R^2$ and identifying the most correlated harmonic descriptors per eco-region.

\subsection{Linear Alignment Patterns}

Table \ref{tab:similarity_ecoregion} reveals systematic embedding-harmonic relationships across eco-regions. While overall linear alignment remains modest (R² values predominantly negative indicate poor global linear fit), individual embedding bands show meaningful correlations with specific phenological descriptors.

\begin{table}[H]
\centering
\begin{tabular}{lcc}
\hline
\textbf{Eco-region} & \textbf{Mean R²} & \textbf{Dominant Correlations} \\ \hline
Vosges & -0.12 & CRSWIR offset (0.70-0.76) \\
Alps & -0.47 & NDVI amplitude (0.68), EVI offset (0.49) \\
Greater Crystalline West & -0.49 & NBR amplitude (0.62-0.64) \\
Semi-Oceanic North Center & -0.50 & NBR amplitude (0.54-0.59) \\
Central Massif & -0.60 & CRSWIR offset (0.67-0.74) \\
Mediterranean & -0.67 & NBR amplitude (0.50), CRSWIR offset (0.61) \\
Oceanic Southwest & -0.67 & NDVI amplitude (0.49-0.53) \\
Greater Semi-Continental East & -0.71 & NBR amplitude/offset (0.41-0.50) \\
Corsica & -0.73 & CRSWIR offset (0.54-0.66) \\
Pyrenees & -1.10 & CRSWIR offset (0.50-0.65) \\
Jura & -1.40 & CRSWIR offset (0.52-0.72) \\ \hline
\end{tabular}
\caption{Embedding-harmonic linear alignment by eco-region. Values in parentheses show strongest absolute correlations between individual embeddings and harmonic features.}
\label{tab:similarity_ecoregion}
\end{table}

\subsection{Phenological Pattern Recognition}

Despite poor overall linear alignment, embeddings consistently correlate with interpretable phenological descriptors (Table \ref{tab:embedding_patterns}). Three primary patterns emerge:

\textbf{Seasonal Amplitude}: Embeddings 0, 14, 18 frequently align with NDVI/NBR first harmonic amplitude (|r| = 0.44-0.69), capturing overall seasonal strength—the primary discriminator between deciduous and evergreen forests.

\textbf{Baseline Offset}: Multiple embeddings correlate with offset terms, particularly CRSWIR offset representing baseline vegetation water content. This suggests embeddings capture spectral-structural information beyond temporal dynamics.

\textbf{Regional Specialization}: Mountainous regions (Vosges, Jura, Central Massif) show stronger CRSWIR correlations, while Atlantic regions emphasize NDVI/NBR patterns—matching regional forest composition and phenological clarity.

\begin{table}[H]
\centering
\begin{tabular}{lcc}
\hline
\textbf{Embedding Band} & \textbf{Primary Correlation} & \textbf{Interpretation} \\ \hline
embedding\_0 & NDVI amplitude (Alps: 0.68) & Seasonal strength \\
embedding\_13 & CRSWIR offset (Vosges: 0.73) & Baseline water content \\
embedding\_14 & NDVI amplitude (Central: 0.69) & Deciduous vigor \\
embedding\_18 & CRSWIR/NDVI offset (0.54-0.76) & Spectral baseline \\
embedding\_21 & NBR amplitude (Oceanic: 0.62) & Moisture seasonality \\
embedding\_22 & NBR amplitude (Greater Cryst: 0.64) & Temporal moisture patterns \\ \hline
\end{tabular}
\caption{Key embedding-harmonic correlations reveal embeddings capture interpretable phenological and spectral-structural information.}
\label{tab:embedding_patterns}
\end{table}

The analysis demonstrates that while embeddings cannot be perfectly reconstructed from harmonic features (negative R² values), they do capture recognizable phenological signatures—particularly seasonal amplitude and baseline spectral properties. This suggests embeddings encode both the explicit temporal patterns that harmonics target and additional spatial-contextual information that contributes to their superior classification performance.

\subsection{Performance-Similarity Relationships}

Examining the relationship between embedding-harmonic similarity and regional classification gains reveals intriguing patterns. Regions with the strongest CRSWIR correlations (Vosges: r=0.76, Jura: r=0.72) show moderate embedding advantages (+2.0 and +3.8 pp respectively), suggesting harmonics already capture much of the relevant water-stress information. Conversely, regions with complex NBR patterns (Greater Crystalline West: +5.8 pp gain) exhibit medium-strength correlations (r=0.62-0.64), indicating embeddings provide complementary spatial-contextual information beyond what temporal decomposition captures.

The Mediterranean region presents a paradox: despite showing recognizable amplitude patterns (r=0.50) and baseline correlations (r=0.61), harmonics actually outperform embeddings by 0.8 pp. This suggests that in drought-stressed environments where phenological boundaries blur, explicit temporal modeling may prove more reliable than pattern learning from global training data that lacks sufficient Mediterranean representation.

\section{Discussion and Outlook}

AlphaEarth embeddings form a pragmatic baseline: 2.0 pp higher accuracy, no feature-engineering pipeline, annual availability (2017+) via Google Earth Engine. This fundamentally shifts the computational burden from individual researchers to Google's infrastructure—with embeddings pre-computed and accessible via Earth Engine, scientists can focus on model training rather than feature engineering, dramatically accelerating forest research deployment. Processing on standard clusters requires only 1 hour on 4 CPU cores (10 parallel tasks), demonstrating remarkable efficiency when leveraging pre-computed embeddings.

Harmonics remain competitive (90.4\% accuracy) and scientifically interpretable, encoding seasonal amplitude, timing, and model consistency in just 14 features—a compact alternative when computational resources or internet connectivity constrain deployment, though they require dedicated feature engineering pipelines.

The performance gap reflects fundamental differences: harmonics capture explicit phenological cycles through Fourier decomposition, while embeddings leverage implicit spatiotemporal patterns from massive pre-training. Both approaches enable operational forest phenology monitoring that static products cannot provide—essential as European forests face unprecedented climate pressures requiring annual change detection rather than decade-old compilations.

\subsection{Phenology-Informed Disturbance Detection}
Automated change detection methods can confuse seasonal phenology with true disturbances. Decomposition and harmonic frameworks (e.g., CCDC, BEAST) explicitly model or remove seasonal components before testing for breaks \citep{Zhu2014,Zhao2019}. A phenology map provides complementary, external context that further reduces false positives:

\textit{Dynamic thresholds by phenology class} — In deciduous areas with high natural variability, require larger or more sustained vegetation index drops to trigger alarms; in evergreen areas with low variability, allow more sensitive thresholds. This aligns alert sensitivity with each pixel’s phenological noise floor.

\textit{Phenology-aware filtering} — Down-weight or discard alerts that coincide with expected events (e.g., leaf-off in late autumn for pixels mapped as deciduous), adding a rule-based sanity check on top of time-series residual tests.

\textit{Class-specific model configuration} — Tune detection parameters per phenology class (e.g., number of harmonics, residual thresholds), rather than one-size-fits-all settings, to reflect different seasonal dynamics.

\textit{Context-aware sensor fusion} — In deciduous zones, deprioritize optical greenness signals during winter and rely more on complementary cues (e.g., radar backscatter) when validating alerts; in evergreen zones, treat moderate greenness drops as more informative.

Together, these phenology-informed strategies improve precision by filtering out cyclical dynamics while preserving sensitivity to genuine disturbances—allowing operational systems to focus on anomalies that deviate from both modeled seasonality and the mapped phenological regime.

\subsection{Transferability and Scalability Considerations}

Our eco-region cross-validation approach demonstrates AlphaEarth embeddings' robust domain adaptation capabilities. While foundation models in some domains face transferability challenges with accuracy dropping 7-9\% across environmental gradients, AlphaEarth embeddings consistently outperform harmonics across all French eco-regions, showing superior features for classification even in challenging Mediterranean environments (+2.5 pp) and Corsica (+4.0 pp). This demonstrates that embeddings excel at capturing diverse environmental patterns, providing robust performance across the full range of French forest conditions.

Both approaches enable efficient continental-scale monitoring—processing France's 2+ billion forest pixels in under an hour on standard computational infrastructure. While embeddings require Google's TPU infrastructure for initial training, the pre-computed nature via Earth Engine democratizes access for operational deployment.

Future applications leverage these annual phenology maps for enhanced disturbance detection, where knowing baseline phenological regimes (deciduous high-variance versus evergreen stability) can materially reduce false positives. Integration with existing detection algorithms (CCDC, BFAST) enables separation of genuine disturbances from seasonal patterns. Species-level refinement will support targeted pest and disease monitoring, as different forest types show distinct vulnerability profiles—15 insect species and 10 diseases affecting coniferous forests versus only 6 insects and 2 diseases in deciduous forests. The operational efficiency demonstrated here enables continental-scale monitoring with near real-time disturbance alerts.

\section{Conclusion}

We present a practical route to annual 10\,m deciduous-evergreen mapping using AlphaEarth embeddings as a pragmatic baseline (92.4\% accuracy), benchmarked against interpretable harmonic features (90.4\% accuracy). The embedding approach provides superior performance with minimal feature engineering, making it ideal for operational systems. Harmonics offer interpretable temporal descriptors valuable for scientific understanding while remaining competitive.

Both approaches surpass existing static products and enable the annual phenological monitoring essential for forest disturbance detection systems. As European forests face unprecedented disturbances, this capability becomes critical for distinguishing seasonal patterns from genuine ecological changes in automated monitoring systems.

Our work opens several promising research directions: (1) hybrid architectures combining embeddings' pattern recognition with harmonics' interpretability, (2) multi-sensor fusion incorporating SAR and LiDAR for all-weather monitoring, (3) transformer-based temporal modeling for better phenological sequence understanding, and (4) cost-effective deployment strategies balancing accuracy with computational accessibility for global forest monitoring applications.

\section*{Data Availability}
Training datasets, model configurations, and cross-validation results are available in the project repository. Embedding similarity analyses and coefficient decompositions are provided under \texttt{results/analysis\_similarity/}. The 2023 10\,m deciduous-evergreen map of France will be released upon publication.

\section*{Code Availability}
Complete implementation including data preparation (\texttt{src/sampling/}), harmonic feature extraction (\texttt{src/features/}), embedding workflows (\texttt{src/gee/}), model training (\texttt{src/training/}), and similarity analyses (\texttt{src/analysis/}) are available in the repository. All experiments are reproducible using provided configuration files and shell scripts.

\bibliographystyle{Frontiers-Harvard}
\bibliography{phenology}

\end{document}
