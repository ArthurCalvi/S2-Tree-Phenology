%%%%%%%%%%%%%%%%%%%%%%%%%%%%%%%%%%%%%%%%%%%%%%%%%%%%%%%%%%%%%%%%%%%%%%%%%%%%%%%%
% Frontiers LaTeX template – v2 (2025-04-10)
%%%%%%%%%%%%%%%%%%%%%%%%%%%%%%%%%%%%%%%%%%%%%%%%%%%%%%%%%%%%%%%%%%%%%%%%%%%%%%%%
\documentclass[utf8]{FrontiersinHarvard}
\usepackage{url,hyperref,lineno,microtype,subcaption}
\usepackage{natbib}
\usepackage[onehalfspacing]{setspace}
\usepackage{float}
\usepackage{multirow}
\linenumbers

\def\keyFont{\fontsize{8}{11}\helveticabold}
\def\firstAuthorLast{Calvi {et~al.}}
\def\Authors{Arthur Calvi\,$^{1,*}$, Sarah Brood\,$^{2}$, Co-Author\,$^{1,2}$, OpenAI Codex\,$^{3}$, Anthropic Claude Code\,$^{4}$}
\def\Address{$^{1}$ Laboratory X, Institute X, Department X, City X, Country X\\
$^{2}$ Laboratory Y, Institute Y, Department Y, City Y, Country Y\\
$^{3}$ OpenAI, San Francisco, CA, USA\\
$^{4}$ Anthropic, San Francisco, CA, USA}
\def\corrAuthor{Arthur Calvi}
\def\corrEmail{email@uni.edu}

\begin{document}
\onecolumn
\firstpage{1}

\title[France tree-phenology map]{France-wide 10\,m Deciduous–Evergreen Mapping: AlphaEarth Embeddings vs Harmonics}
%je mentionnerai le capteur dans le titre Mapping accuracy for evergreen and deciduous forests using AlphaEarth embeddings vs. sentinel 2 time series. ou Comparing model performances with AlphaEarth embeddings vs. sentinel 2 time series for classifying evergreen and decduous forests pas besoin de mettre ‘in France’

\author[\firstAuthorLast]{\Authors}
\address{}
\correspondance{}

\maketitle

\begin{abstract}
We present an operational route to annua (#a model to map … ), 10\,m deciduous–evergreen mapping for France. We compare two feature families under identical data and evaluation: (i) AlphaEarth foundation‑model embeddings (Emb‑Top14: 14 dimensions selected within folds from the 64‑band annual vectors) and (ii) (#features extracted from time series of Sentinel 2 images fitted with harmonic curves) physics‑informed harmonic descriptors (Harm‑Top14: 14 interpretable indices summarizing seasonal amplitude, timing, offset, and residuals). Using eco‑region–balanced (#rephrase by explaining what is an ecoregion, can be «  the features and weights have been balanced over different forest regions « ) folds and weights across 14.1M forest pixels, embeddings deliver modest, consistent gains (#gains of accuracy + cite the metrics)over harmonics (92.4\% vs 90.4\% OA) together with smoother, more coherent maps. Calibration is strong for both (ECE 0.033 vs 0.059), enabling stable thresholding (#je ne comprends pas cette phrase, la reécrre stp). Agreement with legacy products (DLT, BD Forêt) (# full names and refs) is comparable nationally while revealing expected regional divergences where annual phenology departs from static compilations. Because embeddings are pre‑computed and accessible in Earth Engine, labels—not features—are the practical bottleneck, making a country‑scale, annually updated phenology layer (#it is map of evergreen / deciduous forest types) achievable for downstream disturbance monitoring (#forest type is not necessarily for disturbance monitoring, can remove this part).

\keyFont{\section{Keywords:} deciduous-evergreen, AlphaEarth embeddings, harmonic analysis, Random Forest, phenology}
\end{abstract}

\section{Introduction}

% Teaser figure pinned within Introduction
\begin{figure}[H]
    \centering
    \includegraphics[width=\textwidth]{images/sat_emb_classes.png}
    \caption{South‑east of France: (left) true‑color satellite view; (middle) learned embedding visualization with AlphaEarth dimensions A46, A18, A05 mapped to RGB; (right) classification probabilities where more orange indicates deciduous and more blue indicates evergreen. This teaser illustrates how embeddings accentuate phenological structure that drives the downstream classifier.}
    \label{fig:teaser_sat_emb_class}
\end{figure}

An annually refreshed deciduous–evergreen layer (#forest classification) at 10\,m would help stabilize forest disturbance detection (#be more general, help to monitor forest changes, in particular after disturbances when a forest type can change). Existing continental products illustrate the gap: Copernicus DLT classifies broadleaf versus conifer but cannot separate deciduous from evergreen broadleaf \citep{EU2024a} (#here explain in very few words what they use eg sentinel satellites); BD Forêt V2 offers detailed classes yet remains static (2007–2018) \citep{IGN2024}. Disturbance algorithms (CCDC, BEAST, BFAST, LandTrendr) are sensitive to phenological swings (#periodicity and abrupt changes) \citep{Zhu2014,Zhao2019,Verbesselt2010a,Verbesselt2010b,Kennedy2010,Kennedy2018} (#accurate ?); a current phenology prior would let analysts tighten thresholds for deciduous pixels (high natural variance) and relax them for evergreen stands (low variance), reducing false alarms while highlighting genuine change.

Foundation models provide a practical way to build such priors. Recent efforts (SatMAE, Prithvi-EO, AlphaEarth) train on multi-temporal satellite imagery and transfer well to downstream tasks \citep{Cong2022,Szwarcman2024PrithviEO2,AlphaEarth2025} (#explain main satellite and other data used). AlphaEarth, in particular, publishes annual 64-dimensional embeddings at 10\,m. Because these vectors are precomputed and temporally consistent, practitioners can treat them as a ready-made feature stack; the learned spatial context makes them well suited to simple per-pixel classifiers, shifting the bottleneck from feature engineering to assembling reliable labels.

Against this backdrop we compare two compact feature sets under identical data and evaluation: Emb-Top14 (fourteen dimensions selected within cross-validation folds from the 64D AlphaEarth vectors) and Harm-Top14 (fourteen (#how many bands or indices from the satellite) physics-informed descriptors summarizing seasonal amplitude, timing, offset, and residual variance of Sentinel-2 indices). We ask whether embeddings can compete with carefully engineered harmonics while simplifying the pipeline, whether embedding maps remain smoother without additional filtering (#why is smoother derirable for a good classification ?), and whether the resulting confidence scores are stable enough for practical thresholding (#last part is jargon, please rewrite. Better have one sentence per question).

Our processing chain is intentionally simple: tile mainland France into 2.5\,km cells; assemble 14.1M labeled forest pixels with eco-region tags and area-aware weights; compute Emb-Top14 or Harm-Top14 features; train identical Random Forests with eco-region–stratified (#was not explained), tile-grouped cross-validation; and evaluate accuracy, spatial coherence, and confidence, including comparisons to DLT and BD Forêt restricted to forest pixels. (#rewrite and break the sentence into a classical structure. The model chosen for classification, data preparation, training and cross validation + then a new sentence for the comparison with other products)

Our contributions are threefold: (i) a France-wide, annual 10\,m phenology map (#in the whole paper replace « phenology’ » by evergreen-deciduous classification ( you can use PC for phenology classification in case the same word is used many times) trained and evaluated with eco-region–balanced folds and weights (#unclear to readers); (ii) a controlled comparison of Emb-Top14 and Harm-Top14 under identical data and Random Forests (#with the same train and test data), including spatial-coherence and confidence analyses; and (iii) a national comparison with DLT and BD Forêt, with eco-region breakdowns that explain where annual phenology diverges from static products (#our annual PC maps).

Embedding-based approaches have already shown competitive performance across EO tasks (#applications) \citep{Cong2022,Xie2024FoundationEffective}, and pre-trained representations are increasingly used for rapid environmental monitoring (#can you cite here some concrete aplications)\citep{Szwarcman2024PrithviEO2}. Dataset design also matters: phenology-informed, globally uniform pretraining improves ecological transfer (#what does the last word means ?) \citep{Plekhanova2025SSL4Eco}. Nevertheless, questions remain about interpretability and cross-regional reliability \citep{Xie2024FoundationEffective}. By benchmarking embeddings against explicit harmonics under matched conditions (#with the same clasification model set up), we quantify the trade-offs between accuracy, interpretability, and operational feasibility (#not sure this is done in the paper, but see … 

quantiying interpretabilty is diffrent from comparing two models ).

Recent studies reinforce the motivation. Houriez et al. extend LANDFIRE (#is it a land cover ? ) vegetation types into Canada with simple classifiers on AlphaEarth features \citep{Houriez2025AEFDataGen}; a Siamese U-Net on AlphaEarth imagery reports strong burned-area (#was used for mapig burned area ? ) mapping across continents \citep{Seydi2025AlphaEarthBurnedArea}; and national phenology mapping efforts (#is it land cover maps or phenology ?) continue to emphasize temporal descriptors for evergreen discrimination \citep{Inglada2017,Li2023,Low2020,Bolton2020}. Our comparison situates harmonic analysis and AlphaEarth embeddings within this landscape: harmonics provide interpretable seasonal signals useful for scientific inquiry, while embeddings offer scalable, precomputed features that agencies can deploy immediately.

\section{Data and Methods}

\subsection{Dataset composition and sampling}
\section{Methods}

\subsection{Study area and eco-regional stratification}
\label{sec:greco}
We focus on mainland France and Corsica, where temperate, montane, and Mediterranean forest types co-exist within short distances. To preserve ecological diversity in both training and evaluation we rely on the eleven \emph{Grandes Régions Écologiques} (GRECO) defined by the French National Forest Inventory \citep{IGN2013GRECO}. These eco-regions capture the dominant gradients from Atlantic mixed forests through semi-continental plateaus to Mediterranean evergreen stands and Alpine massifs. We partitioned the national forest mask into 639 non-overlapping 2.5\,km\,$\times$\,2.5\,km tiles (Figure~\ref{fig:training_tiles}), retaining tiles that contain at least ten labelled pixels and are separated by 5\,km to limit spatial autocorrelation. Sample weights compensate for the unequal forest area of each eco-region: for region \(r\), the weight \(w_r\) equals the ratio between its effective forest area fraction \(A_r/\sum_{j} A_j\) and the observed sample fraction \(n_r/N\). We normalise the weights so that \(\sum_i w_i = N\); the resulting weights range from 0.89 in densely sampled regions to 1.48 in rare mountainous tiles.

The GRECO framework offers concise descriptors that explain the spatial behaviour of our models. Region A (Grand Ouest cristallin et océanique) is a humid Atlantic bocage of oak–chestnut forests interspersed with post-war conifer plantations and dense hedgerows. Region B (Centre Nord semi-océanique) covers the Paris Basin cuestas with extensive pedunculate/sessile oak and hornbeam on loess plateaus and Scots pine on sandy soils. Region C (Grand Est semi-continental) spans the Ardennes and Lorraine, with beech–fir forests on cool uplands. Region D (Vosges) combines silver fir–beech on steep crystalline slopes, while Region E (Jura) hosts calcareous beech–fir mosaics above karst plateaus. Region F (Sud-Ouest océanique) includes the Landes maritime pine massif and humid mixed stands. Region G (Massif Central) contains montane beech–fir and chestnut belts, Region H (Alps) sharply transitions from deciduous foothills to coniferous subalpine belts, and Region I (Pyrenees) alternates Atlantic and Mediterranean influences over steep gradients. Region J (Mediterranean) is dominated by evergreen holm and cork oak maquis with summer drought stress, and Region K (Corsica) blends lowland maquis with high-elevation Laricio pine forests. These descriptors provide the ecological backdrop for the regional analyses reported in Section~\ref{sec:results} and are summarised in Table~\ref{tab:greco_summary}.

\begin{table}[H]
    \centering
    \small
    \begin{tabular}{p{0.9cm}p{3.2cm}p{4.1cm}p{5.2cm}}
        \toprule
        \textbf{Code} & \textbf{Region} & \textbf{Climate and terrain} & \textbf{Dominant forest structure} \\
        \midrule
        A & Grand Ouest cristallin et océanique & Humid Atlantic plains and low hills, dense hedgerow landscapes & Oak–chestnut coppice with maritime pine and Sitka spruce plantations \\
        B & Centre Nord semi-océanique & Loess plateaus and chalk cuestas of the Paris Basin & Pedunculate/sessile oak with hornbeam/beech; Scots and Corsican pine on sandy soils \\
        C & Grand Est semi-continental & Ardennes and Lorraine uplands with colder winters & Beech–fir and spruce forests on mesic plateaus, mixed oak lowlands \\
        D & Vosges & Steep crystalline range with high precipitation & Silver fir–beech high forests with spruce and Douglas-fir plantings \\
        E & Jura & Limestone plateaus, cool montane climate & Calcareous beech–fir mosaics and mixed spruce on karst slopes \\
        F & Sud-Ouest océanique & Landes coastal plain and Atlantic piedmont & Maritime pine estates interleaved with humid oak, alder, and chestnut stands \\
        G & Massif Central & Volcanic plateaus and valleys with montane climate & Beech–fir belts, chestnut groves, Scots and Douglas pine on poorer soils \\
        H & Alps & Strong elevational gradient, deep glacial valleys & Deciduous foothills (oak, beech) grading to spruce–larch–fir subalpine belts \\
        I & Pyrenees & Atlantic–Mediterranean transition, steep valleys & Oak/beech foothills with fir-spruce upper slopes and Mediterranean pine on southern flanks \\
        J & Mediterranean & Coastal ranges and plateaus with summer drought & Evergreen holm and cork oak maquis, Aleppo and maritime pine stands \\
        K & Corsica & Crystalline massif with rugged relief & Lowland evergreen maquis and extensive Laricio pine forests above 900\,m \\
        \bottomrule
    \end{tabular}
    \caption{Summary of the eleven GRECO eco-regions used for sampling and evaluation.}
    \label{tab:greco_summary}
\end{table}

\begin{figure}[H]
    \centering
    \includegraphics[width=0.8\textwidth]{images/tiles_2_5_km_final_visualization.png}
    \caption{Distribution of the 639 eco-region balanced training tiles. Warm colours indicate Mediterranean and montane eco-regions, while cool colours denote Atlantic and semi-continental domains. Each tile contains 10\,m pixels labelled as deciduous or evergreen and is used as an indivisible unit during cross-validation.}
    \label{fig:training_tiles}
\end{figure}

\subsection{Reference labels}
The supervised dataset aggregates 14{,}086{,}937 forest pixels (10\,m resolution) drawn from four complementary sources. (i) PureForest provides 135\,000 lidar-guided patches with expert-verified dominant species, mainly covering monospecific stands in southern France \citep{gaydon2024pureforestlargescaleaeriallidar}. (ii) The RENECOFOR long-term monitoring network supplies plot inventories with tree measurements above 15\,m height collected between 2019 and 2020 \citep{ulrich:hal-03444393}. (iii) The Tree Position Calibration campaign geolocates dominant trees with airborne lidar to refine field GPS positions and species attribution \citep{ONF,IGN_LiDARHD}. (iv) BD Forêt V2 contributes mapped stands ( > 5{,}000\,m$^2$) updated between 2005 and 2019 \citep{IGN2024}. We apply a 100\,m negative buffer to BD Forêt polygons to mitigate edge drift before rasterisation. Each pixel inherits its eco-region and tile identifier; categorical attributes (phenology, genus, species, source, acquisition year) are encoded via consistent look-up tables. Deciduous pixels represent 75.5\,\% of the samples (10{,}639{,}124 pixels), evergreens 24.5\,\% (3{,}447{,}813 pixels). Buffered BD Forêt polygons supply 88.6\,\% of the pixels, while in-situ inventories anchor 11.4\,\%, ensuring we retain authoritative field labels within every eco-region.

\subsection{Feature extraction}
\subsubsection{Sentinel-2 harmonic descriptors}
We reconstruct Sentinel-2 Level-2A surface reflectances from Google Earth Engine monthly MEDIAN composites, applying the s2cloudless cloud-probability mask with a 75\,\% threshold, removing QA60 bits 10 and 11 (opaque and cirrus clouds), and discarding scenes with more than 95\,\% cloudy pixels. No gap filling or scene classification layer is applied. For each tile we derive the vegetation indices NDVI, EVI, NBR, and the SWIR ratio (CRSWIR). The annual signal of each index is fitted with two sinusoidal harmonics through ordinary least squares,
\[
  x(t) = C + \sum_{k=1}^{2} \big[a_k \cos\!\big(2\pi k t/T\big) + b_k \sin\!\big(2\pi k t/T\big)\big], \quad T = 1\,\text{year}.
\]
We recover offsets, amplitudes, phases, and residual variance; phases are transformed into their sine and cosine components to avoid wrap-around discontinuities. Recursive feature elimitation (Section~\ref{subsubsec:rfe}) retained 14 descriptors (Table~\ref{tab:harmonic14}) that balance interpretability and cross-validated accuracy (Supplementary Section~S1). 

\begin{table}[H]
    \centering
    \small
    \begin{tabular}{lp{0.62\textwidth}}
        \hline
        \textbf{Feature} & \textbf{Ecological signal} \\
        \hline
        NDVI first-harmonic amplitude & Seasonal strength of broadleaf greenness pulses \\
        NDVI first-harmonic phase (cosine/sine) & Timing of green-up and senescence transitions \\
        NDVI second-harmonic phase (sine) & Asymmetry between rapid spring and gradual autumn trajectories \\
        NDVI offset & Mean canopy greenness throughout the year \\
        NBR first-harmonic amplitude & Annual moisture and structural contrast between canopies and soil \\
        NBR first-harmonic phase (cosine) & Calendar timing of minimum fuel moisture \\
        NBR second-harmonic phase (cosine) & Secondary moisture cycle in bimodal climates \\
        NBR offset & Average woody biomass signal \\
        NBR residual variance & Short-term disturbance departures from harmonic behaviour \\
        CRSWIR first-harmonic phase (cosine) & Timing of peak shortwave-infrared water stress \\
        CRSWIR second-harmonic phase (cosine) & Recovery pattern of evergreen water content \\
        CRSWIR offset & Mean canopy water and lignin content \\
        CRSWIR residual variance & Fine-scale heterogeneity in SWIR response \\
        \hline
    \end{tabular}
    \caption{Harmonic descriptors retained for the HARM-14 feature set. Phases are represented through sine and cosine components to avoid angular discontinuities.}
    \label{tab:harmonic14}
\end{table}

\subsubsection{AlphaEarth embeddings}
AlphaEarth provides annual 64-dimensional embeddings at 10\,m resolution, learned from multi-sensor stacks (Sentinel-1/2 optical reflectances, Sentinel-1 backscatter, Landsat seasonal composites, GEDI structure, ERA5-Land climate predictors) by masking and reconstructing spatio-temporal tokens \citep{AlphaEarth2025}. The resulting vector can be viewed as a “spectro-temporal fingerprint” of each pixel: it summarises how the site looks across the year, how it relates to its neighbourhood, and how ancillary drivers (elevation, climate) modulate that pattern. Because the embeddings are already computed and distributed through Earth Engine, practitioners can treat them as off-the-shelf features without training deep networks. We retrieved the 2018, 2020, 2022, and 2023 embedding rasters, converted the tiles to Parquet, and removed pixels with non-finite components (mostly coastline artifacts). To align with the harmonic dataset we perform an inner join on tile, row, and column indices so that both feature families use the exact same labelled pixels and eco-region weights. The 14 embedding dimensions retained after feature elimination (Section~\ref{subsubsec:rfe}) define the \textbf{EMB-14} feature set used in training and evaluation.

\subsubsection{Recursive feature elimination}
\label{subsubsec:rfe}
Both feature families are pruned with the same eco-region-balanced recursive elimination. Starting from the full candidate list (all harmonic descriptors after circular transforms or the 64 embedding channels), we train weighted random forests on each of the five spatial folds, rank features by their mean importance, and remove the weakest features at every stage. The elimination schedule removes features successively, ensuring that at least 10 predictors remain. After every pruning step we record macro-F1 and accuracy along with eco-region metrics; the curve plateaus at 14 features for the harmonics. We therefore retain the best-performing 14-feature configuration and refer to it as HARM-14 for harmonics and EMB-14 for embeddings. Alternative top-$K$ lists extracted from the same procedure support the ablation studies summarised in Supplementary Section~S2.

\subsection{Random Forest training and cross-validation}
We train separate random forest classifiers for HARM-14 and EMB-14 using 50 trees, maximum depth 30, minimum samples split 30, and minimum leaf size 15. Class imbalances are mitigated through scikit-learn’s balanced class weights in combination with the eco-region weights described above. Five-fold cross-validation preserves tile integrity: the helper routine groups tiles within each eco-region and assigns entire tiles to folds, ensuring that no validation tile shares pixels with the training subset. The folds are identical for both feature sets, and we record out-of-fold predictions, per-fold metrics, and the aggregated confusion matrix. After cross-validation we retrain each model on the full weighted dataset for downstream inference.

\subsection{Evaluation metrics and derived summaries}
Fold-level metrics include overall accuracy, class-specific precision and recall, macro- and weighted-F1, and the four entries of the confusion matrix. Summary statistics report the mean, standard deviation, and interquartile range (25th and 75th percentiles) across folds. Eco-region metrics average the validation folds within each region using the eco-region weights. Calibration is assessed with 10 equal-width probability bins on the evergreen posterior; we compute the expected calibration error (ECE) and maximum calibration error (MCE) and examine threshold sensitivity at decision thresholds 0.45, 0.50, and 0.55. Spatial coherence is quantified on the final maps by extracting each 2.5\,km tile and measuring (i) edge density, expressed as metres of deciduous–evergreen boundary per km$^2$, and (ii) 8-connected patch density per 100\,km$^2$ for total forest, deciduous, and evergreen classes. Harmonic predictions are also evaluated after applying a $3 \times 3$ median filter to isolate the effect of simple post-processing. Tile-level cross-validation predictions are aggregated to compute accuracy and macro-F1 per tile; we report the difference (EMB minus HARM) and normalise it by the national standard deviation, yielding a $z$-score that categorises tiles into an embedding advantage (\(z \ge +1\sigma\)), rough parity (\(|z| < 1\sigma\)), or a harmonic advantage (\(z \le -1\sigma\)). Pearson and Spearman correlations between tile-level deltas and contextual covariates support the heterogeneity analysis.

\subsection{Ancillary tile context for heterogeneity analysis}
To interpret spatial performance patterns we enrich each tile with topography, climate, soil, and class-composition attributes. Elevation, slope, and aspect statistics (mean, min, max, 90th percentile, standard deviation) are derived from the CGIAR-CSI SRTM v4 digital elevation model at 90\,m resolution \citep{Jarvis2008SRTM}. We summarise ERA5-Land monthly aggregated reanalysis for 2018, 2020, 2022, and 2023 \citep{MunozSabater2021ERA5Land}, computing annual means and extremes of 2\,m temperature, dew point, surface pressure, soil moisture, and total precipitation over each tile. Soil properties combine the OpenLandMap USDA texture mode at 250\,m \citep{Hengl2021OpenLandMap} with SoilGrids 2.0 mean and standard deviation of clay, sand, silt, bulk density, and soil organic carbon for the 0–5\,cm layer \citep{Poggio2021SoilGrids}. Finally, we aggregate the label parquet to count deciduous and evergreen pixels per tile, derive their ratios, and compute the Shannon diversity index over the two classes. The resulting dataset contains 86 covariates that underpin the tile-level heterogeneity analysis presented in the Results.

\subsection{Temporal stability evaluation}
Temporal transfer is evaluated by applying the 2023 Emb-14 random forest to the 2022, 2020, and 2018 embedding datasets. The evaluation script reuses the cross-validation manifest so that each pixel inherits its original fold assignment, enabling direct comparison between cross-validation predictions and cross-year inference. For every evaluated year we emit the same artefacts as during training: overall metrics, fold tables, eco-region breakdowns, per-pixel probabilities, and per-tile metrics. Metrics are summarised relative to the 2023 inference baseline and to the 2023 cross-validation reference to quantify year-to-year drift (Supplementary Tables~S4–S5).

\paragraph{Hypotheses.} The experiments test three hypotheses. \textbf{H1} posits that pre-trained embeddings (EMB-14) match or exceed the accuracy and calibration of physics-informed harmonics (HARM-14) under identical data, folds, and classifiers. \textbf{H2} states that embeddings, by encoding spatial context, deliver smoother classification maps and exhibit systematic links with ancillary drivers such as soils, climate, and forest composition. \textbf{H3} asserts that a random forest trained on 2023 embeddings remains robust when applied to other annual embedding vintages and preserves compatibility with legacy national products.

\subsection{Embedding–harmonic similarity analysis}
To probe the relationship between embeddings and explicit phenology we regress each embedding dimension onto the full set of harmonic descriptors using ridge regression. Within every eco-region we standardise predictors and responses, adopt a GroupKFold split that keeps tiles intact, and select the regularisation strength through RidgeCV with up to five folds. Weighted $R^2$ scores are computed per tile and summarised by eco-region; we also catalogue the normalised absolute ridge coefficients and the harmonic feature with highest Pearson correlation for each embedding component. Detailed coefficients, $R^2$ distributions, and tile-level diagnostics appear in Supplementary Section~S3.

\section{Results}
\label{sec:results}
We present four groups of findings: national accuracy and calibration, eco-regional behaviour, spatial/ancillary heterogeneity, and cross-year robustness, before comparing the maps with existing national products.
\subsection{H1 -- Accuracy and calibration}
To address H1 we compare cross-validated metrics for the harmonic and embedding feature sets under identical folds and models. Table~\ref{tab:regional_performance} reports the eco-regional scores, while Table~\ref{tab:cv_summary} summarises the fold distributions.
\begin{table}[H]
    \centering
    \small
    \begin{tabular}{lrrrrrr}
        \toprule
        \textbf{GRECO region} & $\mathrm{OA}_{\mathrm{H}}$ & $\mathrm{OA}_{\mathrm{E}}$ & $\Delta\mathrm{OA}$ (pp) & $\mathrm{F1}_{\mathrm{H}}$ & $\mathrm{F1}_{\mathrm{E}}$ & $\Delta\mathrm{F1}$ (pp) \\
        \midrule
        France (weighted CV) & 0.904 & \textbf{0.926} & \textbf{+2.3} & 0.874 & \textbf{0.905} & \textbf{+3.1} \\
        Central Massif (C) & 0.895 & 0.928 & +3.3 & 0.887 & 0.923 & +3.6 \\
        Oceanic Southwest (F) & 0.925 & 0.951 & +2.6 & 0.867 & 0.933 & +6.5 \\
        Semi-Oceanic North Centre (B) & 0.937 & 0.956 & +1.9 & 0.856 & 0.899 & +4.3 \\
        Alps (H) & 0.869 & 0.895 & +2.6 & 0.854 & 0.877 & +2.2 \\
        Vosges (D) & 0.880 & 0.907 & +2.7 & 0.840 & 0.875 & +3.5 \\
        Greater Crystalline and Oceanic West (A) & 0.873 & 0.923 & +5.0 & 0.815 & 0.883 & +6.8 \\
        Greater Semi-Continental East (C$^\prime$) & 0.952 & 0.960 & +0.9 & 0.810 & 0.851 & +4.1 \\
        Mediterranean (J) & 0.811 & 0.818 & +0.7 & 0.791 & 0.798 & +0.7 \\
        Jura (E) & 0.882 & 0.921 & +3.8 & 0.775 & 0.794 & +1.9 \\
        Pyrenees (I) & 0.930 & 0.957 & +2.7 & 0.775 & 0.883 & +10.8 \\
        Corsica (K) & 0.654 & 0.675 & +2.1 & 0.554 & 0.642 & +8.8 \\
        \bottomrule
    \end{tabular}
    \caption{Cross-validated overall accuracy (OA) and macro-F1 for the harmonic (H) and embedding (E) feature sets across GRECO eco-regions. Deltas are embedding minus harmonic in percentage points. All folds use identical eco-region weights and tile assignments.}
    \label{tab:regional_performance}
\end{table}

The embedding baseline outperforms harmonics by \(2.3\) percentage points nationally, reaching \(0.926 \pm 0.0059\) accuracy (interquartile range 0.922–0.932) with the same Random Forest architecture. Harmonic features remain competitive but plateau at \(0.904 \pm 0.0059\) accuracy (IQR 0.900–0.903) and a macro-F1 of \(0.874 \pm 0.0099\) (IQR 0.867–0.879). Embeddings also reduce fold-to-fold variance, shrinking the macro-F1 standard deviation from 0.0099 to 0.0027.

\begin{table}[H]
    \centering
    \small
    \begin{tabular}{lcc}
        \toprule
        \textbf{Metric} & \textbf{HARM-14} & \textbf{EMB-14} \\
        \midrule
        Overall accuracy & \(0.904 \pm 0.0059\) [0.900, 0.903] & \(0.926 \pm 0.0059\) [0.922, 0.932] \\
        Macro-F1 & \(0.874 \pm 0.0099\) [0.867, 0.879] & \(0.905 \pm 0.0027\) [0.905, 0.906] \\
        Weighted-F1 & \(0.905 \pm 0.0054\) [0.902, 0.905] & \(0.927 \pm 0.0056\) [0.924, 0.933] \\
        \bottomrule
    \end{tabular}
    \caption{Five-fold cross-validation summary (\( \mu \pm \sigma\)) with interquartile ranges [Q1, Q3]. Embeddings improve all metrics while reducing variance.}
    \label{tab:cv_summary}
\end{table}

\subsection{Regional Performance Patterns}

Performance varies meaningfully across eco-regions, reflecting forest composition and phenological complexity. Table~\ref{tab:regional_performance} shows that EMB-14 delivers the largest macro-F1 gains in Atlantic mosaics (Greater Crystalline and Oceanic West: +6.8\,pp; Oceanic Southwest: +6.5\,pp) and mountainous regions (Pyrenees: +10.8\,pp; Corsica: +8.8\,pp), where mixed stands and steep gradients challenge purely temporal descriptors. These uplifts mirror the GRECO descriptions: the Armorican bocage (Region A) alternates humid oak–chestnut forests with post-war conifer plantations, while the Pyrenean chain (Region I) juxtaposes deciduous foothills with evergreen montane belts within a few hundred metres of elevation. Central Massif and the semi-oceanic North (Regions C and B) each gain about 3–4\,pp, consistent with their heterogeneous oak–hornbeam mosaics and scattered Scots pine estates. Mediterranean forests remain the hardest case: both models converge around \(0.80\) macro-F1 with a slim +0.7\,pp uplift, reflecting the drought-adapted evergreen broadleaf stands of Region J where seasonal amplitude is inherently low.

\subsection{Calibration and threshold stability}

Despite higher accuracy, EMB-14 retains reliable probabilities. The expected calibration error drops from 0.0586 (HARM-14) to 0.0335, and the maximum calibration error halves from 0.184 to 0.114. Macro-F1 remains steady when the decision threshold is shifted within \([0.45,0.55]\): embeddings vary by only 0.003 (0.903–0.906), whereas harmonics drift by 0.009 (0.870–0.879). This robustness simplifies operational tuning, particularly when analysts tighten thresholds for high-variance deciduous regions (Supplementary Figure~S6).

\subsection{H2 -- Spatial coherence and landscape context}
Hypothesis H2 examines whether the embedding representation yields smoother outputs and whether these gains relate to ancillary environmental gradients.

\begin{figure}[H]
    \centering
    \includegraphics[width=\textwidth]{images/France_Map_emb.png}
    \caption{France 2023 deciduous--evergreen map from the embedding model (EMB-14). Orange denotes deciduous, cyan denotes evergreen; the inset shows Corsica. Spatial patterns match major ecological gradients: evergreen dominance along Mediterranean and Atlantic pine regions (Landes) and at higher elevations (Alps, Jura, Vosges), with deciduous prevalence across lowland temperate belts. Compared to hand-designed features, embeddings yield smoother, more coherent patches while preserving sharp transitions.}
    \label{fig:national_map}
\end{figure}

National edge density drops from 6.69 to 2.76\,km\,km\(^{-2}\) when switching from HARM-14 to EMB-14, a 59\,\% reduction. Patch density declines from 6.81\,k to 2.16\,k components per 100\,km² (−68\,%). The largest smoothing occurs in maritime pine mosaics (Greater Crystalline and Oceanic West), where the harmonic map fragments mixed stands into speckled artefacts. EMB-14 also maintains crisp elevational boundaries in the Alps, Jura, and Vosges, while matching National Forest Inventory proportions (64.4\,\% deciduous, 35.6\,\% evergreen; Figure~\ref{fig:national_map}). Table~\ref{tab:coherence_summary} quantifies the national reduction in edge and patch densities.

To visualise these differences across scales we will add Figure~\ref{fig:h2_multiscale_placeholder}, which juxtaposes Sentinel-2 reference chips with the harmonic and embedding predictions for representative Atlantic, montane, and Mediterranean tiles. The forthcoming figure will mirror the hypothesis structure by highlighting locations where spatial coherence delivers measurable gains.

\begin{figure}[H]
    \centering
    \fbox{\parbox[c][8cm][c]{0.9\textwidth}{\centering Placeholder for multi-scale HARM vs EMB comparison figure (H2).}}
    \caption{Placeholder for the H2 comparison figure. The final version will show paired Sentinel-2 imagery with harmonic and embedding predictions across several eco-regions to illustrate spatial coherence and map smoothness.}
    \label{fig:h2_multiscale_placeholder}
\end{figure}

\begin{table}[H]
    \centering
    \small
    \begin{tabular}{lccc}
        \toprule
        \textbf{Metric} & \textbf{HARM-14} & \textbf{EMB-14} & \textbf{Relative change} \\
        \midrule
        Median edge density (km\,km\(^{-2}\)) & 6.69 & 2.76 & −59\,\% \\
        Median patch density (per 100\,km\(^2\)) & 6\,810 & 2\,159 & −68\,\% \\
        \bottomrule
    \end{tabular}
    \caption{National spatial coherence metrics aggregated over the 639 evaluation tiles.}
    \label{tab:coherence_summary}
\end{table}

\subsubsection{Tile-level heterogeneity and ancillary drivers}

Tile-level analysis confirms that most areas behave consistently across feature sets. Of the 612 evaluation tiles, 510 (83\,\%) fall within \(|z| < 1\sigma\) for accuracy differences, 89 tiles (14\,\%) exhibit a marked embedding advantage (mean Δaccuracy +12.6\,pp), and only 13 tiles (2\,\%) favour harmonics (mean −13.0\,pp). Embedding wins concentrate in Atlantic and western eco-regions where clay-rich soils, high deciduous ratios, and cooler/wetter ERA5 conditions prevail; correlations reach \(r=0.17\) with clay fraction, \(r=0.13\) with deciduous share, and \(r=-0.10\) with the ERA5 temperature range. Greater Crystalline and Oceanic West displays the highest share of advantage tiles (41\,\%) followed by Vosges (31\,\%) and Corsica (33\,\%), while the Mediterranean basin records the largest proportion of harmonic-favoured tiles (12\,\%). Table~\ref{tab:tile_buckets} summarises the distribution of the tile buckets, and Eco-region summaries (Supplementary Figure~S7) show that Greater Crystalline and Oceanic West gains \(+9.3\)\,pp average accuracy whereas the Mediterranean averages \(+0.9\)\,pp, reinforcing the idea that spatial context helps most in humid, mixed-stand mosaics and contributes modestly in evergreen-dominated shrublands.

\begin{table}[H]
    \centering
    \small
    \begin{tabular}{lccc}
        \toprule
        \textbf{Tile bucket} & \textbf{Count} & \textbf{Share (\%)} & \textbf{Mean Δaccuracy (pp)} \\
        \midrule
        Embedding advantage (\(z \ge +1\sigma\)) & 89 & 14.5 & +12.6 \\
        Harmonic win (\(z \le -1\sigma\)) & 13 & 2.1 & −13.0 \\
        Rough parity (\(|z| < 1\sigma\)) & 510 & 83.3 & +1.4 \\
        \bottomrule
    \end{tabular}
    \caption{Distribution of tile-level performance buckets derived from cross-validated predictions. Positive values favour embeddings.}
    \label{tab:tile_buckets}
\end{table}

\subsection{H3 -- Temporal robustness and legacy alignment}
Hypothesis H3 evaluates whether the 2023 random forest remains stable across years while maintaining alignment with existing national products.

Applying the 2023 EMB-14 model to earlier AlphaEarth embeddings yields accuracies of 0.940 (2018), 0.938 (2020), and 0.935 (2022). These values sit within 1.5 standard deviations of the 2023 cross-validation mean (0.926) and trail the 2023 inference baseline by 3.8–5.4\,pp in macro-F1. The largest year-on-year degradation occurs in Corsica (Region K, −0.16\,pp macro-F1 in 2018) and along the Mediterranean coast (≈−0.07\,pp), consistent with wildfire disturbances and drought stress recorded in the AlphaEarth provenance. Continental regions—Vosges, Jura, Central Massif—remain within −0.05\,pp. These results indicate that the frozen Random Forest generalises well across annual embeddings, with performance largely governed by the quality of the embedding vintage (Supplementary Tables~S4–S5). Table~\ref{tab:temporal_stability} summarises the year-by-year metrics.

Figure~\ref{fig:h3_temporal_placeholder} (placeholder) will complement these statistics by displaying Sentinel-2 chips from the same patch across 2018, 2020, 2022, and 2023 alongside the corresponding embedding inference, making the temporal drift tangible.

\begin{figure}[H]
    \centering
    \fbox{\parbox[c][8cm][c]{0.9\textwidth}{\centering Placeholder for temporal stability figure (H3) showing multi-year Sentinel-2 imagery with EMB predictions.}}
    \caption{Placeholder for the H3 temporal stability figure. The final panel will present one site observed in 2018, 2020, 2022, and 2023 with the embedding-based predictions to illustrate cross-year consistency.}
    \label{fig:h3_temporal_placeholder}
\end{figure}

\begin{table}[H]
    \centering
    \small
    \begin{tabular}{lcccc}
        \toprule
        \textbf{Year} & \textbf{Source} & \textbf{Accuracy} & \textbf{Macro-F1} & \textbf{ΔF1 vs 2023 (pp)} \\
        \midrule
        2018 & Inference & 0.940 & 0.912 & −3.8 \\
        2020 & Inference & 0.938 & 0.917 & −4.4 \\
        2022 & Inference & 0.935 & 0.905 & −4.3 \\
        2023 & Cross-validation & 0.927 & 0.897 & −5.5 \\
        2023 & Inference baseline & 0.969 & 0.951 & 0.0 \\
        \bottomrule
    \end{tabular}
    \caption{Temporal evaluation of the 2023 random forest applied to AlphaEarth embeddings from different years. All metrics use the identical inference pipeline; deltas are expressed in percentage points relative to the 2023 inference baseline.}
    \label{tab:temporal_stability}
\end{table}

\subsubsection{Legacy products}

When compared against Copernicus DLT (broadleaf vs conifer) and BD Forêt V2 (deciduous vs evergreen polygons), the embedding map matches the harmonic baseline nationally: overall accuracy is 0.628 vs DLT and 0.638 vs BD Forêt, with macro-F1 differences below 0.4 percentage points (Table~\ref{tab:product_comparison_national}). Regional deltas follow the expected pattern—larger disagreements along recently disturbed Atlantic stands and Mediterranean shrublands where reference products either lag in time (BD Forêt) or do not separate evergreen broadleaf (DLT). Embeddings therefore improve internal consistency without breaking compatibility with incumbent datasets (Supplementary Section~S9).

\begin{table}[H]
    \centering
    \small
    \caption{National agreement with existing products (forest pixels only). Metrics derived from aggregated confusion counts over the eco-region grid.}
    \begin{tabular}{lccc}
        \toprule
        \textbf{Comparison} & \textbf{Overall accuracy} & \textbf{Cohen's $\kappa$} & \textbf{Macro-F1} \\
        \midrule
        HARM-14 vs DLT & 0.627 & 0.17 & 0.578 \\
        EMB-14 vs DLT & 0.628 & 0.16 & 0.574 \\
        HARM-14 vs BD Forêt & 0.639 & 0.17 & 0.585 \\
        EMB-14 vs BD Forêt & 0.638 & 0.16 & 0.582 \\
        \bottomrule
    \end{tabular}
    \label{tab:product_comparison_national}
\end{table}

Qualitatively, embeddings reduce speckle in mixed mosaics such as the Landes maritime pine plantations and Corsican Castagniccia chestnut groves, while DLT and BD Forêt retain their strengths in planted conifer estates and mature deciduous stands. These products remain complementary: our annual map reflects the current canopy state, whereas DLT and BD Forêt provide longer-term typologies.

\subsubsection{Embedding–harmonic alignment}

To interpret what the embeddings retain from traditional descriptors we projected each embedding dimension onto the harmonic feature space using ridge regression. Across all GRECO regions the mean out-of-sample \(R^2\) remains negative (from −0.31 in the Alps to −0.90 in the Pyrenees), confirming that no linear combination of harmonic descriptors can reconstruct the embedding activations. In other words, the pre-trained representation captures information that is orthogonal to the explicit sinusoidal components.

Although global reconstruction fails, the normalized ridge coefficients reveal which harmonic families the embeddings lean on. Averaged over regions, almost half of the coefficient mass targets spectral offsets—especially the CRSWIR ratio—while first-harmonic amplitudes account for a further 23\,\%, residual variances 16\,\%, and phase terms 11\,\%. The nearest-neighbour analysis shows that 57 out of 154 embedding dimensions align most strongly with CRSWIR offsets, 33 with NBR first-harmonic amplitudes, and 27 with NDVI amplitudes. This echoes the eco-regional behaviour: mountainous regions (Vosges, Jura, Central Massif) emphasise structural moisture cues (CRSWIR offsets), Atlantic mosaics emphasise moisture-driven amplitude contrasts (NBR amplitude), and Mediterranean/Corsican regions highlight NDVI amplitude differences between evergreen maquis and deciduous stands.

\begin{table}[H]
    \centering
    \small
    \begin{tabular}{lcc}
        \toprule
        \textbf{Region (GRECO)} & \textbf{Mean $R^2$} & \textbf{Dominant harmonic cue} \\
        \midrule
        Alps (H) & −0.31 & NDVI amplitude (seasonal vigor) \\
        Central Massif (C) & −0.32 & CRSWIR offset (baseline moisture) \\
        Corsica (K) & −0.72 & CRSWIR offset (baseline moisture) \\
        Grand Ouest (A) & −0.35 & NBR amplitude (moisture contrast) \\
        Grand Est (C$^\prime$) & −0.56 & NBR offset (woody baseline) \\
        Jura (E) & −0.72 & CRSWIR offset (baseline moisture) \\
        Mediterranean (J) & −0.49 & NBR amplitude (moisture contrast) \\
        Sud-Ouest (F) & −0.87 & NDVI amplitude (seasonal vigor) \\
        Pyrenees (I) & −0.90 & CRSWIR offset (baseline moisture) \\
        Semi-Oceanic North (B) & −0.33 & NBR amplitude (moisture contrast) \\
        Vosges (D) & −0.06 & CRSWIR offset (baseline moisture) \\
        \bottomrule
    \end{tabular}
    \caption{Regional embedding–harmonic alignment. Mean out-of-sample \(R^2\) remains negative, but each region shows a dominant harmonic cue highlighted by the ridge projections.}
    \label{tab:similarity_summary}
\end{table}

Table~\ref{tab:similarity_summary} summarises the regional mean $R^2$ values and dominant cues. The absence of linear reconstruction combined with these alignment patterns supports a hybrid interpretation: AlphaEarth embeddings retain interpretable phenological axes—seasonal vigor and baseline structure—while injecting additional spatial context learned during pre-training. This extra context likely underpins the smoother boundaries and higher accuracy gains observed in mixed Atlantic and montane landscapes, where purely temporal descriptors struggle.

\begin{figure}[H]
    \centering
    \includegraphics[width=\textwidth]{images/Comparison_dlt_bdforet_harmonic_embedding.png}
    \caption{Qualitative comparison across three sites (rows) and five sources (columns: reference Sentinel-2 image, HARM-14, EMB-14, BD Forêt V2, Copernicus DLT). Orange denotes deciduous (broadleaf for DLT); cyan denotes evergreen (conifer for DLT). Landes (local scale) illustrates plantation edges, Corsica--Castagniccia (medium scale) highlights deciduous chestnut versus evergreen oak, and Fontainebleau (regional scale) contrasts managed temperate forests.}
    \label{fig:comparison_products}
\end{figure}

Figure~\ref{fig:comparison_products} highlights a consistent behavior: embeddings encode neighborhood context, yielding smoother, more coherent class patches without isolated pixels, while harmonic features provide crisp pixelwise decisions that can appear speckled at fine scales. Simple post‑processing (e.g., median filtering) narrows this gap for harmonics, but embedding‑based inference attains similar regularization intrinsically.


\section{Discussion and Outlook}

AlphaEarth embeddings \citep{AlphaEarth2025} offer a pragmatic baseline for national phenology mapping. Under identical folds and models, EMB-14 improves national accuracy by 2.3 percentage points over HARM-14 and nearly halves the macro-F1 variance across folds (Table~\ref{tab:cv_summary}). The eco-region breakdown (Table~\ref{tab:regional_performance}) shows that the largest gains occur in Atlantic mosaics and mountain systems where deciduous and evergreen stands mix at short range, confirming Hypothesis~H1 and aligning with the GRECO descriptors in Section~\ref{sec:greco}.

Embedding advantages are spatially selective. The tile buckets in Table~\ref{tab:tile_buckets} and the ancillary correlations reveal that clay-rich, humid regions concentrate the strongest uplifts, whereas Mediterranean shrublands remain challenging and occasionally favour harmonics. The ridge projections (Table~\ref{tab:similarity_summary}) indicate that embeddings still align with interpretable cues—most notably CRSWIR offsets and NBR amplitudes—while capturing additional context that the harmonic basis cannot reconstruct. Figure~\ref{fig:national_map} and Table~\ref{tab:coherence_summary} illustrate how this context translates into markedly smoother maps, supporting Hypothesis~H2.

Calibration quality underpins operational usability. EMB-14 reaches an expected calibration error of 0.033 versus 0.059 for HARM-14, and maintains macro-F1 stability when the decision threshold moves within the \([0.45,0.55]\) range (Supplementary Figure~S6). Combined with the reduced speckle visible in Figure~\ref{fig:comparison_products}, these traits simplify threshold tuning because spatial regularisation emerges from the representation rather than from ad-hoc filtering.

Temporal generalisation (Table~\ref{tab:temporal_stability}) shows that the 2023 random forest transfers to 2018--2022 embeddings with modest macro-F1 degradations (3.8--5.4\,pp) concentrated in fire-affected and drought-prone regions. Compatibility with Copernicus DLT and BD Forêt V2 remains intact nationally (Table~\ref{tab:product_comparison_national}) while exposing the expected regional discrepancies where reference products are outdated or lack evergreen broadleaf separation. The qualitative comparison in Figure~\ref{fig:comparison_products} exemplifies these differences and supports Hypothesis~H3.

Limitations stem from the binary label space and from eco-regions with low seasonal contrast. Mediterranean evergreen broadleaf stands, Corsican maquis, and recently disturbed Atlantic tiles contribute most residual errors. Extending the analysis to multi-class taxa is promising—Supplementary Section~S10 reports genus-level macro-F1 of 0.387—but requires stronger supervision for rare classes. Future work should also test whether explicit topography or climate covariates, now available tile-wise, help close the remaining gaps.

\subsection{Phenology-informed disturbance detection}
Time-series change detectors such as CCDC, BEAST, BFAST, and LandTrendr \citep{Zhu2014,Zhao2019,Verbesselt2010a,Verbesselt2010b,Kennedy2010,Kennedy2018} benefit from accurate priors on seasonal variance. The calibrated probabilities and reduced speckle achieved by EMB-14 (Table~\ref{tab:coherence_summary}; Supplementary Figure~S6) enable class-conditioned thresholds: deciduous pixels can tolerate larger residual excursions while evergreen pixels can trigger alerts at lower magnitudes without inflating false positives. The smoother patches in Figure~\ref{fig:comparison_products} also reduce isolated outliers that commonly confound disturbance filters, making the annual phenology layer a practical companion to seasonal decomposition pipelines.

\subsection{Transferability and scalability considerations}
The inference pipeline scales because AlphaEarth embeddings are pre-computed on Earth Engine and the random forest remains lightweight. Cross-year evaluations (Table~\ref{tab:temporal_stability}) demonstrate that a single model covers contrasting vintages, while the ancillary correlations suggest that remaining errors originate from genuine ecological change rather than numerical drift. Related AlphaEarth applications report similar robustness for burned-area mapping with Siamese U-Net architectures \citep{Seydi2025AlphaEarthBurnedArea} and for cross-border physiognomic mapping \citep{Houriez2025AEFDataGen}. These studies, together with the national evidence presented here, indicate that pre-trained embeddings provide a dependable substrate for continental monitoring; the main bottleneck shifts to updating labels and safeguarding eco-region coverage.


\section{Conclusion}

We present a practical method to annual 10\,m deciduous–evergreen mapping using AlphaEarth embeddings as a pragmatic baseline (92.4\% accuracy), benchmarked against interpretable harmonic features (90.4\% accuracy) (#I woudld give a mean and an IQR across all validation pixels). Embeddings provide modest, consistent gains with minimal feature engineering; harmonics remain competitive and interpretable.

Both approaches enable the annual phenological monitoring essential for disturbance detection systems. Knowing the prevailing phenology (deciduous high variance versus evergreen stability) helps reduce false positives and focus alerts on genuine change.

Future directions include: (i) continuing to use and refine AlphaEarth‑style embeddings as strong per‑pixel features for national mapping; (ii) improving change‑detection algorithms with class‑aware thresholds and temporally consistent priors; and (iii) exploiting embedding geometry (e.g., cosine similarity) to classify and cluster forest disturbances by type and progression.

\section*{Data Availability}
Training data summaries, model configurations, cross-validation splits, and similarity analysis outputs will be made available with the publication. The 2023 10\,m deciduous–evergreen map of France will also be released.

\section*{Code Availability}
Code to reproduce data preparation, feature extraction, model training, evaluation, and similarity analyses will be released upon publication.

\bibliographystyle{Frontiers-Harvard}
\bibliography{phenology}

\end{document}
