%%%%%%%%%%%%%%%%%%%%%%%%%%%%%%%%%%%%%%%%%%%%%%%%%%%%%%%%%%%%%%%%%%%%%%%%%%%%%%%%
% Frontiers LaTeX template – v2 (2025-04-10)
%%%%%%%%%%%%%%%%%%%%%%%%%%%%%%%%%%%%%%%%%%%%%%%%%%%%%%%%%%%%%%%%%%%%%%%%%%%%%%%%
\documentclass[utf8]{FrontiersinHarvard}
\usepackage{url,hyperref,lineno,microtype,subcaption,amsmath}
\usepackage{natbib}
\usepackage[onehalfspacing]{setspace}
\usepackage{float}
\usepackage{multirow}
\linenumbers

\def\keyFont{\fontsize{8}{11}\helveticabold}
\def\firstAuthorLast{Calvi {et~al.}}
\def\Authors{Arthur Calvi\,$^{1,*}$, Sarah Brood\,$^{2}$, Co-Author\,$^{1,2}$, OpenAI Codex\,$^{3}$, Anthropic Claude Code\,$^{4}$}
\def\Address{$^{1}$ Laboratory X, Institute X, Department X, City X, Country X\\
$^{2}$ Laboratory Y, Institute Y, Department Y, City Y, Country Y\\
$^{3}$ OpenAI, San Francisco, CA, USA\\
$^{4}$ Anthropic, San Francisco, CA, USA}
\def\corrAuthor{Arthur Calvi}
\def\corrEmail{email@uni.edu}

\begin{document}
\onecolumn
\firstpage{1}

\title[Towards universal forest mapping]{Towards Universal Forest Mapping: Foundation Embeddings Outperform Hand-Crafted Harmonics for Deciduous–Evergreen Classification}
%je mentionnerai le capteur dans le titre Mapping accuracy for evergreen and deciduous forests using AlphaEarth embeddings vs. sentinel 2 time series. ou Comparing model performances with AlphaEarth embeddings vs. sentinel 2 time series for classifying evergreen and decduous forests pas besoin de mettre ‘in France’

\author[\firstAuthorLast]{\Authors}
\address{}
\correspondance{}

\maketitle

\begin{abstract}
Forest agencies need annual deciduous–evergreen maps to monitor carbon budgets, disturbance recovery, and biodiversity compliance, yet most lack a workflow that scales beyond bespoke campaigns. Existing options demand a compute-or-craft trade-off: either engineer harmonic descriptors with local phenology expertise or train deep networks on GPU fleets.
We evaluate pre-computed AlphaEarth embeddings—the state-of-the-art foundation model now hosted on Google Earth Engine—as a third path that keeps operations lightweight while encoding neighbourhood context that temporal harmonics miss. Using 14.1 million labelled pixels grouped into 639 eco-region-balanced tiles, we train logistic regression, linear support-vector machines, and eco-region-weighted Random Forests on matched 14-dimensional feature sets: physics-informed harmonics (HARM) and AlphaEarth embeddings (EMB).
Across the three learners, EMB lifts accuracy by $2.8 \pm 0.4$ percentage points, macro-F1 by $3.7 \pm 0.5$, halves expected calibration error (0.033 vs 0.059), and reduces national map fragmentation by 68\,\% relative to HARM. A 2023 model transfers to 2018–2022 embeddings with macro IoU between 0.786 and 0.801, showing that the embeddings track inter-annual drift without re-fitting.
By removing bespoke compute and feature engineering, AlphaEarth shifts the bottleneck to label coverage, letting agencies focus on ground truth collection rather than infrastructure.

\keyFont{\section{Keywords:} deciduous-evergreen, AlphaEarth embeddings, harmonic analysis, Random Forest, logistic regression, support vector machine, phenology}
\end{abstract}

\section{Introduction}

% Teaser figure pinned within Introduction
\begin{figure}[H]
    \centering
    \includegraphics[width=\textwidth]{images/sat_emb_classes.png}
    \caption{South‑east of France: (left) true‑color satellite view; (middle) learned embedding visualization (AlphaEarth dimensions A46, A18, A05 mapped to RGB)—notice how embeddings sharpen phenological structure without manual feature design; (right) classification probabilities where orange indicates deciduous and blue evergreen. This teaser previews the central finding: embeddings encode spatial context that harmonics miss.}
    \label{fig:teaser_sat_emb_class}
\end{figure}

High-resolution, annually refreshed deciduous–evergreen maps underpin carbon accounting, disturbance response, and biodiversity reporting \citep{Zhu2014,Zhao2019,Verbesselt2010a,Verbesselt2010b,Kennedy2010,Kennedy2018}. Yet most agencies still rely on legacy inventories or continental products that update slowly and blur key forest types. The Copernicus Dominant Leaf Type map separates broadleaf from conifer but merges evergreen oaks with pines \citep{EU2024a}. BD Forêt V2 provides detailed stand labels yet reflects field campaigns from 2007–2018 \citep{IGN2024}. Delivering a current nationwide map therefore remains a bespoke effort rather than an operational service.

Bespoke projects confront a compute-or-craft dilemma. Physics-informed workflows fit harmonic curves to Sentinel-2 indices, extracting amplitudes, phases, and residual variances that summarise seasonal behaviour \citep{Inglada2017,Li2023,Bolton2020}. The descriptors are interpretable and appealing to domain experts, but the tuning effort scales with every eco-region, sensor update, and disturbance regime.

Deep networks promise automatic feature learning \citep{Low2020,Xie2024FoundationEffective}, yet they demand GPU clusters, large labelled datasets, and machine-learning operations that many agencies cannot maintain. The result is a compute-or-craft bottleneck that still blocks operational deployment.

Foundation embeddings offer a third path by shifting the constraint. Pre-trained models distributed through Google Earth Engine remove both the GPU requirement and the fragility of hand-crafted harmonics, concentrating effort on the one ingredient that cannot be outsourced: ground-truth labels. These embeddings encode spatial neighbourhood context alongside temporal phenology, a capability that traditional harmonic descriptors miss. Figure~\ref{fig:teaser_sat_emb_class} previews how embeddings sharpen structure without manual feature design.

Satellite pretraining initiatives such as SatMAE, Prithvi-EO, and AlphaEarth learn multi-sensor, multi-temporal representations that transfer across tasks \citep{Cong2022,Szwarcman2024PrithviEO2,AlphaEarth2025}. AlphaEarth, a state-of-the-art foundation model, is particularly compelling because it publishes annual 64-dimensional embeddings at 10\,m resolution via Google Earth Engine. The vectors have already supported cross-border vegetation typing and disturbance analyses \citep{Houriez2025AEFDataGen,Seydi2025AlphaEarthBurnedArea}, yet no study has benchmarked them against carefully engineered harmonics under matched conditions.

We address that gap with a France-wide benchmark that evaluates harmonics and embeddings on equal footing. Our dataset aggregates 14.1 million labelled forest pixels from four sources, groups them into 639 non-overlapping 2.5\,km tiles, and balances sampling across the eleven GRECO eco-regions. Two compact 14-dimensional feature sets feed three lightweight classifiers—logistic regression, linear support-vector machines, and eco-region-weighted Random Forests. HARM denotes the physics-informed Sentinel-2 harmonic descriptors, and EMB denotes the fold-selected AlphaEarth subset. Tile-grouped cross-validation measures accuracy and calibration, ancillary drivers explain spatial heterogeneity, temporal experiments test whether a frozen 2023 embedding model transfers to 2018–2022 vintages, and national comparisons situate the outputs against Copernicus DLT and BD Forêt.

We hypothesise that pre-computed embeddings outperform engineered harmonics for deciduous–evergreen mapping while remaining operationally lightweight. We test this claim through accuracy and calibration analyses, spatial coherence and environmental driver assessments, and temporal robustness experiments that include comparisons with national inventories. The remainder of the paper details the data and processing pipeline, reports the results, and concludes with operational guidance and remaining limitations.

\section{Data and Methods}

\subsection{Study area and eco-regional stratification}
\label{sec:greco}
We require spatially representative training data that span the main forest gradients of mainland France and Corsica. Temperate lowlands, montane belts, and Mediterranean shrublands sit within short distances, so we rely on the eleven \emph{Grandes Régions Écologiques} (GRECO) defined by the French National Forest Inventory \citep{IGN2013GRECO}. The GRECO zones distil climate and forest structure transitions from Atlantic mixed forests through semi-continental plateaus to Mediterranean evergreen stands and Alpine massifs.

We partition the national forest mask into 639 non-overlapping 2.5\,km\,$\times$\,2.5\,km tiles (Figure~\ref{fig:training_tiles}). Each tile contains at least ten inventory or plot-labelled pixels and is separated from its neighbours by 5\,km to limit spatial autocorrelation. Once the reference quota is met, remaining forest pixels in each tile are populated with buffered BD Forêt V2 polygons so that both feature families share an identical footprint.

To keep training and evaluation aligned with the true forest extent we apply eco-region weights. For each region \(r\) we compute an effective forest area \(A_r\) by multiplying the GRECO polygons by the mean forest cover ratio reported by the inventory and summing them to obtain \(\sum_j A_j\). With \(N\) labelled pixels overall and \(n_r\) drawn from region \(r\), the per-pixel weight becomes \(w_r = (A_r / \sum_j A_j) / (n_r / N)\). Regions covering 22\,\% of forests but only 21\,\% of pixels, such as the Oceanic Southwest, are scaled by \(w_r \approx 1.06\); over-represented plains drop to \(w_r \approx 0.89\). We normalise the weights so that \(\sum_i w_i = N\), which keeps rare mountain forests from being drowned out by the plains while avoiding over-sampling common stands. The resulting weights range from 0.89 in densely sampled regions to 1.48 in rare mountainous tiles. A small fraction of coastal pixels lies outside the GRECO mask and is excluded from stratified summaries.

The GRECO framework also provides ecological descriptors that contextualise the spatial behaviour reported in Section~\ref{sec:results}. Atlantic bocage and plantation mosaics dominate Regions A and F, semi-continental oak–beech plateaus span Regions B and C, montane belts extend across the Central Massif, Alps, and Pyrenees (Regions G–I), while drought-prone evergreen maquis defines the Mediterranean mainland and Corsica (Regions J and K). Table~\ref{tab:greco_summary} summarises the climate, forest structure, and training sample counts for each region.

\begin{table}[H]
    \centering
    \small
    \begin{tabular}{p{0.8cm}p{3.6cm}p{3.9cm}p{4.6cm}p{1.7cm}}
        \toprule
        \textbf{Code} & \textbf{Region (English name)} & \textbf{Climate and terrain} & \textbf{Dominant forest structure} & \textbf{Training pixels} \\
        \midrule
        A & Greater Crystalline and Oceanic West & Humid Atlantic plains and low hills, dense bocage networks & Oak–chestnut coppice with maritime pine and Sitka spruce plantations & 0.49\,M \\
        B & Semi-Oceanic North Centre & Loess plateaus and chalk cuestas of the Paris Basin & Pedunculate/sessile oak with hornbeam or beech; Scots/Corsican pine on sandy soils & 2.39\,M \\
        C & Greater Semi-Continental East & Ardennes and Lorraine uplands with colder winters & Beech–fir and spruce on mesic plateaus, mixed oak lowlands & 2.80\,M \\
        D & Vosges & Steep crystalline range with high precipitation & Silver fir–beech high forests with spruce and Douglas-fir plantations & 0.39\,M \\
        E & Jura & Limestone plateaus under cool montane climate & Calcareous beech–fir mosaics and mixed spruce on karst slopes & 0.20\,M \\
        F & Oceanic Southwest & Landes coastal plain and Atlantic piedmont & Maritime pine estates interleaved with humid oak, alder, and chestnut stands & 2.91\,M \\
        G & Central Massif & Volcanic plateaus and valleys with montane climate & Beech–fir belts, chestnut groves, Scots/Douglas pine on poorer soils & 1.92\,M \\
        H & Alps & Sharp elevational gradients and deep glacial valleys & Deciduous foothills grading to spruce–larch–fir subalpine belts & 0.65\,M \\
        I & Pyrenees & Atlantic–Mediterranean transition, steep valleys & Oak/beech foothills with fir–spruce upper slopes and Mediterranean pine on south-facing flanks & 0.56\,M \\
        J & Mediterranean & Coastal ranges and plateaus with intense summer drought & Evergreen holm and cork oak maquis, Aleppo and maritime pine stands & 1.43\,M \\
        K & Corsica & Crystalline massif with rugged relief & Lowland evergreen maquis and extensive Laricio pine forests above 900\,m & 0.35\,M \\
        \bottomrule
    \end{tabular}
    \caption{Summary of the eleven GRECO eco-regions used for sampling and evaluation. Training pixel counts correspond to the supervised dataset and sum to 14.1 million samples (rounded to two decimals in millions).}
    \label{tab:greco_summary}
\end{table}

\begin{figure}[H]
    \centering
    \includegraphics[width=0.8\textwidth]{images/tiles_2_5_km_final_visualization.png}
    \caption{Distribution of the 639 eco-region balanced training tiles. Warm colours indicate Mediterranean and montane eco-regions, while cool colours denote Atlantic and semi-continental domains. Each tile contains at least ten labelled pixels, is separated from its neighbours by 5\,km, and serves as an indivisible unit during cross-validation.}
    \label{fig:training_tiles}
\end{figure}

\subsection{Reference labels}
The supervised dataset aggregates 14{,}086{,}937 forest pixels (10\,m resolution) drawn from four complementary sources. (i) PureForest provides 135\,000 lidar-guided patches with expert-verified dominant species, mainly covering monospecific stands in southern France \citep{gaydon2024pureforestlargescaleaeriallidar}. (ii) The RENECOFOR long-term monitoring network supplies plot inventories with tree measurements above 15\,m height collected between 2019 and 2020 \citep{ulrich:hal-03444393}. (iii) The Tree Position Calibration campaign geolocates dominant trees with airborne lidar to refine field GPS positions and species attribution \citep{ONF,IGN_LiDARHD}. (iv) BD Forêt V2 contributes mapped stands ( > 5{,}000\,m$^2$) updated between 2005 and 2019 \citep{IGN2024}. We apply a 100\,m negative buffer to BD Forêt polygons to mitigate edge drift before rasterisation. Each pixel inherits its eco-region and tile identifier; categorical attributes (phenology, genus, species, source, acquisition year) are encoded via consistent look-up tables. Deciduous pixels represent 75.5\,\% of the samples (10{,}639{,}124 pixels), evergreens 24.5\,\% (3{,}447{,}813 pixels). Buffered BD Forêt polygons supply 88.6\,\% of the pixels, while in-situ inventories anchor 11.4\,\%, ensuring we retain authoritative field labels within every eco-region.

\subsection{Feature extraction}
\subsubsection{Sentinel-2 harmonic descriptors}
We reconstruct Sentinel-2 Level-2A surface reflectances from Google Earth Engine monthly MEDIAN composites, applying the s2cloudless cloud-probability mask with a 75\,\% threshold, removing QA60 bits 10 and 11 (opaque and cirrus clouds), and discarding scenes with more than 95\,\% cloudy pixels. No gap filling or scene classification layer is applied. For each tile we derive the vegetation indices NDVI, EVI, NBR, and the SWIR ratio (CRSWIR). The annual signal of each index is fitted with two sinusoidal harmonics through ordinary least squares,
\[
  x(t) = C + \sum_{k=1}^{2} \big[a_k \cos\!\big(2\pi k t/T\big) + b_k \sin\!\big(2\pi k t/T\big)\big], \quad T = 1\,\text{year}.
\]
We recover offsets, amplitudes, phases, and residual variance; phases are transformed into their sine and cosine components to avoid wrap-around discontinuities. These features are purely temporal: each pixel's time series is modelled independently and ignores the surrounding context. Figure~\ref{fig:harmonics_decomposition} illustrates how the two-harmonic model decomposes an NDVI trajectory into interpretable amplitude and phase components while keeping residuals explicit. Recursive feature elimination (Section~\ref{subsubsec:rfe}) retained 14 descriptors that balance interpretability and cross-validated accuracy (Supplementary Section~S1). These descriptors (Table~\ref{tab:harmonic14}) span amplitude, phase, offset, and residual components across four indices: NDVI first-harmonic amplitude captures seasonal greenness pulses; NBR amplitude reflects moisture and structural contrast; CRSWIR offset encodes baseline water content. We denote this 14-dimensional subset as \textbf{HARM}.

\begin{table}[H]
    \centering
    \small
    \begin{tabular}{lp{0.62\textwidth}}
        \hline
        \textbf{HARM descriptor} & \textbf{Ecological signal} \\
        \hline
        NDVI first-harmonic amplitude & Seasonal strength of broadleaf greenness pulses \\
        NDVI first-harmonic phase (cosine/sine) & Timing of green-up and senescence transitions \\
        NDVI second-harmonic phase (sine) & Asymmetry between rapid spring and gradual autumn trajectories \\
        NDVI offset & Mean canopy greenness throughout the year \\
        NBR first-harmonic amplitude & Annual moisture and structural contrast between canopies and soil \\
        NBR first-harmonic phase (cosine) & Calendar timing of minimum fuel moisture \\
        NBR second-harmonic phase (cosine) & Secondary moisture cycle in bimodal climates \\
        NBR offset & Average woody biomass signal \\
        NBR residual variance & Short-term disturbance departures from harmonic behaviour \\
        CRSWIR first-harmonic phase (cosine) & Timing of peak shortwave-infrared water stress \\
        CRSWIR second-harmonic phase (cosine) & Recovery pattern of evergreen water content \\
        CRSWIR offset & Mean canopy water and lignin content \\
        CRSWIR residual variance & Fine-scale heterogeneity in SWIR response \\
        \hline
    \end{tabular}
    \caption{Fourteen harmonic descriptors retained after eco-region-balanced recursive feature elimination. Together they define the \textbf{HARM} feature set used in the Random Forest baseline. Higher values increase seasonal amplitude or evergreen likelihood depending on the index. Phases are represented via sine and cosine components to avoid angular discontinuities.}
    \label{tab:harmonic14}
\end{table}

\begin{figure}[H]
    \centering
    \includegraphics[width=\textwidth]{images/harmonics_decomposition.png}
    \caption{Harmonic decomposition of an NDVI annual curve. The first and second harmonics capture the asymmetric spring green-up and autumn senescence, while the residual component highlights departures linked to disturbance or mixed pixels. These interpretable pieces constitute the descriptors retained in \textbf{HARM}.}
    \label{fig:harmonics_decomposition}
\end{figure}

\subsubsection{AlphaEarth embeddings}
AlphaEarth provides annual 64-dimensional embeddings at 10\,m resolution, learned from multi-sensor stacks (Sentinel-1/2 optical reflectances, Sentinel-1 backscatter, Landsat seasonal composites, GEDI structure, ERA5-Land climate predictors) by masking and reconstructing spatio-temporal tokens \citep{AlphaEarth2025}. Each embedding pixel therefore encodes how a site looks across the seasons and how it relates to its neighbouring pixels; the pre-training injects spatial context that purely temporal harmonics cannot capture. This neighbourhood awareness helps separate mixed forest patches that follow similar temporal curves but diverge in structure. Because the embeddings are already computed and distributed through Earth Engine, practitioners can treat them as off-the-shelf features without training deep networks. We retrieved the 2018, 2020, 2022, and 2023 embedding rasters, converted the tiles to Parquet, and removed pixels with non-finite components (mostly coastline artifacts). To align with the harmonic dataset we perform an inner join on tile, row, and column indices so that both feature families use the exact same labelled pixels and eco-region weights. The 14 embedding dimensions retained after feature elimination (Section~\ref{subsubsec:rfe}) define the \textbf{EMB} feature set used in training and evaluation (Table~\ref{tab:emb14}); they summarise the dominant spatial–seasonal fingerprints encoded by AlphaEarth.

\begin{table}[H]
    \centering
    \small
    \begin{tabular}{ll}
        \hline
        \textbf{Rank} & \textbf{EMB dimension} \\
        \hline
        1 & embedding\_46 \\
        2 & embedding\_18 \\
        3 & embedding\_5 \\
        4 & embedding\_30 \\
        5 & embedding\_39 \\
        6 & embedding\_0 \\
        7 & embedding\_57 \\
        8 & embedding\_23 \\
        9 & embedding\_6 \\
        10 & embedding\_15 \\
        11 & embedding\_13 \\
        12 & embedding\_22 \\
        13 & embedding\_11 \\
        14 & embedding\_24 \\
        \hline
    \end{tabular}
    \caption{Fourteen AlphaEarth embedding dimensions retained after eco-region-balanced recursive feature elimination, ranked by mean importance across folds. Together they define the \textbf{EMB} feature set. Higher values increase evergreen probability in the fitted Random Forest.}
    \label{tab:emb14}
\end{table}

\subsubsection{Recursive feature elimination}
\label{subsubsec:rfe}
Both feature families are pruned with the same eco-region-balanced recursive elimination. Starting from the full harmonic descriptor list (after circular transforms) or the 64 embedding channels, we train weighted Random Forests on each spatial fold, rank predictors by mean importance, and drop the weakest until performance plateaus. The 14-feature configuration maximises macro-F1 for both representations and defines the HARM and EMB subsets used throughout. Supplementary Section~S1 reports the full elimination schedule and ablations for alternative top-$K$ lists.

\subsection{Classifier training, tuning, and cross-validation}
We evaluate three lightweight classifiers on each feature family: logistic regression, linear support-vector machines (SVM), and Random Forests. This choice keeps modelling simple so that any performance gap reflects feature quality rather than model complexity; if embeddings prevail under these conditions, agencies can deploy them without deep learning infrastructure. Every model is trained with class-balanced sample weights on the same five eco-region-balanced folds described above, ensuring that differences in performance arise from the feature representation rather than from split variability.

Hyperparameters are selected with scikit-learn’s successive halving grid search. For each estimator we define a compact grid, evaluate candidates on progressively larger fractions of the harmonic dataset, and reuse the best configuration when fitting on the embeddings. This harmonics-first tuning strategy slightly favours the handcrafted baseline and therefore yields conservative estimates of the embedding gains. Logistic regression combines \texttt{StandardScaler} with an $\ell_2$ or elastic-net penalty (grid over $C\in\{0.01,0.1,1,10\}$ and $l_1$ ratios $\{0,0.5,0.9\}$) using the \texttt{saga} solver. The linear SVM applies the same scaling and explores hinge versus squared-hinge losses with $C\in\{0.01,0.1,1,10\}$. The Random Forest grid matches the original pipeline (50--100 trees, maximum depth 15--None, minimum split 15--60, minimum leaf 10--20), and the selected configuration uses 50 trees, maximum depth 30, minimum split 30, and minimum leaf 15. Logistic regression and the Random Forest provide calibrated probabilities used later for reliability analysis; the linear SVM contributes decision scores only. After hyperparameter selection, each model is retrained on the full weighted dataset (harmonics or embeddings) and the cross-validation artefacts—fold metrics, out-of-fold predictions, and fitted models—are archived for subsequent analysis.

\subsection{Evaluation metrics and derived summaries}
\label{sec:metrics}
Fold-level metrics include overall accuracy, class-specific precision and recall, macro- and weighted-F1, and the four entries of the confusion matrix. Macro-F1 averages the class-specific F1-scores with equal weight so that deciduous and evergreen contribute symmetrically; weighted-F1 applies the empirical class proportions \(n_c/N\) observed in the fold, i.e.
\begin{equation*}
    \mathrm{F1}_{\text{macro}} = \frac{1}{2} \sum_{c \in \{\text{dec}, \text{ever}\}} \mathrm{F1}_c,
    \qquad
    \mathrm{F1}_{\text{weighted}} = \sum_{c \in \{\text{dec}, \text{ever}\}} \frac{n_c}{N} \mathrm{F1}_c,
\end{equation*}
where \(n_c\) is the number of validation pixels of class \(c\) and \(N = n_{\text{dec}} + n_{\text{ever}}\). This averaging prevents overall accuracy from masking poor minority-class performance, particularly for evergreen stands. We report the mean, standard deviation, and interquartile range across folds, while eco-region summaries average validation folds using the eco-region weights. Calibration is assessed with 10 equal-width probability bins on the evergreen posterior, i.e. the random forest probability that a pixel belongs to the evergreen class. For each bin \(b\) with sample set \(S_b\) we compute the mean confidence \(\hat{p}_b = |S_b|^{-1} \sum_{i \in S_b} p_i\) and the observed evergreen fraction \(\hat{y}_b = |S_b|^{-1} \sum_{i \in S_b} \mathbf{1}\{y_i = \text{evergreen}\}\). The expected calibration error is the weighted average of the absolute gaps between these two quantities,
\begin{equation*}
    \mathrm{ECE} = \sum_{b=1}^{10} \frac{|S_b|}{N} \left| \hat{p}_b - \hat{y}_b \right|,
\end{equation*}
with \(N\) the total number of validation pixels, while the maximum calibration error (MCE) records \(\max_b \left| \hat{p}_b - \hat{y}_b \right|\). Low ECE and MCE indicate that stated probabilities match observed frequencies, making the model's confidence trustworthy for threshold tuning. When a model is perfectly calibrated, \(\hat{p}_b\) and \(\hat{y}_b\) coincide in every bin and both metrics vanish. This reliability analysis clarifies how users can adjust the decision threshold: we also track macro-F1 when the evergreen threshold moves to 0.45 or 0.55 to confirm that modest shifts—such as raising the threshold to suppress evergreen false positives—do not destabilise overall performance.

Spatial coherence is evaluated on the final mosaics tile by tile. Edge density measures how many metres of deciduous–evergreen boundary lie within a tile, normalised by tile area (km of edge per km$^2$); fewer edges mean smoother, less fragmented patches. Connected patch density uses an 8-neighbour definition (horizontal, vertical, and diagonal adjacency) to identify every contiguous cluster of forest pixels that share the same class. High patch density therefore signals salt-and-pepper artefacts, while low values indicate coherent stands. We label the binary masks for deciduous and evergreen pixels with a \(3\times3\) structure matrix of ones, yielding component counts \(C_{\text{dec}}\) and \(C_{\text{ever}}\). The corresponding densities are
\begin{equation*}
    D_{\text{dec}} = \frac{100\,C_{\text{dec}}}{A}, \qquad
    D_{\text{ever}} = \frac{100\,C_{\text{ever}}}{A}, \qquad
    D_{\text{tot}} = D_{\text{dec}} + D_{\text{ever}},
\end{equation*}
where \(A\) is the forest area (km$^2$) within the tile. When predictions are smooth, large coherent stands dominate and the component counts remain small, producing low densities; salt-and-pepper artefacts fragment stands into many tiny clusters, inflating \(D_{\text{tot}}\). We also gauge the effect of simple post-processing by applying a \(3 \times 3\) median filter to the harmonic predictions. Tile-level cross-validation predictions yield accuracy and macro-F1 per tile; we convert the embedding-minus-harmonic deltas to $z$-scores relative to the national distribution, categorising tiles into embedding advantage (\(z \ge +1\sigma\)), rough parity (\(|z| < 1\sigma\)), or harmonic advantage (\(z \le -1\sigma\)). Pearson and Spearman correlations between these deltas and contextual covariates underpin the heterogeneity analysis.

\subsection{Ancillary tile context for heterogeneity analysis}
To interpret spatial performance patterns we enrich each tile with topography, climate, soil, and class-composition attributes. Elevation, slope, and aspect statistics (mean, min, max, 90th percentile, standard deviation) are derived from the CGIAR-CSI SRTM v4 digital elevation model at 90\,m resolution \citep{Jarvis2008SRTM}. We summarise ERA5-Land monthly aggregated reanalysis for 2018, 2020, 2022, and 2023 \citep{MunozSabater2021ERA5Land}, computing annual means and extremes of 2\,m temperature, dew point, surface pressure, soil moisture, and total precipitation over each tile. Soil properties combine the OpenLandMap USDA texture mode at 250\,m \citep{Hengl2021OpenLandMap} with SoilGrids 2.0 mean and standard deviation of clay, sand, silt, bulk density, and soil organic carbon for the 0–5\,cm layer \citep{Poggio2021SoilGrids}. Finally, we aggregate the label parquet to count deciduous and evergreen pixels per tile, derive their ratios, and compute the Shannon diversity index over the two classes. The resulting dataset contains 86 covariates that underpin the tile-level heterogeneity analysis presented in the Results.

\subsection{Temporal stability evaluation}
Temporal transfer is evaluated by freezing the 2023 EMB random forest and applying it to the AlphaEarth embedding vintages released for 2022, 2020, and 2018. We reload the cross-validation manifest so that each labelled pixel keeps the fold identifier it had during training. This alignment serves two purposes: (i) it guarantees that the cross-year evaluations use exactly the same supervision set and eco-region weights as the cross-validation runs, and (ii) it allows per-fold comparisons between the original out-of-fold predictions and the cross-year inference.

For every evaluated year we emit the full metric bundle used in training: confusion matrices aggregated over all folds, fold-level tables, eco-region summaries, per-pixel probabilities, and per-tile metrics. Intersection-over-Union (IoU) is derived from the confusion counts for each class \(c \in \{\text{dec}, \text{ever}\}\) as
\begin{equation*}
    \mathrm{IoU}_c = \frac{\mathrm{TP}_c}{\mathrm{TP}_c + \mathrm{FP}_c + \mathrm{FN}_c},
\end{equation*}
with \(\mathrm{TP}_c\), \(\mathrm{FP}_c\), and \(\mathrm{FN}_c\) defined relative to the reference labels (evergreen is the positive class). Macro-IoU averages the two classes with equal weight. IoU quantifies the overlap between the predicted evergreen/deciduous sets and their reference counterparts, complementing the precision/recall perspective. Alongside IoU we report the same accuracy and F1 statistics as for cross-validation; deltas are computed both against the 2023 inference baseline and the 2023 cross-validation reference to characterise year-to-year drift (Supplementary Tables~S4–S5). All artefacts are stored under \texttt{results/evaluation/embeddings\_<year>/}.

\subsection{Embedding–harmonic similarity analysis}
To understand whether embeddings simply compress harmonic information, we project each embedding dimension onto the full harmonic basis using ridge regression. For a given eco-region and embedding channel \(k\), we assemble a response vector \(\mathbf{y}^{(k)} \in \mathbb{R}^n\) with one entry per pixel (standardised to zero mean and unit variance) and a design matrix \(\mathbf{X} \in \mathbb{R}^{n \times p}\) containing the \(p\) harmonic descriptors (also z-scored). Ridge regression estimates coefficients \(\hat{\boldsymbol{\beta}}^{(k)}\) by solving
\begin{equation*}
    \hat{\boldsymbol{\beta}}^{(k)} = \arg\min_{\boldsymbol{\beta}} \left\| \mathbf{y}^{(k)} - \mathbf{X}\boldsymbol{\beta} \right\|_2^2 + \lambda \|\boldsymbol{\beta}\|_2^2,
\end{equation*}
where the regularisation strength \(\lambda\) controls the amount of shrinkage applied to the coefficients. We select \(\lambda\) through RidgeCV on a logarithmic grid (20 values between \(10^{-4}\) and \(10^{4}\)), using a GroupKFold split that keeps tiles intact so that no tile contributes to both training and validation folds. Because \(\lambda\) is merely a stabilising hyperparameter—and different embedding channels prefer different values—we do not report it explicitly in the Results; instead we focus on what the fitted models reveal (variance explained and dominant harmonic cues).

After fitting, we compute tile-level coefficients of determination
\begin{equation*}
    R^2_{t,k} = 1 - \frac{\|\mathbf{y}^{(k)}_t - \mathbf{X}_t \hat{\boldsymbol{\beta}}^{(k)}\|_2^2}{\|\mathbf{y}^{(k)}_t - \bar{y}^{(k)}_t\|_2^2},
\end{equation*}
where the subscript \(t\) denotes the subset of pixels belonging to tile \(t\) and \(\bar{y}^{(k)}_t\) their mean. These $R^2$ values quantify how much of the embedding variance the harmonic basis can reconstruct; negative values indicate that the linear model performs worse than simply predicting the mean. We summarise \(R^2_{t,k}\) across tiles using eco-region weights and across embedding channels by averaging over the 14 dimensions. The reported “mean $R^2$” in Table~\ref{tab:similarity_summary} is therefore the eco-region–weighted average over both tiles and embedding channels.

To interpret the fitted coefficients we normalise \(\hat{\boldsymbol{\beta}}^{(k)}\) by their $\ell_1$ norm and record the harmonic descriptor with the largest absolute weight; we also compute Pearson correlations between \(\mathbf{y}^{(k)}\) and each harmonic feature, retaining the largest-magnitude correlation as a complementary cue. Detailed coefficient tables, $R^2$ distributions, and tile-level diagnostics appear in Supplementary Section~S3.

\section{Results}
\label{sec:results}
We present four groups of findings: national accuracy and calibration, eco-regional behaviour, spatial/ancillary heterogeneity, and cross-year robustness, before comparing the maps with existing national products.
\subsection{Accuracy and calibration}
First, we test whether embeddings match or exceed harmonic accuracy and calibration under identical conditions. The harmonic subset \textbf{HARM} groups amplitudes, phases, offsets, and residual variances from the four Sentinel-2 indices described in Table~\ref{tab:harmonic14}, while the embedding subset \textbf{EMB} retains the 14 AlphaEarth channels that survived the same recursive elimination.

Across the three classifiers, embeddings consistently outperform harmonics (Table~\ref{tab:multi_model_cv}). Accuracy rises by \(2.8 \pm 0.4\) percentage points and macro-F1 by \(3.7 \pm 0.5\) points on average, with every estimator favouring EMB.
\begin{table}[H]
    \centering
    \small
    \begin{tabular}{lcccc}
        \toprule
        \textbf{Estimator} & \textbf{Accuracy (HARM)} & \textbf{Accuracy (EMB)} & \textbf{Macro-F1 (HARM)} & \textbf{Macro-F1 (EMB)} \\
        \midrule
        Logistic regression & 0.883 & \textbf{0.915} & 0.851 & \textbf{0.893} \\
        Linear SVM & 0.886 & \textbf{0.915} & 0.854 & \textbf{0.893} \\
        Random Forest & 0.904 & \textbf{0.927} & 0.874 & \textbf{0.905} \\
        \midrule
        Mean & 0.891 & \textbf{0.919} & 0.860 & \textbf{0.897} \\
        \bottomrule
    \end{tabular}
    \caption{Cross-validated accuracy and macro-F1 for three classifiers trained on harmonic (HARM) and embedding (EMB) feature sets. Higher values are better for both metrics; embeddings consistently win across classifiers even though hyperparameters were tuned on HARM first. Means average the three estimators.}
    \label{tab:multi_model_cv}
\end{table}

These gains hold even though hyperparameters were tuned on harmonics first, a conservative bias toward the baseline.

The Random Forest retains the strongest individual scores, reaching \(0.926 \pm 0.0059\) accuracy (interquartile range 0.922–0.932) compared with \(0.904 \pm 0.0059\) for HARM, and lifting macro-F1 from \(0.874 \pm 0.0099\) to \(0.905 \pm 0.0027\). Embeddings also reduce fold-to-fold variance, shrinking the Random-Forest macro-F1 standard deviation from 0.0099 to 0.0027. These national gains vary meaningfully across eco-regions (Table~\ref{tab:regional_performance}).
\begin{table}[H]
    \centering
    \small
    \begin{tabular}{lrrrrrr}
        \toprule
        \textbf{GRECO region} & $\mathrm{OA}_{\mathrm{H}}$ & $\mathrm{OA}_{\mathrm{E}}$ & $\Delta\mathrm{OA}$ (pp) & $\mathrm{F1}_{\mathrm{H}}$ & $\mathrm{F1}_{\mathrm{E}}$ & $\Delta\mathrm{F1}$ (pp) \\
        \midrule
        France (weighted CV) & 0.904 $\pm$ 0.006 & \textbf{0.926 $\pm$ 0.006} & \textbf{+2.3} & 0.874 $\pm$ 0.010 & \textbf{0.905 $\pm$ 0.003} & \textbf{+3.1} \\
        Central Massif (C) & 0.895 & \textbf{0.928} & \textbf{+3.3} & 0.887 & \textbf{0.923} & \textbf{+3.6} \\
        Oceanic Southwest (F) & 0.925 & \textbf{0.951} & \textbf{+2.6} & 0.867 & \textbf{0.933} & \textbf{+6.5} \\
        Semi-Oceanic North Centre (B) & 0.937 & \textbf{0.956} & \textbf{+1.9} & 0.856 & \textbf{0.899} & \textbf{+4.3} \\
        Alps (H) & 0.869 & \textbf{0.895} & \textbf{+2.6} & 0.854 & \textbf{0.877} & \textbf{+2.2} \\
        Vosges (D) & 0.880 & \textbf{0.907} & \textbf{+2.7} & 0.840 & \textbf{0.875} & \textbf{+3.5} \\
        Greater Crystalline and Oceanic West (A) & 0.873 & \textbf{0.923} & \textbf{+5.0} & 0.815 & \textbf{0.883} & \textbf{+6.8} \\
        Greater Semi-Continental East (C$^\prime$) & 0.952 & \textbf{0.960} & \textbf{+0.9} & 0.810 & \textbf{0.851} & \textbf{+4.1} \\
        Mediterranean (J) & 0.811 & \textbf{0.818} & \textbf{+0.7} & 0.791 & \textbf{0.798} & \textbf{+0.7} \\
        Jura (E) & 0.882 & \textbf{0.921} & \textbf{+3.8} & 0.775 & \textbf{0.794} & \textbf{+1.9} \\
        Pyrenees (I) & 0.930 & \textbf{0.957} & \textbf{+2.7} & 0.775 & \textbf{0.883} & \textbf{+10.8} \\
        Corsica (K) & 0.654 & \textbf{0.675} & \textbf{+2.1} & 0.554 & \textbf{0.642} & \textbf{+8.8} \\
        \bottomrule
    \end{tabular}
    \caption{Cross-validated overall accuracy (OA) and macro-F1 for the harmonic (HARM) and embedding (EMB) feature sets across GRECO eco-regions. Higher values are better. Embeddings deliver the largest gains in mixed Atlantic mosaics (Greater West: +6.8\,pp macro-F1) and steep mountain gradients (Pyrenees: +10.8\,pp), while Mediterranean convergence (+0.7\,pp) reflects low seasonal contrast. All folds use identical eco-region weights and tile assignments.}
    \label{tab:regional_performance}
\end{table}

This regional pattern makes sense: spatial context helps most where deciduous and evergreen mix at short range.

Beyond raw accuracy improvements, embeddings demonstrate superior calibration quality. The expected calibration error drops from 0.0586 (HARM) to 0.0335, and the maximum calibration error halves from 0.184 to 0.114. This calibration stability translates to operational robustness: macro-F1 remains steady when the decision threshold shifts within \([0.45,0.55]\), varying by only 0.003 for embeddings versus 0.009 for harmonics. These characteristics simplify deployment, particularly when analysts need to adjust thresholds for high-variance deciduous regions.

\subsection{Regional Performance Patterns}

Performance varies meaningfully across eco-regions (recall Table~\ref{tab:greco_summary} for ecological context), reflecting forest composition and phenological complexity. Table~\ref{tab:regional_performance} shows that EMB delivers the largest macro-F1 gains in Atlantic mosaics (Greater Crystalline and Oceanic West: +6.8\,pp; Oceanic Southwest: +6.5\,pp) and mountainous regions (Pyrenees: +10.8\,pp; Corsica: +8.8\,pp), where mixed stands and steep gradients challenge purely temporal descriptors. These uplifts mirror the GRECO descriptions (Table~\ref{tab:greco_summary}): the Armorican bocage (Region A) alternates humid oak–chestnut forests with post-war conifer plantations, while the Pyrenean chain (Region I) juxtaposes deciduous foothills with evergreen montane belts within a few hundred metres of elevation. Central Massif and the semi-oceanic North (Regions C and B) each gain about 3–4\,pp, consistent with their heterogeneous oak–hornbeam mosaics and scattered Scots pine estates. Mediterranean forests remain the hardest case: both models converge around \(0.80\) macro-F1 with a slim +0.7\,pp uplift, reflecting the drought-adapted evergreen broadleaf stands of Region J where seasonal amplitude is inherently low.

\begin{figure}[H]
    \centering
    \includegraphics[width=\textwidth]{images/France_Map_emb.png}
    \caption{France 2023 deciduous--evergreen map from the embedding model (EMB). Orange denotes deciduous, cyan denotes evergreen; the inset shows Corsica. Spatial patterns match major ecological gradients: evergreen dominance along Mediterranean and Atlantic pine regions (Landes) and at higher elevations (Alps, Jura, Vosges), with deciduous prevalence across lowland temperate belts. Compared to hand-designed features, embeddings yield smoother, more coherent patches while preserving sharp transitions.}
    \label{fig:national_map}
\end{figure}

\subsection{Spatial coherence and environmental drivers}
Second, we examine whether embeddings produce smoother maps and whether gains correlate with environmental gradients. Using the densities defined in Section~\ref{sec:metrics}, national edge density drops from 6.69 to 2.76\,km\,km\(^{-2}\) when switching from HARM to EMB, a 59\,\% reduction. The median total patch density \(D_{\text{tot}}\) falls from 6.81\,k to 2.16\,k components per 100\,km² (−68\,%), indicating far fewer tiny clusters. The largest smoothing occurs in maritime pine mosaics (Greater Crystalline and Oceanic West), where the harmonic map fragments mixed stands into speckled artefacts. EMB also maintains crisp elevational boundaries in the Alps, Jura, and Vosges, while matching National Forest Inventory proportions (64.4\,\% deciduous, 35.6\,\% evergreen; Figure~\ref{fig:national_map}). Table~\ref{tab:coherence_summary} summarises the national reduction in edge and patch densities, with supplementary Table~S3 providing the full statistics including the median-filter baseline.

Figure~\ref{fig:h2_multiscale} visualises these differences across five representative tiles. Each row pairs a Sentinel-2 chip (left) with the embedding and harmonic predictions (middle and right). The rows cover GRECO letters G (Greater Semi-Continental East), C (Corsica), A (Alps), O (Oceanic Southwest), and the Central Massif. Embeddings preserve large deciduous blocks (amber, \#e3712c) and evergreen belts (azure, \#2693c1) with minimal speckle; the harmonic counterpart produces many 1–2 pixel patches, consistent with its higher \(D_{\text{tot}}\). Additional summaries, including the effect of median filtering, appear in Supplementary Table~S3 and `results/analysis_coherence/coherence_summary.csv`.

\begin{figure}[H]
    \centering
    \includegraphics[width=\textwidth]{images/figure4_h2_panel.png}
    \caption{Five representative tiles (rows) showing the 2023 Sentinel-2 composite (left), EMB prediction (middle), and HARM prediction (right). Orange = deciduous (\#e3712c) and cyan = evergreen (\#2693c1). Key observation: embeddings preserve large coherent stands in Atlantic mosaics (rows G and O) and maintain sharp alpine and Corsican belts (rows A and C), while the harmonic baseline introduces speckled artefacts and broken edges. This smoothness arises because each embedding pixel encodes spatial context from neighbouring pixels, whereas harmonics model every pixel independently.}
    \label{fig:h2_multiscale}
\end{figure}

\begin{table}[H]
    \centering
    \small
    \begin{tabular}{lccc}
        \toprule
        \textbf{Metric} & \textbf{HARM} & \textbf{EMB} & \textbf{Relative change} \\
        \midrule
        Median edge density (km\,km\(^{-2}\)) & 6.69 & \textbf{2.76} & −59\,\% \\
        Median patch density (per 100\,km\(^2\)) & 6\,810 & \textbf{2\,159} & −68\,\% \\
        \bottomrule
    \end{tabular}
    \caption{National spatial coherence metrics aggregated over the 639 evaluation tiles. Lower values are better. Embeddings cut edge clutter by 59\,\% and patch fragmentation by 68\,\%, yielding maps that better match stand structure.}
    \label{tab:coherence_summary}
\end{table}

For agencies this means the raw embedding map already meets stand-level coherence targets without additional filtering.

\subsubsection{Tile-level heterogeneity and ancillary drivers}

Tile-level analysis confirms that most areas behave consistently across feature sets. The GRECO-balanced grid contains 639 tiles; 612 satisfy the label-density requirement and enter the tile-level evaluation. Within that subset, 83\,\% (510 tiles) fall within \(|z| < 1\sigma\) for accuracy differences, 14\,\% (89 tiles) exhibit a marked embedding advantage with mean Δaccuracy +12.6\,pp, and only 2\,\% (13 tiles) favour harmonics (mean −13.0\,pp). Embedding wins concentrate in Atlantic and western eco-regions where clay-rich soils, high deciduous ratios, and cooler/wetter ERA5 conditions prevail; correlations reach \(r=0.17\) with clay fraction, \(r=0.13\) with deciduous share, and \(r=-0.10\) with the ERA5 temperature range. Greater Crystalline and Oceanic West displays the highest share of advantage tiles (41\,\%) followed by Vosges (31\,\%) and Corsica (33\,\%), while the Mediterranean basin records the largest proportion of harmonic-favoured tiles (12\,\%). Table~\ref{tab:tile_buckets} summarises the distribution, and eco-region summaries (Supplementary Figure~S7) show that Greater Crystalline and Oceanic West gains \(+9.3\)\,pp average accuracy whereas the Mediterranean averages \(+0.9\)\,pp, reinforcing the idea that spatial context helps most in humid, mixed-stand mosaics and contributes modestly in evergreen-dominated shrublands.

The ancillary dataset enables a focused follow-up on the 89 embedding-advantaged tiles: we quantify which covariates (soil texture, moisture regime, diversity indices) best explain the observed lifts.

\begin{figure}[H]
    \centering
    \includegraphics[width=\textwidth]{images/embedding_harmonic_driver_dual.png}
    \caption{Relative covariate shifts for tiles with strong embedding gains (top row, \(z \ge +1\sigma\)) and harmonic gains (bottom row, \(z \le -1\sigma\)) compared with parity tiles. Left panels show percent deviations relative to parity means; right panels report Spearman correlations with Δaccuracy. Positive bars indicate that the attribute is higher in advantage tiles. Embedding gains cluster in wetter, more diverse mosaics (rainfall +2.1\%, Shannon diversity +24\%, narrower temperature range −1.4\%), whereas harmonic wins emerge in dry, evergreen-dominated tiles (rainfall −10\%, deciduous share −49\%, wider temperature range +2.2\%). This environmental contrast explains the regional performance patterns in Table~\ref{tab:regional_performance}.}
    \label{fig:driver_deltas}
\end{figure}

Figure~\ref{fig:driver_deltas} confirms that the embedding subset sits in wetter-than-average tiles (ERA5 rainfall +2.1\,\%), with higher canopy diversity (+24\,\% Shannon index) and only a weak reduction in deciduous share (−2\,\%). It also shows a slightly narrower annual temperature range (−1.4\,\%; \(\rho = -0.16\)), consistent with smoother seasonal dynamics. Harmonics achieve their rare wins in markedly drier pixels (−10\,\% rainfall) where evergreen dominance is much stronger (−49\,\% deciduous share, \(\rho = +0.22\)) and seasonal temperature swings are amplified (+2.2\,\%; \(\rho = +0.47\)). These shifts emphasise that embedding advantages stem from wetter, heterogeneous mosaics, whereas harmonics perform best in dry evergreen stands with pronounced seasonal forcing.

\begin{table}[H]
    \centering
    \small
    \begin{tabular}{lccc}
        \toprule
        \textbf{Tile bucket} & \textbf{Count} & \textbf{Share (\%)} & \textbf{Mean Δaccuracy (pp)} \\
        \midrule
        Embedding advantage (\(z \ge +1\sigma\)) & 89 & 14.5 & +12.6 \\
        Harmonic win (\(z \le -1\sigma\)) & 13 & 2.1 & −13.0 \\
        Rough parity (\(|z| < 1\sigma\)) & 510 & 83.3 & +1.4 \\
        \bottomrule
    \end{tabular}
    \caption{Distribution of tile-level performance buckets derived from cross-validated predictions (612 tiles with sufficient training labels). Most tiles (83.3\%) sit near parity. Embeddings dominate the advantage bucket (14.5\%), gaining +12.6\,pp in Atlantic and montane zones, while harmonic wins (2.1\%) concentrate in Mediterranean shrublands. Positive values favour embeddings.}
    \label{tab:tile_buckets}
\end{table}

\subsection{Temporal robustness and legacy alignment}
Third, we test whether a frozen 2023 model transfers to past years and stays compatible with existing national products.

Applying the 2023 EMB model to earlier AlphaEarth embeddings yields accuracies of 0.940 (2018), 0.938 (2020), and 0.935 (2022). These values sit within 1.5 standard deviations of the 2023 cross-validation mean (0.926) and trail the 2023 inference baseline by 3.8–5.4\,pp in macro-F1. The largest year-on-year degradation occurs in Corsica (Region K, −0.16\,pp macro-F1 in 2018) and along the Mediterranean coast (≈−0.07\,pp), consistent with wildfire disturbances and drought stress recorded in the AlphaEarth provenance. Continental regions—Vosges, Jura, Central Massif—remain within −0.05\,pp. These results indicate that the frozen Random Forest generalises well across annual embeddings, with performance largely governed by the quality of the embedding vintage (Supplementary Tables~S4–S5). Table~\ref{tab:temporal_stability} summarises the year-by-year metrics. Across the ten-tile cohort used for Figures~\ref{fig:h2_multiscale} and \ref{fig:h3_temporal}, the macro intersection-over-union (IoU) between the 2023 map and earlier vintages ranges from 0.786 (2018) to 0.801 (2022), with deciduous IoU consistently above 0.83—confirming that inter-annual drift is modest even in spatially heterogeneous sites.

Figure~\ref{fig:h3_temporal} illustrates the temporal behaviour for three representative tiles (G, O, C). Each column shows the Sentinel-2 composite for a given year (top row) alongside the corresponding EMB prediction (bottom row). Colour conventions match Figure~\ref{fig:h2_multiscale}. Despite varying illumination and disturbance cues, the embedding model preserves deciduous and evergreen patches across years, with the largest deviations occurring in the Corsican tile (row C) where post-fire succession alters the canopy mix between 2018 and 2020.

\begin{figure}[H]
    \centering
    \includegraphics[width=\textwidth]{images/figure5_h3_panel.png}
    \caption{Three GRECO tiles (rows G, O, C) observed across 2018, 2020, 2022, and 2023 (columns). Each column pairs the Sentinel-2 composite (top) with the EMB prediction (bottom) using the manuscript palette (deciduous \#e3712c, evergreen \#2693c1). Key finding: deciduous patches and evergreen belts remain stable through the 2020--2022 drought years, with the largest shifts occurring in Corsica (row C) where post-fire succession alters the canopy between 2018 and 2020. This robustness enables multi-year reanalysis with a single model.}
    \label{fig:h3_temporal}
\end{figure}

\begin{table}[H]
    \centering
    \small
    \begin{tabular}{lccccc}
        \toprule
        \textbf{Year} & \textbf{Source} & \textbf{Accuracy} & \textbf{Macro-F1} & \textbf{ΔF1 vs 2023 (pp)} & \textbf{IoU (macro)} \\
        \midrule
        2018 & Inference & 0.940 & 0.912 & −3.8 & 0.786 \\
        2020 & Inference & 0.938 & 0.917 & −4.4 & 0.796 \\
        2022 & Inference & 0.935 & 0.905 & −4.3 & 0.801 \\
        2023 & Cross-validation & 0.927 & 0.897 & −5.5 & -- \\
        2023 & Inference baseline & \textbf{0.969} & \textbf{0.951} & \textbf{0.0} & \textbf{1.000} \\
        \bottomrule
    \end{tabular}
    \caption{Temporal evaluation of the 2023 Random Forest applied to AlphaEarth embeddings from different years. Higher values are better for accuracy and macro-F1. The frozen model maintains macro-F1 within 5 percentage points of the 2023 inference baseline, and macro-IoU remains between 0.786 and 0.801. Degradation concentrates in fire- and drought-affected regions (Corsica, Mediterranean), consistent with genuine canopy change. Metrics are computed with the identical inference pipeline; deltas are expressed in percentage points relative to the 2023 inference baseline. Macro-IoU compares each vintage against the 2023 inference map over the ten tiles used in Figures~\ref{fig:h2_multiscale} and \ref{fig:h3_temporal}.
    \label{tab:temporal_stability}
\end{table}

Performance dips line up with the 2018 Corsican fires and Mediterranean drought years, indicating that remaining errors stem from genuine canopy change rather than model drift.

\subsubsection{Legacy products}

When compared against Copernicus DLT (broadleaf vs conifer) and BD Forêt V2 (deciduous vs evergreen polygons), the embedding map matches the harmonic baseline nationally: overall accuracy is 0.628 vs DLT and 0.638 vs BD Forêt, with macro-F1 differences below 0.4 percentage points (Table~\ref{tab:product_comparison_national}). Regional deltas follow the expected pattern—larger disagreements along recently disturbed Atlantic stands and Mediterranean shrublands where reference products either lag in time (BD Forêt) or do not separate evergreen broadleaf (DLT). Embeddings therefore improve internal consistency without breaking compatibility with incumbent datasets (Supplementary Section~S9).

\begin{table}[H]
    \centering
    \small
    \caption{National agreement with existing products (forest pixels only). Higher values are better. Embeddings maintain compatibility with Copernicus DLT and BD Forêt (Δmacro-F1 < 0.4 pp) while improving internal coherence relative to HARM. Metrics derive from aggregated confusion counts over the eco-region grid.}
    \begin{tabular}{lccc}
        \toprule
        \textbf{Comparison} & \textbf{Overall accuracy} & \textbf{Cohen's $\kappa$} & \textbf{Macro-F1} \\
        \midrule
        HARM vs DLT & 0.627 & 0.17 & 0.578 \\
        EMB vs DLT & 0.628 & 0.16 & 0.574 \\
        HARM vs BD Forêt & 0.639 & 0.17 & 0.585 \\
        EMB vs BD Forêt & 0.638 & 0.16 & 0.582 \\
        \bottomrule
    \end{tabular}
    \label{tab:product_comparison_national}
\end{table}

This confirms that agencies can adopt embeddings without breaking alignment with existing reporting products, while benefiting from smoother maps.

Qualitatively, embeddings reduce speckle in mixed mosaics such as the Landes maritime pine plantations and Corsican Castagniccia chestnut groves, while DLT and BD Forêt retain their strengths in planted conifer estates and mature deciduous stands (Figure~\ref{fig:comparison_products}). These products remain complementary: our annual map reflects the current canopy state, whereas DLT and BD Forêt provide longer-term typologies.

\begin{figure}[H]
    \centering
    \includegraphics[width=\textwidth]{images/Comparison_dlt_bdforet_harmonic_embedding.png}
    \caption{Three sites (rows) across five sources (columns): Sentinel-2 reference, HARM, EMB, BD Forêt V2, and Copernicus DLT. Orange = deciduous (broadleaf for DLT); cyan = evergreen (conifer for DLT). Observation: embeddings reduce the isolated pixels apparent in HARM (Landes row) while matching BD Forêt structure in managed stands (Fontainebleau). DLT merges evergreen broadleaf with conifers (Corsica row), highlighting limitations of legacy taxonomies.}
    \label{fig:comparison_products}
\end{figure}

Figure~\ref{fig:comparison_products} highlights a consistent behavior: embeddings encode neighborhood context, yielding smoother, more coherent class patches without isolated pixels, while harmonic features provide crisp pixelwise decisions that can appear speckled at fine scales. Simple post‑processing (e.g., median filtering) narrows this gap for harmonics, but embedding‑based inference attains similar regularization intrinsically.

\subsection{What do embeddings retain from harmonics?}

To interpret what the embeddings retain from traditional descriptors we projected each embedding dimension onto the harmonic feature space using ridge regression. Across all GRECO regions the mean out-of-sample \(R^2\) remains negative (from −0.31 in the Alps to −0.90 in the Pyrenees), confirming that no linear combination of harmonic descriptors can reconstruct the embedding activations. In other words, the pre-trained representation captures information that is orthogonal to the explicit sinusoidal components.

Although global reconstruction fails, the normalized ridge coefficients reveal which harmonic families the embeddings lean on. Averaged over regions, almost half of the coefficient mass targets spectral offsets—especially the CRSWIR ratio—while first-harmonic amplitudes account for a further 23\,\%, residual variances 16\,\%, and phase terms 11\,\%. The nearest-neighbour analysis shows that 57 out of 154 embedding dimensions align most strongly with CRSWIR offsets, 33 with NBR first-harmonic amplitudes, and 27 with NDVI amplitudes. This echoes the eco-regional behaviour: mountainous regions (Vosges, Jura, Central Massif) emphasise structural moisture cues (CRSWIR offsets), Atlantic mosaics emphasise moisture-driven amplitude contrasts (NBR amplitude), and Mediterranean/Corsican regions highlight NDVI amplitude differences between evergreen maquis and deciduous stands.

\begin{table}[H]
    \centering
    \small
    \begin{tabular}{lcc}
        \toprule
        \textbf{Region (GRECO)} & \textbf{Mean $R^2$} & \textbf{Dominant harmonic cue} \\
        \midrule
        Alps (H) & −0.31 & NDVI amplitude (seasonal vigor) \\
        Central Massif (C) & −0.32 & CRSWIR offset (baseline moisture) \\
        Corsica (K) & −0.72 & CRSWIR offset (baseline moisture) \\
        Grand Ouest (A) & −0.35 & NBR amplitude (moisture contrast) \\
        Grand Est (C$^\prime$) & −0.56 & NBR offset (woody baseline) \\
        Jura (E) & −0.72 & CRSWIR offset (baseline moisture) \\
        Mediterranean (J) & −0.49 & NBR amplitude (moisture contrast) \\
        Sud-Ouest (F) & −0.87 & NDVI amplitude (seasonal vigor) \\
        Pyrenees (I) & −0.90 & CRSWIR offset (baseline moisture) \\
        Semi-Oceanic North (B) & −0.33 & NBR amplitude (moisture contrast) \\
        Vosges (D) & −0.06 & CRSWIR offset (baseline moisture) \\
        \bottomrule
    \end{tabular}
    \caption{Mean out-of-sample $R^2$ across tiles and embedding dimensions remains negative for every eco-region, confirming that embeddings are not linear recombinations of harmonics. Dominant cues from the ridge projections highlight which phenological axes the embeddings still reference (e.g., CRSWIR offsets, NBR amplitudes).}
    \label{tab:similarity_summary}
\end{table}

Table~\ref{tab:similarity_summary} reports the eco-region–weighted mean of \(R^2_{t,k}\) across tiles and all 14 embedding dimensions, together with the harmonic descriptor that carries the largest normalised ridge weight (the same cue usually matches the top Pearson correlation). Most regions exhibit negative mean $R^2$, confirming that no linear combination of harmonics can reconstruct the embeddings wholesale, even though specific dimensions still correlate with interpretable cues such as CRSWIR offsets or NBR amplitudes. This absence of linear reconstruction, combined with the alignment patterns, supports a hybrid interpretation: AlphaEarth embeddings retain phenological axes while injecting spatial context learned during pre-training. That extra context likely underpins the smoother boundaries and higher accuracy gains observed in mixed Atlantic and montane landscapes, where purely temporal descriptors struggle. Per-dimension coefficients and correlation tables appear in Supplementary Section~S3 for readers who need the fine-grained breakdown.

\section{Discussion and Outlook}

AlphaEarth embeddings provide a pragmatic baseline for national phenology mapping. Under matched folds they raise accuracy by \(2.8 \pm 0.4\) percentage points and macro-F1 by \(3.7 \pm 0.5\) across logistic regression, linear SVM, and Random Forest classifiers (Table~\ref{tab:multi_model_cv}). The Random Forest halves the expected calibration error (0.033 vs 0.059) and cuts edge and patch densities by 59\,\% and 68\,\% (Table~\ref{tab:coherence_summary}). Crucially, these gains arrive without GPU clusters or bespoke harmonic tuning, so the historical compute-or-craft trade-off disappears: the only scarce resource is labelled ground truth.

Embedding advantages are spatially selective. Table~\ref{tab:tile_buckets} shows that 14.5\,\% of tiles deliver marked embedding gains (+12.6\,pp) while only 2.1\,\% favour harmonics. These tiles cluster in wetter, clay-rich mosaics with high stand diversity (Figure~\ref{fig:driver_deltas})—Atlantic bocage, Pyrenean gradients, alpine foothills—where deciduous and evergreen mix over 10--100\,m. Harmonic wins persist in Mediterranean shrublands where evergreen dominance and strong seasonal forcing reward explicit drought indices. A hybrid workflow can therefore keep EMB as the default while adding drought-sensitive harmonic cues in those shrublands.

Calibration is now an operational asset. Embeddings halve expected calibration error (0.033 vs 0.059) and keep macro-F1 stable when thresholds shift between 0.45 and 0.55 (Section~\ref{sec:metrics}). Agencies can raise the evergreen threshold to 0.55 to suppress false positives in mixed stands, confident that predicted probabilities remain honest because the maximum calibration error stays at 0.114. This reliability removes the guesswork that often accompanies pixel-based products.

Three limitations point to concrete extensions. First, label scarcity in Mediterranean regions still caps performance; expanding field campaigns in drought-prone eco-regions would close the remaining gap. Second, the current binary label space hides genus-level structure—Supplementary Section~S10 already shows 82.8\,\% genus accuracy, so denser supervision could unlock richer taxonomies. Third, ancillary covariates such as soil moisture and climate anomalies could be added to the lightweight classifiers to recover the 2\,\% of tiles where harmonics still lead.

Phenology-aware disturbance monitoring will benefit directly. Change detectors such as CCDC, BFAST, and LandTrendr rely on class-conditioned thresholds; accurate deciduous-versus-evergreen priors let analysts tolerate larger residuals for deciduous stands while tightening evergreen thresholds. Embedding-derived maps therefore lower false alerts without suppressing genuine anomalies, especially when paired with annual updates of the AlphaEarth embeddings.

Transferability and scalability follow naturally. The 2023 model generalises to 2018–2022 embeddings with macro-IoU 0.786–0.801 (Table~\ref{tab:temporal_stability}), and compatibility with BD Forêt and Copernicus DLT stays intact (Table~\ref{tab:product_comparison_national}). The bottleneck has shifted from modelling skill to label coverage: forest agencies now need to organise field campaigns and QA pipelines rather than build GPU infrastructure. That inversion plays to their strengths and sets the stage for extending this workflow to other biomes.

\section{Conclusion}

AlphaEarth embeddings close the gap between labor-intensive feature engineering and compute-heavy deep networks for national deciduous--evergreen mapping. Across three lightweight classifiers trained on 14 million pixels, embeddings raise overall accuracy to 91.9\,\%, macro-F1 to 89.7\,\%, halve calibration error, and cut map fragmentation by 68\,\%—all without manual feature design or GPU clusters. Gains concentrate in humid, mixed-stand mosaics where spatial context separates adjacent patches, while dry evergreen strongholds show smaller advantages.

Operational robustness follows. A frozen 2023 model transfers to 2018--2022 embeddings with macro-IoU 0.79--0.80 (Table~\ref{tab:temporal_stability}), and compatibility with Copernicus DLT and BD Forêt remains intact (Table~\ref{tab:product_comparison_national}). The main limitation is label coverage in drought-prone regions. Future work should add ancillary covariates to the lightweight classifiers, expand genus-level supervision, and replicate this benchmark across other biomes.

Because embeddings are pre-computed and freely distributed, the bottleneck shifts from compute and feature engineering to ground-truth collection. Forest agencies can now focus on field campaigns and QA pipelines—the components they already control.

\section*{Data Availability}
Training data summaries, model configurations, cross-validation splits, and similarity analysis outputs will be made available with the publication. The 2023 10\,m deciduous–evergreen map of France will also be released.

\section*{Code Availability}
Code to reproduce data preparation, feature extraction, model training, evaluation, and similarity analyses will be released upon publication.

\bibliographystyle{Frontiers-Harvard}
\bibliography{phenology}

\end{document}
