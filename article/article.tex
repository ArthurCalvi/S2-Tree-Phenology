%%%%%%%%%%%%%%%%%%%%%%%%%%%%%%%%%%%%%%%%%%%%%%%%%%%%%%%%%%%%%%%%%%%%%%%%%%%%%%%%
% Frontiers LaTeX template – v2 (2025-04-10)
%%%%%%%%%%%%%%%%%%%%%%%%%%%%%%%%%%%%%%%%%%%%%%%%%%%%%%%%%%%%%%%%%%%%%%%%%%%%%%%%
\documentclass[utf8]{FrontiersinHarvard}
\usepackage{url,hyperref,lineno,microtype,subcaption,amsmath}
\usepackage{natbib}
\usepackage[onehalfspacing]{setspace}
\usepackage{float}
\usepackage{multirow}
\linenumbers

\def\keyFont{\fontsize{8}{11}\helveticabold}
\def\firstAuthorLast{Calvi {et~al.}}
\def\Authors{Arthur Calvi\,$^{1,*}$, Sarah Brood\,$^{2}$, Co-Author\,$^{1,2}$, OpenAI Codex\,$^{3}$, Anthropic Claude Code\,$^{4}$}
\def\Address{$^{1}$ Laboratory X, Institute X, Department X, City X, Country X\\
$^{2}$ Laboratory Y, Institute Y, Department Y, City Y, Country Y\\
$^{3}$ OpenAI, San Francisco, CA, USA\\
$^{4}$ Anthropic, San Francisco, CA, USA}
\def\corrAuthor{Arthur Calvi}
\def\corrEmail{email@uni.edu}

\begin{document}
\onecolumn
\firstpage{1}

\title[Towards universal forest mapping]{Towards Universal Forest Mapping: Foundation Embeddings Outperform Hand-Crafted Harmonics for Deciduous–Evergreen Classification}
%je mentionnerai le capteur dans le titre Mapping accuracy for evergreen and deciduous forests using AlphaEarth embeddings vs. sentinel 2 time series. ou Comparing model performances with AlphaEarth embeddings vs. sentinel 2 time series for classifying evergreen and decduous forests pas besoin de mettre ‘in France’

\author[\firstAuthorLast]{\Authors}
\address{}
\correspondance{}

\maketitle

\begin{abstract}
High-resolution, annually updated deciduous–evergreen maps underpin forest carbon accounting, disturbance surveillance, and biodiversity reporting, yet most agencies still lack a practical way to produce them. Building such maps through mathematical simulation over remote sensing time series typically means choosing between hand-crafted harmonic descriptors that demand local phenology expertise and deep models that require GPU fleets and machine-learning specialists—the ``compute-or-craft'' bottleneck. Unfortunately, neither option removes the bottleneck for resource-limited agencies. We test whether pre-computed AlphaEarth embeddings, distributed through Google Earth Engine, can remove that constraint when evaluated fairly against physics-informed harmonics. Using 14.1 million labelled pixels grouped into 639 GRECO-balanced tiles, we train identical eco-region-weighted Random Forests on two 14-dimensional feature sets: a fold-selected subset of AlphaEarth embeddings (EMB-14) and Sentinel-2 harmonic descriptors (HARM-14). EMB-14 increases overall accuracy from $0.904 \pm 0.0059$ to $0.926 \pm 0.0059$, lifts macro-F1 from 0.874 to 0.905, halves the expected calibration error (0.033 vs 0.059), and reduces national edge and patch densities by 59\,\% and 68\,\%, while a frozen 2023 model transfers to 2018–2022 embeddings with macro IoU ranging from 0.786 to 0.801. Because the strongest performance now arrives without bespoke compute or bespoke features, label coverage becomes the primary lever for improving nationwide deciduous–evergreen maps.

\keyFont{\section{Keywords:} deciduous-evergreen, AlphaEarth embeddings, harmonic analysis, Random Forest, phenology}
\end{abstract}

\section{Introduction}

% Teaser figure pinned within Introduction
\begin{figure}[H]
    \centering
    \includegraphics[width=\textwidth]{images/sat_emb_classes.png}
    \caption{South‑east of France: (left) true‑color satellite view; (middle) learned embedding visualization (AlphaEarth dimensions A46, A18, A05 mapped to RGB)—notice how embeddings sharpen phenological structure without manual feature design; (right) classification probabilities where orange indicates deciduous and blue evergreen. This teaser previews the central finding: embeddings encode spatial context that harmonics miss.}
    \label{fig:teaser_sat_emb_class}
\end{figure}

High-resolution, annually refreshed deciduous–evergreen maps are essential for tracking carbon fluxes, prioritising disturbance response, and reporting biodiversity trends \citep{Zhu2014,Zhao2019,Verbesselt2010a,Verbesselt2010b,Kennedy2010,Kennedy2018}. Yet most agencies still rely on legacy inventories or continental products that update slowly and blur key forest types. The Copernicus Dominant Leaf Type (DLT) map, for instance, separates broadleaf from conifer but merges evergreen oaks with pines \citep{EU2024a}, while BD Forêt V2 offers detailed stand labels frozen to field campaigns from 2007–2018 \citep{IGN2024}. Delivering a current nationwide map therefore remains a bespoke effort rather than an operational service.

Those bespoke efforts confront the same dilemma: build hand-crafted features (labour-intensive) or train deep networks (compute-intensive). Physics-informed workflows fit harmonic curves to Sentinel-2 indices, extracting amplitudes, phases, and residual variances that capture seasonal behaviour \citep{Inglada2017,Li2023,Bolton2020}. The descriptors are interpretable and appealing to domain experts, but the tuning effort scales with every eco-region, sensor update, and disturbance regime. In parallel, end-to-end deep networks can learn spatial context automatically \citep{Low2020,Xie2024FoundationEffective}, yet they demand GPU clusters, large labelled datasets, and dedicated ML operations. This compute-or-craft bottleneck still limits operational deployment of high-resolution forest mapping.

Foundation embeddings offer a potential way out by shifting the constraint: pre-trained models published through Earth Engine eliminate both GPU clusters and fragile harmonic tuning, concentrating effort on the one input that cannot be outsourced—ground-truth labels. Satellite pretraining initiatives such as SatMAE, Prithvi-EO, and AlphaEarth learn multi-sensor, multi-temporal representations that transfer across tasks \citep{Cong2022,Szwarcman2024PrithviEO2,AlphaEarth2025}. AlphaEarth is particularly compelling for practitioners because it publishes annual 64-dimensional embeddings at 10\,m directly through Google Earth Engine—no GPU, no feature tuning required. These vectors have already supported cross-border vegetation typing and disturbance analyses \citep{Houriez2025AEFDataGen,Seydi2025AlphaEarthBurnedArea}, but their value relative to carefully engineered harmonics has not been benchmarked under matched conditions. Figure~\ref{fig:teaser_sat_emb_class} previews the contrast: embeddings sharpen phenological structure without bespoke tuning.

We address that gap with a France-wide benchmark that evaluates harmonics and embeddings on equal footing. Our dataset aggregates 14.1 million labelled forest pixels from four sources, groups them into 639 non-overlapping 2.5\,km tiles, and balances sampling across the eleven GRECO eco-regions. Two compact 14-dimensional feature sets feed identical eco-region-weighted Random Forests: the fold-selected AlphaEarth subset (\textbf{EMB-14}) and physics-informed Sentinel-2 harmonic descriptors (\textbf{HARM-14}). Tile-grouped cross-validation measures accuracy and calibration, ancillary drivers explain spatial heterogeneity, temporal experiments test whether a frozen 2023 embedding model transfers to 2018–2022 vintages, and national comparisons situate the outputs against Copernicus DLT and BD Forêt.

This design lets us interrogate three hypotheses. \textbf{H1} asks whether embeddings match or exceed harmonic accuracy and calibration when data, folds, and classifiers are held constant. \textbf{H2} examines whether embeddings deliver smoother maps whose gains co-vary with soils, climate, and forest composition. \textbf{H3} tests temporal robustness and compatibility with legacy national products. The remainder of the paper details the data and processing pipeline (Section~2), presents the accuracy, coherence, ancillary, and temporal analyses (Section~3), and concludes with operational guidance and remaining limitations (Section~4).

\section{Data and Methods}

\subsection{Study area and eco-regional stratification}
\label{sec:greco}
We focus on mainland France and Corsica, where temperate, montane, and Mediterranean forest types co-exist within short distances. To preserve ecological diversity in both training and evaluation we rely on the eleven \emph{Grandes Régions Écologiques} (GRECO) defined by the French National Forest Inventory \citep{IGN2013GRECO}. These eco-regions capture the dominant gradients from Atlantic mixed forests through semi-continental plateaus to Mediterranean evergreen stands and Alpine massifs. We partitioned the national forest mask into 639 non-overlapping 2.5\,km\,$\times$\,2.5\,km tiles (Figure~\ref{fig:training_tiles}), retaining tiles that contain at least ten inventory or plot-labelled pixels and are separated by 5\,km to limit spatial autocorrelation. After the reference quota is met, remaining forest pixels in each tile are populated with buffered BD Forêt V2 polygons so that both feature families share an identical footprint. Sample weights then align the dataset with the actual forest extent of each eco-region. We estimate that extent through an effective forest area \(A_r\), obtained by multiplying each GRECO polygon by the region's mean forest cover ratio provided by the National Forest Inventory, and we sum these areas across regions to get \(\sum_{j} A_j\). The training dataset contains \(N\) labelled forest pixels, with \(n_r\) of them belonging to region \(r\); the per-pixel weight is therefore defined as \(w_r = (A_r / \sum_{j} A_j) / (n_r / N)\). Regions that occupy a large share of forest but are sparse in the sample receive \(w_r > 1\), whereas regions that are already over-represented are down-weighted. For example, the Oceanic Southwest accounts for roughly 22\% of the effective forest area yet only about 21\% of the labelled pixels, so its observations are scaled by \(w_r \approx 1.06\). We normalise the weights so that \(\sum_i w_i = N\); the resulting weights range from 0.89 in densely sampled regions to 1.48 in rare mountainous tiles. A small fraction of coastal pixels remains outside the GRECO mask and is excluded from stratified summaries.

The GRECO framework offers concise descriptors that explain the spatial behaviour of our models. Region A (Greater Crystalline and Oceanic West) is a humid Atlantic bocage of oak–chestnut forests interspersed with post-war conifer plantations and dense hedgerows. Region B (Semi-Oceanic North Centre) covers the Paris Basin cuestas with extensive pedunculate/sessile oak and hornbeam on loess plateaus and Scots pine on sandy soils. Region C (Greater Semi-Continental East) spans the Ardennes and Lorraine, with beech–fir forests on cool uplands. Region D (Vosges) combines silver fir–beech on steep crystalline slopes, while Region E (Jura) hosts calcareous beech–fir mosaics above karst plateaus. Region F (Oceanic Southwest) includes the Landes maritime pine massif and humid mixed stands. Region G (Central Massif) contains montane beech–fir and chestnut belts, Region H (Alps) sharply transitions from deciduous foothills to coniferous subalpine belts, and Region I (Pyrenees) alternates Atlantic and Mediterranean influences over steep gradients. Region J (Mediterranean) is dominated by evergreen holm and cork oak maquis with summer drought stress, and Region K (Corsica) blends lowland maquis with high-elevation Laricio pine forests. These descriptors provide the ecological backdrop for the regional analyses reported in Section~\ref{sec:results} and are summarised in Table~\ref{tab:greco_summary}.

\begin{table}[H]
    \centering
    \small
    \begin{tabular}{p{0.8cm}p{3.6cm}p{3.9cm}p{4.6cm}p{1.7cm}}
        \toprule
        \textbf{Code} & \textbf{Region (English name)} & \textbf{Climate and terrain} & \textbf{Dominant forest structure} & \textbf{Training pixels} \\
        \midrule
        A & Greater Crystalline and Oceanic West & Humid Atlantic plains and low hills, dense bocage networks & Oak–chestnut coppice with maritime pine and Sitka spruce plantations & 0.49\,M \\
        B & Semi-Oceanic North Centre & Loess plateaus and chalk cuestas of the Paris Basin & Pedunculate/sessile oak with hornbeam or beech; Scots/Corsican pine on sandy soils & 2.39\,M \\
        C & Greater Semi-Continental East & Ardennes and Lorraine uplands with colder winters & Beech–fir and spruce on mesic plateaus, mixed oak lowlands & 2.80\,M \\
        D & Vosges & Steep crystalline range with high precipitation & Silver fir–beech high forests with spruce and Douglas-fir plantations & 0.39\,M \\
        E & Jura & Limestone plateaus under cool montane climate & Calcareous beech–fir mosaics and mixed spruce on karst slopes & 0.20\,M \\
        F & Oceanic Southwest & Landes coastal plain and Atlantic piedmont & Maritime pine estates interleaved with humid oak, alder, and chestnut stands & 2.91\,M \\
        G & Central Massif & Volcanic plateaus and valleys with montane climate & Beech–fir belts, chestnut groves, Scots/Douglas pine on poorer soils & 1.92\,M \\
        H & Alps & Sharp elevational gradients and deep glacial valleys & Deciduous foothills grading to spruce–larch–fir subalpine belts & 0.65\,M \\
        I & Pyrenees & Atlantic–Mediterranean transition, steep valleys & Oak/beech foothills with fir–spruce upper slopes and Mediterranean pine on south-facing flanks & 0.56\,M \\
        J & Mediterranean & Coastal ranges and plateaus with intense summer drought & Evergreen holm and cork oak maquis, Aleppo and maritime pine stands & 1.43\,M \\
        K & Corsica & Crystalline massif with rugged relief & Lowland evergreen maquis and extensive Laricio pine forests above 900\,m & 0.35\,M \\
        \bottomrule
    \end{tabular}
    \caption{Summary of the eleven GRECO eco-regions used for sampling and evaluation. Training pixel counts correspond to the supervised dataset and sum to 14.1 million samples (rounded to two decimals in millions).}
    \label{tab:greco_summary}
\end{table}

\begin{figure}[H]
    \centering
    \includegraphics[width=0.8\textwidth]{images/tiles_2_5_km_final_visualization.png}
    \caption{Distribution of the 639 eco-region balanced training tiles. Warm colours indicate Mediterranean and montane eco-regions, while cool colours denote Atlantic and semi-continental domains. Each tile contains 10\,m pixels labelled as deciduous or evergreen and is used as an indivisible unit during cross-validation.}
    \label{fig:training_tiles}
\end{figure}

\subsection{Reference labels}
The supervised dataset aggregates 14{,}086{,}937 forest pixels (10\,m resolution) drawn from four complementary sources. (i) PureForest provides 135\,000 lidar-guided patches with expert-verified dominant species, mainly covering monospecific stands in southern France \citep{gaydon2024pureforestlargescaleaeriallidar}. (ii) The RENECOFOR long-term monitoring network supplies plot inventories with tree measurements above 15\,m height collected between 2019 and 2020 \citep{ulrich:hal-03444393}. (iii) The Tree Position Calibration campaign geolocates dominant trees with airborne lidar to refine field GPS positions and species attribution \citep{ONF,IGN_LiDARHD}. (iv) BD Forêt V2 contributes mapped stands ( > 5{,}000\,m$^2$) updated between 2005 and 2019 \citep{IGN2024}. We apply a 100\,m negative buffer to BD Forêt polygons to mitigate edge drift before rasterisation. Each pixel inherits its eco-region and tile identifier; categorical attributes (phenology, genus, species, source, acquisition year) are encoded via consistent look-up tables. Deciduous pixels represent 75.5\,\% of the samples (10{,}639{,}124 pixels), evergreens 24.5\,\% (3{,}447{,}813 pixels). Buffered BD Forêt polygons supply 88.6\,\% of the pixels, while in-situ inventories anchor 11.4\,\%, ensuring we retain authoritative field labels within every eco-region.

\subsection{Feature extraction}
\subsubsection{Sentinel-2 harmonic descriptors}
We reconstruct Sentinel-2 Level-2A surface reflectances from Google Earth Engine monthly MEDIAN composites, applying the s2cloudless cloud-probability mask with a 75\,\% threshold, removing QA60 bits 10 and 11 (opaque and cirrus clouds), and discarding scenes with more than 95\,\% cloudy pixels. No gap filling or scene classification layer is applied. For each tile we derive the vegetation indices NDVI, EVI, NBR, and the SWIR ratio (CRSWIR). The annual signal of each index is fitted with two sinusoidal harmonics through ordinary least squares,
\[
  x(t) = C + \sum_{k=1}^{2} \big[a_k \cos\!\big(2\pi k t/T\big) + b_k \sin\!\big(2\pi k t/T\big)\big], \quad T = 1\,\text{year}.
\]
We recover offsets, amplitudes, phases, and residual variance; phases are transformed into their sine and cosine components to avoid wrap-around discontinuities. Figure~\ref{fig:harmonics_decomposition} illustrates how the two-harmonic model decomposes an NDVI trajectory into interpretable amplitude and phase components while keeping residuals explicit. Recursive feature elimination (Section~\ref{subsubsec:rfe}) retained 14 descriptors that balance interpretability and cross-validated accuracy (Supplementary Section~S1). These descriptors (Table~\ref{tab:harmonic14}) span amplitude, phase, offset, and residual components across four indices: NDVI first-harmonic amplitude captures seasonal greenness pulses; NBR amplitude reflects moisture/structural contrast; CRSWIR offset encodes baseline water content. We denote this subset as \textbf{HARM-14}.

\begin{table}[H]
    \centering
    \small
    \begin{tabular}{lp{0.62\textwidth}}
        \hline
        \textbf{HARM-14 descriptor} & \textbf{Ecological signal} \\
        \hline
        NDVI first-harmonic amplitude & Seasonal strength of broadleaf greenness pulses \\
        NDVI first-harmonic phase (cosine/sine) & Timing of green-up and senescence transitions \\
        NDVI second-harmonic phase (sine) & Asymmetry between rapid spring and gradual autumn trajectories \\
        NDVI offset & Mean canopy greenness throughout the year \\
        NBR first-harmonic amplitude & Annual moisture and structural contrast between canopies and soil \\
        NBR first-harmonic phase (cosine) & Calendar timing of minimum fuel moisture \\
        NBR second-harmonic phase (cosine) & Secondary moisture cycle in bimodal climates \\
        NBR offset & Average woody biomass signal \\
        NBR residual variance & Short-term disturbance departures from harmonic behaviour \\
        CRSWIR first-harmonic phase (cosine) & Timing of peak shortwave-infrared water stress \\
        CRSWIR second-harmonic phase (cosine) & Recovery pattern of evergreen water content \\
        CRSWIR offset & Mean canopy water and lignin content \\
        CRSWIR residual variance & Fine-scale heterogeneity in SWIR response \\
        \hline
    \end{tabular}
    \caption{Fourteen harmonic descriptors retained after eco-region-balanced recursive feature elimination. Together they define the \textbf{HARM-14} feature set used in the Random Forest baseline. Phases are represented via sine and cosine components to avoid angular discontinuities.}
    \label{tab:harmonic14}
\end{table}

\begin{figure}[H]
    \centering
    \includegraphics[width=\textwidth]{images/harmonics_decomposition.png}
    \caption{Harmonic decomposition of an NDVI annual curve. The first and second harmonics capture the asymmetric spring green-up and autumn senescence, while the residual component highlights departures linked to disturbance or mixed pixels. These interpretable pieces constitute the descriptors retained in \textbf{HARM-14}.}
    \label{fig:harmonics_decomposition}
\end{figure}

\subsubsection{AlphaEarth embeddings}
AlphaEarth provides annual 64-dimensional embeddings at 10\,m resolution, learned from multi-sensor stacks (Sentinel-1/2 optical reflectances, Sentinel-1 backscatter, Landsat seasonal composites, GEDI structure, ERA5-Land climate predictors) by masking and reconstructing spatio-temporal tokens \citep{AlphaEarth2025}. The resulting vector can be viewed as a "spectro-temporal fingerprint" of each pixel: it summarises how the site looks across the year, how it relates to its neighbourhood, and how ancillary drivers (elevation, climate) modulate that pattern. Because the embeddings are already computed and distributed through Earth Engine, practitioners can treat them as off-the-shelf features without training deep networks. We retrieved the 2018, 2020, 2022, and 2023 embedding rasters, converted the tiles to Parquet, and removed pixels with non-finite components (mostly coastline artifacts). To align with the harmonic dataset we perform an inner join on tile, row, and column indices so that both feature families use the exact same labelled pixels and eco-region weights. The 14 embedding dimensions retained after feature elimination (Section~\ref{subsubsec:rfe}) define the \textbf{EMB-14} feature set used in training and evaluation (Table~\ref{tab:emb14}).

\begin{table}[H]
    \centering
    \small
    \begin{tabular}{ll}
        \hline
        \textbf{Rank} & \textbf{EMB-14 dimension} \\
        \hline
        1 & embedding\_46 \\
        2 & embedding\_18 \\
        3 & embedding\_5 \\
        4 & embedding\_30 \\
        5 & embedding\_39 \\
        6 & embedding\_0 \\
        7 & embedding\_57 \\
        8 & embedding\_23 \\
        9 & embedding\_6 \\
        10 & embedding\_15 \\
        11 & embedding\_13 \\
        12 & embedding\_22 \\
        13 & embedding\_11 \\
        14 & embedding\_24 \\
        \hline
    \end{tabular}
    \caption{Fourteen AlphaEarth embedding dimensions retained after eco-region-balanced recursive feature elimination, ranked by mean importance across folds. Together they define the \textbf{EMB-14} feature set.}
    \label{tab:emb14}
\end{table}

\subsubsection{Recursive feature elimination}
\label{subsubsec:rfe}
Both feature families are pruned with the same eco-region-balanced recursive elimination. Starting from the full candidate list (all harmonic descriptors after circular transforms or the 64 embedding channels), we train weighted random forests on each of the five spatial folds, rank features by their mean importance, and remove the weakest features at every stage. The elimination schedule removes features successively, ensuring that at least 10 predictors remain. After every pruning step we record macro-F1 and accuracy along with eco-region metrics; the curve plateaus at 14 features for the harmonics. We therefore retain the best-performing 14-feature configuration and refer to it as HARM-14 for harmonics and EMB-14 for embeddings. Alternative top-$K$ lists extracted from the same procedure support the ablation studies summarised in Supplementary Section~S2.

\subsection{Random Forest training and cross-validation}
We train separate random forest classifiers for HARM-14 and EMB-14 using 50 trees, maximum depth 30, minimum samples split 30, and minimum leaf size 15. These hyperparameters come from scikit-learn’s successive halving grid search: candidate configurations are evaluated on progressively larger sample fractions while eliminating the weakest options at each stage, yielding a coarse-to-fine search that is more sample-efficient than an exhaustive grid. We tune the search on the harmonic features and reuse the winning configuration for the embedding model so that any differences stem from the representation rather than model capacity—a choice that slightly favours the harmonic baseline and therefore provides a conservative comparison for EMB-14. Class imbalance is mitigated through scikit-learn’s balanced class weights combined with the eco-region weights described above. Five-fold cross-validation preserves tile integrity: tiles within each eco-region are grouped and assigned wholesale to folds, ensuring that no validation tile shares pixels with the training subset. The folds are identical for both feature sets; we record out-of-fold predictions, per-fold metrics, and the aggregated confusion matrix. After cross-validation we retrain each model on the full weighted dataset for downstream inference.

\subsection{Evaluation metrics and derived summaries}
\label{sec:metrics}
Fold-level metrics include overall accuracy, class-specific precision and recall, macro- and weighted-F1, and the four entries of the confusion matrix. Macro-F1 averages the class-specific F1-scores with equal weight so that deciduous and evergreen contribute symmetrically; weighted-F1 applies the empirical class proportions \(n_c/N\) observed in the fold, i.e.
\begin{equation*}
    \mathrm{F1}_{\text{macro}} = \frac{1}{2} \sum_{c \in \{\text{dec}, \text{ever}\}} \mathrm{F1}_c,
    \qquad
    \mathrm{F1}_{\text{weighted}} = \sum_{c \in \{\text{dec}, \text{ever}\}} \frac{n_c}{N} \mathrm{F1}_c,
\end{equation*}
where \(n_c\) is the number of validation pixels of class \(c\) and \(N = n_{\text{dec}} + n_{\text{ever}}\). We report the mean, standard deviation, and interquartile range across folds, while eco-region summaries average validation folds using the eco-region weights. Calibration is assessed with 10 equal-width probability bins on the evergreen posterior, i.e. the random forest probability that a pixel belongs to the evergreen class. For each bin \(b\) with sample set \(S_b\) we compute the mean confidence \(\hat{p}_b = |S_b|^{-1} \sum_{i \in S_b} p_i\) and the observed evergreen fraction \(\hat{y}_b = |S_b|^{-1} \sum_{i \in S_b} \mathbf{1}\{y_i = \text{evergreen}\}\). The expected calibration error is the weighted average of the absolute gaps between these two quantities,
\begin{equation*}
    \mathrm{ECE} = \sum_{b=1}^{10} \frac{|S_b|}{N} \left| \hat{p}_b - \hat{y}_b \right|,
\end{equation*}
with \(N\) the total number of validation pixels, while the maximum calibration error (MCE) records \(\max_b \left| \hat{p}_b - \hat{y}_b \right|\). When a model is perfectly calibrated, \(\hat{p}_b\) and \(\hat{y}_b\) coincide in every bin and both metrics vanish. This reliability analysis clarifies how users can adjust the decision threshold: we also track macro-F1 when the evergreen threshold moves to 0.45 or 0.55 to confirm that modest shifts—such as raising the threshold to suppress evergreen false positives—do not destabilise overall performance.

Spatial coherence is evaluated on the final mosaics tile by tile. Edge density measures how many metres of deciduous–evergreen boundary lie within a tile, normalised by tile area (km of edge per km$^2$). Connected patch density uses an 8-neighbour definition (horizontal, vertical, and diagonal adjacency) to identify every contiguous cluster of forest pixels that share the same class. We label the binary masks for deciduous and evergreen pixels with a \(3\times3\) structure matrix of ones, yielding component counts \(C_{\text{dec}}\) and \(C_{\text{ever}}\). The corresponding densities are
\begin{equation*}
    D_{\text{dec}} = \frac{100\,C_{\text{dec}}}{A}, \qquad
    D_{\text{ever}} = \frac{100\,C_{\text{ever}}}{A}, \qquad
    D_{\text{tot}} = D_{\text{dec}} + D_{\text{ever}},
\end{equation*}
where \(A\) is the forest area (km$^2$) within the tile. When predictions are smooth, large coherent stands dominate and the component counts remain small, producing low densities; salt-and-pepper artefacts fragment stands into many tiny clusters, inflating \(D_{\text{tot}}\). We also gauge the effect of simple post-processing by applying a \(3 \times 3\) median filter to the harmonic predictions. Tile-level cross-validation predictions yield accuracy and macro-F1 per tile; we convert the embedding-minus-harmonic deltas to $z$-scores relative to the national distribution, categorising tiles into embedding advantage (\(z \ge +1\sigma\)), rough parity (\(|z| < 1\sigma\)), or harmonic advantage (\(z \le -1\sigma\)). Pearson and Spearman correlations between these deltas and contextual covariates underpin the heterogeneity analysis.

\subsection{Ancillary tile context for heterogeneity analysis}
To interpret spatial performance patterns we enrich each tile with topography, climate, soil, and class-composition attributes. Elevation, slope, and aspect statistics (mean, min, max, 90th percentile, standard deviation) are derived from the CGIAR-CSI SRTM v4 digital elevation model at 90\,m resolution \citep{Jarvis2008SRTM}. We summarise ERA5-Land monthly aggregated reanalysis for 2018, 2020, 2022, and 2023 \citep{MunozSabater2021ERA5Land}, computing annual means and extremes of 2\,m temperature, dew point, surface pressure, soil moisture, and total precipitation over each tile. Soil properties combine the OpenLandMap USDA texture mode at 250\,m \citep{Hengl2021OpenLandMap} with SoilGrids 2.0 mean and standard deviation of clay, sand, silt, bulk density, and soil organic carbon for the 0–5\,cm layer \citep{Poggio2021SoilGrids}. Finally, we aggregate the label parquet to count deciduous and evergreen pixels per tile, derive their ratios, and compute the Shannon diversity index over the two classes. The resulting dataset contains 86 covariates that underpin the tile-level heterogeneity analysis presented in the Results.

\subsection{Temporal stability evaluation}
Temporal transfer is evaluated by freezing the 2023 EMB-14 random forest and applying it to the AlphaEarth embedding vintages released for 2022, 2020, and 2018. We reload the cross-validation manifest so that each labelled pixel keeps the fold identifier it had during training. This alignment serves two purposes: (i) it guarantees that the cross-year evaluations use exactly the same supervision set and eco-region weights as the cross-validation runs, and (ii) it allows per-fold comparisons between the original out-of-fold predictions and the cross-year inference.

For every evaluated year we emit the full metric bundle used in training: confusion matrices aggregated over all folds, fold-level tables, eco-region summaries, per-pixel probabilities, and per-tile metrics. Intersection-over-Union (IoU) is derived from the confusion counts for each class \(c \in \{\text{dec}, \text{ever}\}\) as
\begin{equation*}
    \mathrm{IoU}_c = \frac{\mathrm{TP}_c}{\mathrm{TP}_c + \mathrm{FP}_c + \mathrm{FN}_c},
\end{equation*}
with \(\mathrm{TP}_c\), \(\mathrm{FP}_c\), and \(\mathrm{FN}_c\) defined relative to the reference labels (evergreen is the positive class). Macro-IoU averages the two classes with equal weight. IoU quantifies the overlap between the predicted evergreen/deciduous sets and their reference counterparts, complementing the precision/recall perspective. Alongside IoU we report the same accuracy and F1 statistics as for cross-validation; deltas are computed both against the 2023 inference baseline and the 2023 cross-validation reference to characterise year-to-year drift (Supplementary Tables~S4–S5). All artefacts are stored under \texttt{results/evaluation/embeddings\_<year>/}.

\subsection{Embedding–harmonic similarity analysis}
To probe the relationship between embeddings and explicit phenology we regress each embedding dimension onto the full set of harmonic descriptors using ridge regression. For a given eco-region and embedding channel \(k\), we assemble a response vector \(\mathbf{y}^{(k)} \in \mathbb{R}^n\) with one entry per pixel (standardised to zero mean and unit variance) and a design matrix \(\mathbf{X} \in \mathbb{R}^{n \times p}\) containing the \(p\) harmonic descriptors (also z-scored). Ridge regression estimates coefficients \(\hat{\boldsymbol{\beta}}^{(k)}\) by solving
\begin{equation*}
    \hat{\boldsymbol{\beta}}^{(k)} = \arg\min_{\boldsymbol{\beta}} \left\| \mathbf{y}^{(k)} - \mathbf{X}\boldsymbol{\beta} \right\|_2^2 + \lambda \|\boldsymbol{\beta}\|_2^2,
\end{equation*}
where the regularisation strength \(\lambda\) controls the amount of shrinkage applied to the coefficients. We select \(\lambda\) through RidgeCV on a logarithmic grid (20 values between \(10^{-4}\) and \(10^{4}\)), using a GroupKFold split that keeps tiles intact so that no tile contributes to both training and validation folds. Because \(\lambda\) is merely a stabilising hyperparameter—and different embedding channels prefer different values—we do not report it explicitly in the Results; instead we focus on what the fitted models reveal (variance explained and dominant harmonic cues).

After fitting, we compute tile-level coefficients of determination
\begin{equation*}
    R^2_{t,k} = 1 - \frac{\|\mathbf{y}^{(k)}_t - \mathbf{X}_t \hat{\boldsymbol{\beta}}^{(k)}\|_2^2}{\|\mathbf{y}^{(k)}_t - \bar{y}^{(k)}_t\|_2^2},
\end{equation*}
where the subscript \(t\) denotes the subset of pixels belonging to tile \(t\) and \(\bar{y}^{(k)}_t\) their mean. These $R^2$ values quantify how much of the embedding variance the harmonic basis can reconstruct; negative values indicate that the linear model performs worse than simply predicting the mean. We summarise \(R^2_{t,k}\) across tiles using eco-region weights and across embedding channels by averaging over the 14 dimensions. The reported “mean $R^2$” in Table~\ref{tab:similarity_summary} is therefore the eco-region–weighted average over both tiles and embedding channels.

To interpret the fitted coefficients we normalise \(\hat{\boldsymbol{\beta}}^{(k)}\) by their $\ell_1$ norm and record the harmonic descriptor with the largest absolute weight; we also compute Pearson correlations between \(\mathbf{y}^{(k)}\) and each harmonic feature, retaining the largest-magnitude correlation as a complementary cue. Detailed coefficient tables, $R^2$ distributions, and tile-level diagnostics appear in Supplementary Section~S3.

\section{Results}
\label{sec:results}
We present four groups of findings: national accuracy and calibration, eco-regional behaviour, spatial/ancillary heterogeneity, and cross-year robustness, before comparing the maps with existing national products.
\subsection{H1 -- Accuracy and calibration}
To address H1 we compare cross-validated metrics for the harmonic and embedding feature sets under identical folds and models. The harmonic subset \textbf{HARM-14} groups amplitudes, phases, offsets, and residual variances from the four Sentinel-2 indices described in Table~\ref{tab:harmonic14}, while the embedding subset \textbf{EMB-14} retains the 14 AlphaEarth channels that survived the same recursive elimination.

The embedding baseline outperforms harmonics by \(2.3\) percentage points nationally, reaching \(0.926 \pm 0.0059\) accuracy (interquartile range 0.922–0.932) with the same Random Forest architecture. Harmonic features remain competitive but plateau at \(0.904 \pm 0.0059\) accuracy (IQR 0.900–0.903) and a macro-F1 of \(0.874 \pm 0.0099\) (IQR 0.867–0.879). Embeddings also reduce fold-to-fold variance, shrinking the macro-F1 standard deviation from 0.0099 to 0.0027. Table~\ref{tab:cv_summary} summarises these fold distributions.
\begin{table}[H]
    \centering
    \small
    \begin{tabular}{lcc}
        \toprule
        \textbf{Metric} & \textbf{HARM-14} & \textbf{EMB-14} \\
        \midrule
        Overall accuracy & \(0.904 \pm 0.0059\) [0.900, 0.903] & \(0.926 \pm 0.0059\) [0.922, 0.932] \\
        Macro-F1 & \(0.874 \pm 0.0099\) [0.867, 0.879] & \(0.905 \pm 0.0027\) [0.905, 0.906] \\
        Weighted-F1 & \(0.905 \pm 0.0054\) [0.902, 0.905] & \(0.927 \pm 0.0056\) [0.924, 0.933] \\
        \bottomrule
    \end{tabular}
    \caption{Five-fold cross-validation summary (\( \mu \pm \sigma\)) with interquartile ranges [Q1, Q3]. Embeddings improve all metrics while reducing variance.}
    \label{tab:cv_summary}
\end{table}

These national gains vary meaningfully across eco-regions. Table~\ref{tab:regional_performance} reveals that EMB-14 delivers the largest macro-F1 improvements in Atlantic mosaics (Greater Crystalline and Oceanic West: +6.8\,pp; Oceanic Southwest: +6.5\,pp) and mountainous regions (Pyrenees: +10.8\,pp; Corsica: +8.8\,pp), where mixed stands and steep gradients challenge purely temporal descriptors.
\begin{table}[H]
    \centering
    \small
    \begin{tabular}{lrrrrrr}
        \toprule
        \textbf{GRECO region} & $\mathrm{OA}_{\mathrm{H}}$ & $\mathrm{OA}_{\mathrm{E}}$ & $\Delta\mathrm{OA}$ (pp) & $\mathrm{F1}_{\mathrm{H}}$ & $\mathrm{F1}_{\mathrm{E}}$ & $\Delta\mathrm{F1}$ (pp) \\
        \midrule
        France (weighted CV) & 0.904 & \textbf{0.926} & \textbf{+2.3} & 0.874 & \textbf{0.905} & \textbf{+3.1} \\
        Central Massif (C) & 0.895 & 0.928 & +3.3 & 0.887 & 0.923 & +3.6 \\
        Oceanic Southwest (F) & 0.925 & 0.951 & +2.6 & 0.867 & 0.933 & +6.5 \\
        Semi-Oceanic North Centre (B) & 0.937 & 0.956 & +1.9 & 0.856 & 0.899 & +4.3 \\
        Alps (H) & 0.869 & 0.895 & +2.6 & 0.854 & 0.877 & +2.2 \\
        Vosges (D) & 0.880 & 0.907 & +2.7 & 0.840 & 0.875 & +3.5 \\
        Greater Crystalline and Oceanic West (A) & 0.873 & 0.923 & +5.0 & 0.815 & 0.883 & +6.8 \\
        Greater Semi-Continental East (C$^\prime$) & 0.952 & 0.960 & +0.9 & 0.810 & 0.851 & +4.1 \\
        Mediterranean (J) & 0.811 & 0.818 & +0.7 & 0.791 & 0.798 & +0.7 \\
        Jura (E) & 0.882 & 0.921 & +3.8 & 0.775 & 0.794 & +1.9 \\
        Pyrenees (I) & 0.930 & 0.957 & +2.7 & 0.775 & 0.883 & +10.8 \\
        Corsica (K) & 0.654 & 0.675 & +2.1 & 0.554 & 0.642 & +8.8 \\
        \bottomrule
    \end{tabular}
    \caption{Cross-validated overall accuracy (OA) and macro-F1 for the harmonic (H) and embedding (E) feature sets across GRECO eco-regions. Deltas are embedding minus harmonic in percentage points. All folds use identical eco-region weights and tile assignments.}
    \label{tab:regional_performance}
\end{table}

Beyond raw accuracy improvements, embeddings demonstrate superior calibration quality. The expected calibration error drops from 0.0586 (HARM-14) to 0.0335, and the maximum calibration error halves from 0.184 to 0.114. This calibration stability translates to operational robustness: macro-F1 remains steady when the decision threshold shifts within \([0.45,0.55]\), varying by only 0.003 for embeddings versus 0.009 for harmonics. These characteristics simplify deployment, particularly when analysts need to adjust thresholds for high-variance deciduous regions.

\subsection{Regional Performance Patterns}

Performance varies meaningfully across eco-regions (recall Table~\ref{tab:greco_summary} for ecological context), reflecting forest composition and phenological complexity. Table~\ref{tab:regional_performance} shows that EMB-14 delivers the largest macro-F1 gains in Atlantic mosaics (Greater Crystalline and Oceanic West: +6.8\,pp; Oceanic Southwest: +6.5\,pp) and mountainous regions (Pyrenees: +10.8\,pp; Corsica: +8.8\,pp), where mixed stands and steep gradients challenge purely temporal descriptors. These uplifts mirror the GRECO descriptions (Table~\ref{tab:greco_summary}): the Armorican bocage (Region A) alternates humid oak–chestnut forests with post-war conifer plantations, while the Pyrenean chain (Region I) juxtaposes deciduous foothills with evergreen montane belts within a few hundred metres of elevation. Central Massif and the semi-oceanic North (Regions C and B) each gain about 3–4\,pp, consistent with their heterogeneous oak–hornbeam mosaics and scattered Scots pine estates. Mediterranean forests remain the hardest case: both models converge around \(0.80\) macro-F1 with a slim +0.7\,pp uplift, reflecting the drought-adapted evergreen broadleaf stands of Region J where seasonal amplitude is inherently low.

\begin{figure}[H]
    \centering
    \includegraphics[width=\textwidth]{images/France_Map_emb.png}
    \caption{France 2023 deciduous--evergreen map from the embedding model (EMB-14). Orange denotes deciduous, cyan denotes evergreen; the inset shows Corsica. Spatial patterns match major ecological gradients: evergreen dominance along Mediterranean and Atlantic pine regions (Landes) and at higher elevations (Alps, Jura, Vosges), with deciduous prevalence across lowland temperate belts. Compared to hand-designed features, embeddings yield smoother, more coherent patches while preserving sharp transitions.}
    \label{fig:national_map}
\end{figure}

\subsection{H2 -- Spatial coherence and landscape context}
Hypothesis H2 examines whether the embedding representation yields smoother outputs and whether these gains relate to ancillary environmental gradients. Using the densities defined in Section~\ref{sec:metrics}, national edge density drops from 6.69 to 2.76\,km\,km\(^{-2}\) when switching from HARM-14 to EMB-14, a 59\,\% reduction. The median total patch density \(D_{\text{tot}}\) falls from 6.81\,k to 2.16\,k components per 100\,km² (−68\,%), indicating far fewer tiny clusters. The largest smoothing occurs in maritime pine mosaics (Greater Crystalline and Oceanic West), where the harmonic map fragments mixed stands into speckled artefacts. EMB-14 also maintains crisp elevational boundaries in the Alps, Jura, and Vosges, while matching National Forest Inventory proportions (64.4\,\% deciduous, 35.6\,\% evergreen; Figure~\ref{fig:national_map}). Table~\ref{tab:coherence_summary} summarises the national reduction in edge and patch densities, with supplementary Table~S3 providing the full statistics including the median-filter baseline.

Figure~\ref{fig:h2_multiscale} visualises these differences across five representative tiles. Each row pairs a Sentinel-2 chip (left) with the embedding and harmonic predictions (middle and right). The rows cover GRECO letters G (Greater Semi-Continental East), C (Corsica), A (Alps), O (Oceanic Southwest), and the Central Massif. Embeddings preserve large deciduous blocks (amber, \#e3712c) and evergreen belts (azure, \#2693c1) with minimal speckle; the harmonic counterpart produces many 1–2 pixel patches, consistent with its higher \(D_{\text{tot}}\). Additional summaries, including the effect of median filtering, appear in Supplementary Table~S3 and `results/analysis_coherence/coherence_summary.csv`.

\begin{figure}[H]
    \centering
    \includegraphics[width=\textwidth]{images/figure4_h2_panel.png}
    \caption{Figure~4 (H2): multi-scale comparison between Sentinel-2 imagery and model outputs across five representative GRECO tiles (rows). Columns show the 2023 Sentinel-2 composite, EMB-14 predictions, and HARM-14 predictions. Legends use the manuscript colour palette (deciduous \#e3712c, evergreen \#2693c1). Embeddings produce smoother deciduous estates in Atlantic mosaics (rows G and O), maintain sharp evergreen belts in Corsica (row C) and the Alps (row A), and avoid the speckled artefacts visible in the harmonic baseline.}
    \label{fig:h2_multiscale}
\end{figure}

\begin{table}[H]
    \centering
    \small
    \begin{tabular}{lccc}
        \toprule
        \textbf{Metric} & \textbf{HARM-14} & \textbf{EMB-14} & \textbf{Relative change} \\
        \midrule
        Median edge density (km\,km\(^{-2}\)) & 6.69 & 2.76 & −59\,\% \\
        Median patch density (per 100\,km\(^2\)) & 6\,810 & 2\,159 & −68\,\% \\
        \bottomrule
    \end{tabular}
    \caption{National spatial coherence metrics aggregated over the 639 evaluation tiles.}
    \label{tab:coherence_summary}
\end{table}

\subsubsection{Tile-level heterogeneity and ancillary drivers}

Tile-level analysis confirms that most areas behave consistently across feature sets. The GRECO-balanced grid contains 639 tiles; 612 satisfy the label-density requirement and enter the tile-level evaluation. Within that subset, 83\,\% (510 tiles) fall within \(|z| < 1\sigma\) for accuracy differences, 14\,\% (89 tiles) exhibit a marked embedding advantage with mean Δaccuracy +12.6\,pp, and only 2\,\% (13 tiles) favour harmonics (mean −13.0\,pp). Embedding wins concentrate in Atlantic and western eco-regions where clay-rich soils, high deciduous ratios, and cooler/wetter ERA5 conditions prevail; correlations reach \(r=0.17\) with clay fraction, \(r=0.13\) with deciduous share, and \(r=-0.10\) with the ERA5 temperature range. Greater Crystalline and Oceanic West displays the highest share of advantage tiles (41\,\%) followed by Vosges (31\,\%) and Corsica (33\,\%), while the Mediterranean basin records the largest proportion of harmonic-favoured tiles (12\,\%). Table~\ref{tab:tile_buckets} summarises the distribution, and eco-region summaries (Supplementary Figure~S7) show that Greater Crystalline and Oceanic West gains \(+9.3\)\,pp average accuracy whereas the Mediterranean averages \(+0.9\)\,pp, reinforcing the idea that spatial context helps most in humid, mixed-stand mosaics and contributes modestly in evergreen-dominated shrublands.

The ancillary dataset enables a focused follow-up on the 89 embedding-advantaged tiles: we quantify which covariates (soil texture, moisture regime, diversity indices) best explain the observed lifts.

\begin{figure}[H]
    \centering
    \includegraphics[width=\textwidth]{images/embedding_harmonic_driver_dual.png}
    \caption{Relative covariate shifts for tiles with strong embedding gains (top row, z ≥ +1σ) and harmonic gains (bottom row, z ≤ −1σ) compared with parity tiles. Left panels show percent differences versus parity mean; right panels report Spearman correlations with Δaccuracy. Key insight: Embedding advantages cluster in wetter, more diverse mosaics (rainfall +2.1\%, Shannon diversity +24\%, narrower temp range −1.4\%), whereas harmonic wins emerge in dry, evergreen-dominated tiles (rainfall −10\%, deciduous share −49\%, wider temp range +2.2\%). This environmental contrast explains the regional performance patterns in Table~\ref{tab:regional_performance}.}
    \label{fig:driver_deltas}
\end{figure}

Figure~\ref{fig:driver_deltas} confirms that the embedding subset sits in wetter-than-average tiles (ERA5 rainfall +2.1\,\%), with higher canopy diversity (+24\,\% Shannon index) and only a weak reduction in deciduous share (−2\,\%). It also shows a slightly narrower annual temperature range (−1.4\,\%; \(\rho = -0.16\)), consistent with smoother seasonal dynamics. Harmonics achieve their rare wins in markedly drier pixels (−10\,\% rainfall) where evergreen dominance is much stronger (−49\,\% deciduous share, \(\rho = +0.22\)) and seasonal temperature swings are amplified (+2.2\,\%; \(\rho = +0.47\)). These shifts emphasise that embedding advantages stem from wetter, heterogeneous mosaics, whereas harmonics perform best in dry evergreen stands with pronounced seasonal forcing.

\begin{table}[H]
    \centering
    \small
    \begin{tabular}{lccc}
        \toprule
        \textbf{Tile bucket} & \textbf{Count} & \textbf{Share (\%)} & \textbf{Mean Δaccuracy (pp)} \\
        \midrule
        Embedding advantage (\(z \ge +1\sigma\)) & 89 & 14.5 & +12.6 \\
        Harmonic win (\(z \le -1\sigma\)) & 13 & 2.1 & −13.0 \\
        Rough parity (\(|z| < 1\sigma\)) & 510 & 83.3 & +1.4 \\
        \bottomrule
    \end{tabular}
    \caption{Distribution of tile-level performance buckets derived from cross-validated predictions (612 tiles with sufficient training labels). Positive values favour embeddings.}
    \label{tab:tile_buckets}
\end{table}

\subsection{H3 -- Temporal robustness and legacy alignment}
Hypothesis H3 evaluates whether the 2023 random forest remains stable across years while maintaining alignment with existing national products.

Applying the 2023 EMB-14 model to earlier AlphaEarth embeddings yields accuracies of 0.940 (2018), 0.938 (2020), and 0.935 (2022). These values sit within 1.5 standard deviations of the 2023 cross-validation mean (0.926) and trail the 2023 inference baseline by 3.8–5.4\,pp in macro-F1. The largest year-on-year degradation occurs in Corsica (Region K, −0.16\,pp macro-F1 in 2018) and along the Mediterranean coast (≈−0.07\,pp), consistent with wildfire disturbances and drought stress recorded in the AlphaEarth provenance. Continental regions—Vosges, Jura, Central Massif—remain within −0.05\,pp. These results indicate that the frozen Random Forest generalises well across annual embeddings, with performance largely governed by the quality of the embedding vintage (Supplementary Tables~S4–S5). Table~\ref{tab:temporal_stability} summarises the year-by-year metrics. Across the ten-tile cohort used for Figures~\ref{fig:h2_multiscale} and \ref{fig:h3_temporal}, the macro intersection-over-union (IoU) between the 2023 map and earlier vintages ranges from 0.786 (2018) to 0.801 (2022), with deciduous IoU consistently above 0.83—confirming that inter-annual drift is modest even in spatially heterogeneous sites.

Figure~\ref{fig:h3_temporal} illustrates the temporal behaviour for three representative tiles (G, O, C). Each column shows the Sentinel-2 composite for a given year (top row) alongside the corresponding EMB-14 prediction (bottom row). Colour conventions match Figure~\ref{fig:h2_multiscale}. Despite varying illumination and disturbance cues, the embedding model preserves deciduous and evergreen patches across years, with the largest deviations occurring in the Corsican tile (row C) where post-fire succession alters the canopy mix between 2018 and 2020.

\begin{figure}[H]
    \centering
    \includegraphics[width=\textwidth]{images/figure5_h3_panel.png}
    \caption{Figure~5 (H3): temporal stability across three GRECO tiles (rows G, O, C) observed in 2018, 2020, 2022, and 2023 (columns). For each year the top panel shows the Sentinel-2 composite and the bottom panel shows the EMB-14 prediction using the manuscript palette (deciduous \#e3712c, evergreen \#2693c1). Embeddings maintain coherent deciduous blocks through drought years and capture the recovery of evergreen stands after Corsican fires.}
    \label{fig:h3_temporal}
\end{figure}

\begin{table}[H]
    \centering
    \small
    \begin{tabular}{lccccc}
        \toprule
        \textbf{Year} & \textbf{Source} & \textbf{Accuracy} & \textbf{Macro-F1} & \textbf{ΔF1 vs 2023 (pp)} & \textbf{IoU (macro)} \\
        \midrule
        2018 & Inference & 0.940 & 0.912 & −3.8 & 0.786 \\
        2020 & Inference & 0.938 & 0.917 & −4.4 & 0.796 \\
        2022 & Inference & 0.935 & 0.905 & −4.3 & 0.801 \\
        2023 & Cross-validation & 0.927 & 0.897 & −5.5 & -- \\
        2023 & Inference baseline & 0.969 & 0.951 & 0.0 & 1.000 \\
        \bottomrule
    \end{tabular}
    \caption{Temporal evaluation of the 2023 random forest applied to AlphaEarth embeddings from different years. All metrics use the identical inference pipeline; deltas are expressed in percentage points relative to the 2023 inference baseline. Macro IoU compares each vintage against the 2023 inference map over the ten tiles used in Figures~\ref{fig:h2_multiscale} and \ref{fig:h3_temporal}.}
    \label{tab:temporal_stability}
\end{table}

\subsubsection{Legacy products}

When compared against Copernicus DLT (broadleaf vs conifer) and BD Forêt V2 (deciduous vs evergreen polygons), the embedding map matches the harmonic baseline nationally: overall accuracy is 0.628 vs DLT and 0.638 vs BD Forêt, with macro-F1 differences below 0.4 percentage points (Table~\ref{tab:product_comparison_national}). Regional deltas follow the expected pattern—larger disagreements along recently disturbed Atlantic stands and Mediterranean shrublands where reference products either lag in time (BD Forêt) or do not separate evergreen broadleaf (DLT). Embeddings therefore improve internal consistency without breaking compatibility with incumbent datasets (Supplementary Section~S9).

\begin{table}[H]
    \centering
    \small
    \caption{National agreement with existing products (forest pixels only). Metrics derived from aggregated confusion counts over the eco-region grid.}
    \begin{tabular}{lccc}
        \toprule
        \textbf{Comparison} & \textbf{Overall accuracy} & \textbf{Cohen's $\kappa$} & \textbf{Macro-F1} \\
        \midrule
        HARM-14 vs DLT & 0.627 & 0.17 & 0.578 \\
        EMB-14 vs DLT & 0.628 & 0.16 & 0.574 \\
        HARM-14 vs BD Forêt & 0.639 & 0.17 & 0.585 \\
        EMB-14 vs BD Forêt & 0.638 & 0.16 & 0.582 \\
        \bottomrule
    \end{tabular}
    \label{tab:product_comparison_national}
\end{table}

Qualitatively, embeddings reduce speckle in mixed mosaics such as the Landes maritime pine plantations and Corsican Castagniccia chestnut groves, while DLT and BD Forêt retain their strengths in planted conifer estates and mature deciduous stands (Figure~\ref{fig:comparison_products}). These products remain complementary: our annual map reflects the current canopy state, whereas DLT and BD Forêt provide longer-term typologies.

\begin{figure}[H]
    \centering
    \includegraphics[width=\textwidth]{images/Comparison_dlt_bdforet_harmonic_embedding.png}
    \caption{Qualitative comparison across three sites (rows) and five sources (columns: reference Sentinel-2 image, HARM-14, EMB-14, BD Forêt V2, Copernicus DLT). Orange denotes deciduous (broadleaf for DLT); cyan denotes evergreen (conifer for DLT). Landes (local scale) illustrates plantation edges, Corsica--Castagniccia (medium scale) highlights deciduous chestnut versus evergreen oak, and Fontainebleau (regional scale) contrasts managed temperate forests.}
    \label{fig:comparison_products}
\end{figure}

Figure~\ref{fig:comparison_products} highlights a consistent behavior: embeddings encode neighborhood context, yielding smoother, more coherent class patches without isolated pixels, while harmonic features provide crisp pixelwise decisions that can appear speckled at fine scales. Simple post‑processing (e.g., median filtering) narrows this gap for harmonics, but embedding‑based inference attains similar regularization intrinsically.

\subsubsection{Embedding–harmonic alignment}

To interpret what the embeddings retain from traditional descriptors we projected each embedding dimension onto the harmonic feature space using ridge regression. Across all GRECO regions the mean out-of-sample \(R^2\) remains negative (from −0.31 in the Alps to −0.90 in the Pyrenees), confirming that no linear combination of harmonic descriptors can reconstruct the embedding activations. In other words, the pre-trained representation captures information that is orthogonal to the explicit sinusoidal components.

Although global reconstruction fails, the normalized ridge coefficients reveal which harmonic families the embeddings lean on. Averaged over regions, almost half of the coefficient mass targets spectral offsets—especially the CRSWIR ratio—while first-harmonic amplitudes account for a further 23\,\%, residual variances 16\,\%, and phase terms 11\,\%. The nearest-neighbour analysis shows that 57 out of 154 embedding dimensions align most strongly with CRSWIR offsets, 33 with NBR first-harmonic amplitudes, and 27 with NDVI amplitudes. This echoes the eco-regional behaviour: mountainous regions (Vosges, Jura, Central Massif) emphasise structural moisture cues (CRSWIR offsets), Atlantic mosaics emphasise moisture-driven amplitude contrasts (NBR amplitude), and Mediterranean/Corsican regions highlight NDVI amplitude differences between evergreen maquis and deciduous stands.

\begin{table}[H]
    \centering
    \small
    \begin{tabular}{lcc}
        \toprule
        \textbf{Region (GRECO)} & \textbf{Mean $R^2$} & \textbf{Dominant harmonic cue} \\
        \midrule
        Alps (H) & −0.31 & NDVI amplitude (seasonal vigor) \\
        Central Massif (C) & −0.32 & CRSWIR offset (baseline moisture) \\
        Corsica (K) & −0.72 & CRSWIR offset (baseline moisture) \\
        Grand Ouest (A) & −0.35 & NBR amplitude (moisture contrast) \\
        Grand Est (C$^\prime$) & −0.56 & NBR offset (woody baseline) \\
        Jura (E) & −0.72 & CRSWIR offset (baseline moisture) \\
        Mediterranean (J) & −0.49 & NBR amplitude (moisture contrast) \\
        Sud-Ouest (F) & −0.87 & NDVI amplitude (seasonal vigor) \\
        Pyrenees (I) & −0.90 & CRSWIR offset (baseline moisture) \\
        Semi-Oceanic North (B) & −0.33 & NBR amplitude (moisture contrast) \\
        Vosges (D) & −0.06 & CRSWIR offset (baseline moisture) \\
        \bottomrule
    \end{tabular}
    \caption{Regional embedding–harmonic alignment. Mean out-of-sample \(R^2\) remains negative, but each region shows a dominant harmonic cue highlighted by the ridge projections.}
    \label{tab:similarity_summary}
\end{table}

Table~\ref{tab:similarity_summary} reports the eco-region–weighted mean of \(R^2_{t,k}\) across tiles and all 14 embedding dimensions, together with the harmonic descriptor that carries the largest normalised ridge weight (the same cue usually matches the top Pearson correlation). Most regions exhibit negative mean $R^2$, confirming that no linear combination of harmonics can reconstruct the embeddings wholesale, even though specific dimensions still correlate with interpretable cues such as CRSWIR offsets or NBR amplitudes. This absence of linear reconstruction, combined with the alignment patterns, supports a hybrid interpretation: AlphaEarth embeddings retain phenological axes while injecting spatial context learned during pre-training. That extra context likely underpins the smoother boundaries and higher accuracy gains observed in mixed Atlantic and montane landscapes, where purely temporal descriptors struggle. Per-dimension coefficients and correlation tables appear in Supplementary Section~S3 for readers who need the fine-grained breakdown.

\section{Discussion and Outlook}

AlphaEarth embeddings \citep{AlphaEarth2025} offer a pragmatic baseline for national phenology mapping. Under identical folds and models, EMB-14 improves national accuracy by 2.3 percentage points over HARM-14 and nearly halves the macro-F1 variance across folds (Table~\ref{tab:cv_summary}). More importantly, these gains arrive without the typical trade-offs: no GPU clusters, no fragile hyperparameter tuning, no bespoke sensor calibrations. Embedding-based workflows democratize access to high-resolution features for a variety of Earth observation tasks—including land-cover classification, disturbance detection, and ecosystem mapping—with phenology serving here as a demonstration that resource-limited agencies can deploy state-of-the-art methods without specialized infrastructure or domain-specific feature engineering. The eco-region breakdown (Table~\ref{tab:regional_performance}) shows that the largest gains occur in Atlantic mosaics and mountain systems where deciduous and evergreen stands mix at short range, confirming Hypothesis~H1 and aligning with the GRECO descriptors in Section~\ref{sec:greco}.

Embedding advantages are spatially selective. Table~\ref{tab:tile_buckets} shows that the embedding-advantaged cohort (14.5\,\% of evaluated tiles) sits at least \(+1\sigma\) above the national mean for EMB--HARM accuracy differences, while only 2.1\,\% favour harmonics to the same extent. The ancillary contrasts in Figure~\ref{fig:driver_deltas} clarify why: embedding wins accumulate in wetter, clay-rich mosaics with higher canopy diversity (Shannon index \(+24\,\%\)) and narrower annual temperature swings (\(\rho=-0.16\)), whereas the rare harmonic wins concentrate in dry evergreen estates where deciduous cover falls by almost half and the seasonal temperature range widens by \(+2.2\,\%\) (\(\rho=+0.47\)). These environmental gradients explain the regional pattern—Atlantic and montane eco-regions profit most, Mediterranean shrublands remain challenging—and provide a concrete recipe for hybridisation. In dry evergreen strongholds, retaining a handful of harmonic descriptors or blending in drought-sensitive indices could recover the residual 2\,\% of tiles that still prefer physics-based timing cues. The ridge projections (Table~\ref{tab:similarity_summary}) support this view by showing that embeddings retain interpretable axes (CRSWIR offsets, NBR amplitudes) yet layer spatial context that the harmonic basis misses. Figure~\ref{fig:national_map} and Table~\ref{tab:coherence_summary} then translate that context into markedly smoother maps, underpinning Hypothesis~H2.

Calibration quality underpins operational usability. EMB-14 reaches an expected calibration error of 0.033 versus 0.059 for HARM-14, and maintains macro-F1 stability when the decision threshold moves within the \([0.45,0.55]\) range (Supplementary Figure~S6). Most pixels fall either below 0.1 or above 0.9 evergreen probability, and the empirical accuracy curve tracks the diagonal, showing that the forest's stated confidences are largely honest. Where the coloured confidence markers sit above the black empirical markers the model is mildly over-confident (predictions a few points higher than reality); the opposite would signal under-confidence. These deviations never exceed 0.114 (the maximum calibration error), and the average gap is only 0.033 (the expected calibration error), so users can treat the posterior as a direct proxy for evergreen prevalence. A manager who wishes to avoid false evergreen detections can therefore raise the threshold to, say, 0.55, knowing that the selected pixels truly have a \(\gtrsim 55\%\) evergreen prevalence while the overall macro-F1 remains within 0.903–0.906. The remaining calibration gap concentrates around mid-range probabilities (0.4--0.6), which correspond to mixed stands and boundary pixels where either class remains plausible; at the extremes the coloured and black curves overlap, signalling that the forest separates pure deciduous and evergreen pixels cleanly. Full reliability plots and per-threshold confusion tables are provided in Supplementary Figure~S3 and Section~S6. Combined with the reduced speckle visible in Figure~\ref{fig:comparison_products} and the ancillary insights above, these traits simplify both threshold tuning and regional deployment: humid mixed stands can adopt standard thresholds, while dry evergreen regions can either fall back to harmonic cues or tighten evergreen-specific priors without compromising reliability.

Temporal generalisation (Table~\ref{tab:temporal_stability}) shows that the 2023 random forest transfers to 2018--2022 embeddings with modest macro-F1 degradations (3.8--5.4\,pp) concentrated in fire-affected and drought-prone regions. Compatibility with Copernicus DLT and BD Forêt V2 remains intact nationally (Table~\ref{tab:product_comparison_national}) while exposing the expected regional discrepancies where reference products are outdated or lack evergreen broadleaf separation. The qualitative comparison in Figure~\ref{fig:comparison_products} exemplifies these differences and supports Hypothesis~H3.

Limitations stem from the binary label space and from eco-regions with low seasonal contrast. Mediterranean evergreen broadleaf stands, Corsican maquis, and recently disturbed Atlantic tiles contribute most residual errors. Extending to genus-level classification (Supplementary Section~S10), embeddings achieve overall accuracy 0.828 and macro-F1 0.387, demonstrating that the 64-D representation encodes taxonomic cues beyond phenology—though label sparsity limits tail-class performance. The ancillary catalog established for the heterogeneity study invites two concrete extensions: (i) injecting a small set of climatic and soil covariates into the classifier to better separate dry evergreen shrublands, and (ii) using the tile-level diagnostics as a routing signal in hybrid ensembles that switch to harmonic descriptors only where seasonal forcing dominates.

\subsection{Phenology-informed disturbance detection}
Time-series change detectors such as CCDC, BEAST, BFAST, and LandTrendr \citep{Zhu2014,Zhao2019,Verbesselt2010a,Verbesselt2010b,Kennedy2010,Kennedy2018} benefit from accurate priors on seasonal variance. The calibrated probabilities and reduced speckle achieved by EMB-14 (Table~\ref{tab:coherence_summary}; Supplementary Figure~S6) enable class-conditioned thresholds: deciduous pixels can tolerate larger residual excursions while evergreen pixels can trigger alerts at lower magnitudes without inflating false positives. The smoother patches in Figure~\ref{fig:comparison_products} also reduce isolated outliers that commonly confound disturbance filters, making the annual phenology layer a practical companion to seasonal decomposition pipelines.

\subsection{Transferability and scalability considerations}
The inference pipeline scales because AlphaEarth embeddings are pre-computed on Earth Engine and the random forest remains lightweight. Cross-year evaluations (Table~\ref{tab:temporal_stability}) demonstrate that a single model covers contrasting vintages, while the ancillary correlations suggest that remaining errors originate from genuine ecological change rather than numerical drift. Related AlphaEarth applications report similar robustness for burned-area mapping with Siamese U-Net architectures \citep{Seydi2025AlphaEarthBurnedArea} and for cross-border physiognomic mapping \citep{Houriez2025AEFDataGen}. These studies, together with the national evidence presented here, indicate that pre-trained embeddings provide a dependable substrate for continental monitoring; the main bottleneck shifts to updating labels and safeguarding eco-region coverage.


\section{Conclusion}

AlphaEarth embeddings close a long-standing gap between labor-intensive feature-engineered workflows and compute-heavy deep networks for national deciduous–evergreen mapping. Under identical folds, labels, and Random Forest classifiers, the 14-D embedding subset (EMB-14) raises overall accuracy from $0.904 \pm 0.0059$ to $0.926 \pm 0.0059$, improves macro-F1 from 0.874 to 0.904, and halves the expected calibration error (0.033 vs 0.059) while cutting edge and patch densities by 59\,\% and 68\,\% (Tables~\ref{tab:cv_summary} and \ref{tab:coherence_summary}). These gains are achieved without manual feature design and demonstrate that pre-computed embeddings can deliver high-quality 10\,m products when paired with lightweight classifiers.

The tile-level analysis highlights where this advantage matters most. Fourteen and a half percent of the 639 GRECO-balanced tiles lie at least \(+1\sigma\) above the national Δaccuracy mean, clustering in humid, clay-rich mosaics with diverse canopies (Table~\ref{tab:tile_buckets} and Figure~\ref{fig:driver_deltas}). Only 2.1\,\% of tiles—dry evergreen strongholds with wide seasonal temperature ranges—still favour harmonics, suggesting that small hybrid additions or targeted drought-sensitive indices could recover the remaining gap.

Operational robustness follows from the same design. A frozen 2023 model transfers to 2018–2022 embeddings with macro-F1 remaining within 5 percentage points and macro IoU ranging from 0.786 to 0.801 (Table~\ref{tab:temporal_stability}), demonstrating strong temporal transferability even across years with contrasting disturbance regimes. Agreement with Copernicus DLT and BD Forêt remains on par with harmonics (Table~\ref{tab:product_comparison_national}). Embeddings therefore improve national consistency without breaking continuity with incumbent products.

The main limitations stem from label coverage and the binary label space. Mediterranean evergreen broadleaf stands, Corsican maquis, and recently disturbed Atlantic tiles remain harder cases, and extending the analysis to genus or species will require denser supervision. Future work should augment the classifier with a small set of ancillary covariates in drought-prone regions, expand multi-class labels tied to national inventories, and test similar embedding baselines in other biomes. Because embeddings are pre-computed and freely distributed, the bottleneck shifts from how to train models to how to curate labels. This inversion transforms forest mapping from a computational problem into a field campaign problem—one that ecological agencies are already equipped to solve.

\section*{Data Availability}
Training data summaries, model configurations, cross-validation splits, and similarity analysis outputs will be made available with the publication. The 2023 10\,m deciduous–evergreen map of France will also be released.

\section*{Code Availability}
Code to reproduce data preparation, feature extraction, model training, evaluation, and similarity analyses will be released upon publication.

\bibliographystyle{Frontiers-Harvard}
\bibliography{phenology}

\end{document}
