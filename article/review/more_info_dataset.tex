\subsection{Building a Balanced Reference Dataset}

Creating training data for 14 million pixels required careful orchestration of multiple sources while maintaining geographic and ecological balance. We combined four complementary datasets:

\textbf{PureForest} \citep{gaydon2024pureforestlargescaleaeriallidar}: Provides ground truth for monospecific forests with over 135,000 patches (50\,m × 50\,m) primarily from southern France. Labels were semi-automatically predicted and expert-verified, requiring no preprocessing.

\textbf{RENECOFOR} \citep{ulrich:hal-03444393}: Part of ICP Forests, this dataset contains permanent monitoring plots (1992–present) with extensive tree measurements. We selected points from 2019–2020 within Sentinel-2 coverage, filtering trees above 15\,m height.

\textbf{Tree Position Calibration}: Precise individual tree locations recorded in 2022 to help \citep{ONF} calibrate GPS positions using airborne LiDAR \citep{IGN_LiDARHD}. Dominant species were determined by counting, excluding dead or dying trees.

\textbf{BD Forêt V2} \citep{IGN2024}: IGN's comprehensive forest inventory covering patches >5,000\,m² verified between 2005–2019. We applied a 100\,m negative buffer to account for potential edge changes and used it to fill regional gaps where ground measurements lacked coverage.

Our sampling strategy partitioned France into 2.5\,km × 2.5\,km tiles, selecting those with minimum 10 reference pixels while maintaining 5\,km separation to reduce spatial autocorrelation. A weighted sampling approach balanced representation across eco-regions, with weights calculated as the ratio of effective forest area fraction to dataset fraction, ensuring Mediterranean and Alpine forests contributed proportionally despite smaller absolute coverage.

The final dataset of 14.1 million pixels achieved remarkable diversity: 15 genera and 30 species, with oak (33\%) and pine (12\%) most common but no other genus exceeding 7\%. While deciduous pixels outnumber evergreen (75\% to 25\%), we addressed this imbalance through eco-region weighted sampling and balanced class weights during training.